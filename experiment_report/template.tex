\documentclass[aps,reprint,superscriptaddress]{revtex4-2}
\usepackage{kotex}
\usepackage[HWP]{dhucs-interword}
\usepackage[dvips]{color}
\usepackage{graphicx}
\usepackage{bm}
\usepackage{amsmath}
\usepackage{tikz}

\begin{document}
\title{응집물질물리실험 예비보고서 \\
\small 실험주제 : STM}

\author{HuiJae-Lee}
\affiliation{Physics Department, Inha University}
\email{hjlee6674@inha.edu}
\date{\today}


\begin{abstract}
 이번 실험에서는 STM(주사 터널링 현미경)의 작동원리와 사용 방법에 대해 알아보고 
 Graphite의 표면을 직접 관찰하며 응용해본다. 또한, STM을 이용한 관찰로 부터 결정구조에 대해
 공부하고 이해하는 것을 목표로 한다.
\end{abstract}

\maketitle

\section[Introduction]{Introduction}
Ricard Fyenman는 1959년 그의 강연 "There's Plenty of Room at the Bottom: 
An Invitation to Enter a New Field of Physics"에서 각각의 
원자를 뚜렷하게 보고 우리가 원하는 방식으로 배열하는 새로운 연구 분야를 
제시했고, 그로부터 20년 후 과학자들은 STM(Scanning Tunneling Microscope)과 
AFM(Atomic Force Microscope)의 개발로 그 목표를 달성하기 시작했다. 
STM은 80년대 초 IBM 연구소 소속 Gerd Binning과 Heinrich Rohrer에 의해 개발되었고 
Binning과 Rohrer는 그 공로로 86년 노벨 물리학상을 수상하였다.
STM은 나노 스케일에서의 과학과 기술을 더욱 높은 수준으로 끌어올렸고 기초 물리학에 대한 
이해 또한 엄청나게 발전시켰다. 그 중에서도 이번 실험에 사용할 STM은 3차원에서 표면 구조를 
직접, 실제로 제공한다.
\section[Experiment]{Experiment}
\subsection{Theory}
\subsubsection{양자 터널링}
STM의 원리를 이해하기 위해서는 양자 터널링 현상에 대해 알아야 한다. 양자 터널링은 양자역학과
고전역학의 뚜렷한 차이점 중 하나로, 입자의 동력학적 거동을 해석하는데 파동함수와 확률을
도입하여 설명한다. 
\begin{figure}[htbp]
  \centering
  \begin{tikzpicture}
    %axis
    \draw[-latex] (-4,0)  -- (4,0) node[shift={(-0.1,-0.3)}] {$x$} ;
    \draw[-latex] (0,-1)  -- (0,3) node[shift={(-0.3,-0.1)}] {$y$} ;
    %coordinate
    \coordinate (A) at (-1.5,0);
		\coordinate (B) at (-1.5,1.6);
		\coordinate (C) at ( 1.5,1.6);
    \coordinate (D) at ( 1.5,0);
    %potential
    \draw[] (A)  -- (B);
    \draw[] (B)  -- (C);
    \draw[] (C)  -- (D);
    %wave function
    \draw[-latex] (-4,1)  -- (-1.5,1) node[left=20,above=5] {\large$\psi_{in}$} ;
    \draw[-latex] (1.5,1) node[right=25,above=5] {\large$\psi_{trans}$}  -- (4,1) ;
    %node
		\node[above=10,right] at (0,1.5) {$V_0$};
		\node[below] at (A) {$-a$};
		\node[below] at (D) {$a$};

  \end{tikzpicture}
  \caption{높이 $V_0$와 두께 $2a$의 퍼텐셜 장벽과 
  왼쪽에서 입사하는 자유 입자 $\psi_{in}$}
  \label{fig:1}
\end{figure}
다음과 같은 퍼텐셜 장벽과 이 장벽에 대해 왼쪽에서 입사하는 자유 입자를 
고려하자(FIG.~\ref{fig:1}). 퍼텐셜 장벽의 너비는 $2a$이고 퍼텐셜의 크기는 $V_0$이다. 
중요한 점은 입사하는 자유 입자가 가진 에너지 $E$가 $V_0$보다 작다는 것이다. 
즉, $E<V_0$이다. 이 때 투과확률 $T$는 다음과 같다.
\begin{align}
  \begin{split}
    T=\frac{1}{1+\left(\frac{k^2+q^2}{2kq}\right)^2\sinh^2qa},\,\,\,\\
    k=\sqrt{\frac{2mE}{\hbar^2}},\,\,\,
    q=\sqrt{\frac{2m(V_0-E)}{\hbar^2}}.
  \end{split}
\end{align}
중요한 사실은 투과확률 $T$가 $0$이 아니라는 것이다. 고전역학적으로 보았을 때 
퍼텐셜 장벽보다 낮은 에너지를 가진 입자는 분명히 장벽을 넘을 수 없고 $a<x$영역에는 입자가
존재할 수 없다. 하지만 양자역학적인 해석에 의하면, 투과확률이 존재한다는 것은 입자가 퍼텐셜
장벽을 넘어 $a<x$영역에 존재할 수 있음을 의미한다. 퍼텐셜 장벽이 굉장히 두꺼워 $1<<a$인 경우,
극한을 취하여 근사하면 투과확률 $T$가 $e^{-2qa}$에 비례하게 된다.
\begin{align}
  T\sim \frac{16k^2q^2}{(k^2+q^2)^2}e^{-2qa}.
\end{align}

\subsubsection{STM}
STM은 전자의 양자 터널링을 이용해 시료의 표면을 연구할 수 있도록 해준다.
먼저, 시료의 표면을 측정하기 위해 STM에 달린 작은 금속 탐침이 표면에 굉장히 근접한다. 
FIG.~\ref{fig:2}에서 볼 수 있듯이, 이 금속 탐침의 끝은 하나의 원자로 되어있다. 

\begin{figure}[htp]
  \centering
  \includegraphics[scale=0.35]{STM.png}
  \caption{(a)는 STM의 대략적인 구조이다. 
  양자 터널링에 의해 전류는 금속 탐침과 물질 사이 진공을 투과하여 흐를 수 있다.
  (b)에서 볼 수 있듯이, 두 전극이 거리를 두고 떨어져 있을 때 두 전극의 파동함수 A와 B는
  진공에서 지수적으로 감소하지만, 가까울 수록 터널링이 많이 일어난다.}
  \label{fig:2}
\end{figure}

시료에 전압을 걸어주었을 때 시료와 금속 탐침 사이에는 터널링 전류(tunneling current)가 
측정된다. 시료의 표면에 흐르는 전자가 표면을 탈출하기 위해 STM의 탐침으로 흐르기 위해서는 
원자가 전자를 속박하는 에너지보다 큰 에너지가 필요하다. 이는 위에서 살펴본 퍼텐셜 장벽을 
투과하는 자유 입자의 상황과 유사하다. 원자의 속박 에너지가 퍼텐셜 장벽처럼 작용하는 것이다. 
하지만 투과확률이 $0$보다 크기 때문에 전자는 속박 에너지보다 작은 에너지를 가지더라도 속박 
에너지를 극복하고 STM의 탐침으로 흐를 확률을 가진다. STM은 이렇게 흐르는 전류를 이용하여 
표면에 대한 측정을 시도한다.이 때 금속 탐침이 근접하는 거리는 탐침과 표면 사이의 저항을 측정 
가능할 만큼이다. 시료에 전류가 흘러야 하므로, 도체 시료를 이용한다. 터널링 전류는 탐침과 시료 
사이 터널링 확률(tunneling probability)에 비례하고 이 확률은 거리에 지수적으로 민감하다. 
WKB 근사로부터, 터널링 확률 $P$가 거리 $z$와 다음의 비례관계에 있음을 알 수 있다.
\begin{align}
   P \propto \exp\left(-2\sqrt{\frac{2m\phi}{\hbar^2}z}\right).
\end{align}
$\phi$는 터널링을 위한 유효 장벽의 높이이다. STM은 확률이 거리에 민감하게 반응하는 만큼
정확하게 표면을 측정할 수 있다.


\subsubsection{Graphite}
\subsection{Experimental Methods}
\subsection{Theory}

\nocite{*}
\bibliography{ref}



%\begin{thebibliography}{9}
%\end{thebibliography}

\vfill
\end{document}