\documentclass[aps,reprint,superscriptaddress,11pt]{revtex4-2}
\usepackage{kotex}
\usepackage[HWP]{dhucs-interword}
\usepackage[dvips]{color}
\usepackage{graphicx}
\usepackage{bm}
\usepackage{amsmath}
\usepackage{tikz}

\begin{document}
\title{응집물질물리실험 예비보고서 \\
\small 실험주제 : STM}

\author{HuiJae-Lee}\email{hjlee6674@inha.edu}
\affiliation{Physics Department, Inha University}

\date{\today}


\begin{abstract}
  이번 실험에서는 STM(주사 터널링 현미경)의 작동원리와 사용 방법에 대해 알아보고 
  Graphite의 표면을 직접 관찰하며 응용해본다. 또한, STM을 이용한 관찰로 부터 결정구조에 대해
  공부하고 이해하는 것을 목표로 한다.
 \end{abstract}
 
 \maketitle
 
 \section[Introduction]{Introduction}
 STM의 기원은 Ricard Fyenman의 1959년 강연 "There's Plenty of Room at the Bottom: 
 An Invitation to Enter a New Field of Physics"에서 찾을 수 있다. 그는 각각의 
 원자를 뚜렷하게 보고 우리가 원하는 방식으로 배열하는 새로운 연구 분야를 
 제시했고, 그로부터 20년 후 과학자들은 STM(Scanning Tunneling Microscope)과 
 AFM(Atomic Force Microscope)의 개발로 그 목표를 달성하기 시작했다. 
 STM은 80년대 초 IBM 연구소 소속 Gerd Binning과 Heinrich Rohrer에 의해 개발되었고 
 Binning과 Rohrer는 그 공로로 86년 노벨 물리학상을 수상하였다.
 STM은 나노 스케일에서의 과학과 기술을 더욱 높은 수준으로 끌어올렸고 기초 물리학에 대한 
 이해 또한 엄청나게 발전시켰다. 그 중에서도 이번 실험에 사용할 STM은 3차원에서 표면 구조를 
 직접, 실제로 제공한다.

\section[Experiment]{Experiment}
\subsection{Theory}
\subsubsection{양자 터널링}

\subsubsection{STM}



\subsubsection{Graphite}

\subsection{Experimental Methods}


\nocite{*}
\bibliography{ref}



%\begin{thebibliography}{9}
%\end{thebibliography}

\vfill
\end{document}