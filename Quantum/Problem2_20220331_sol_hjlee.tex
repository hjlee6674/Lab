%\documentclass[preprint,tightenlines,showpacs,showkeys,floatfix,
%nofootinbib,superscriptaddress,fleqn]{revtex4} 
\documentclass[floatfix,nofootinbib,superscriptaddress,fleqn]{revtex4-2} 
%\documentclass[aps,epsfig,tightlines,fleqn]{revtex4}
\usepackage{kotex}
\usepackage[HWP]{dhucs-interword}
\usepackage[dvips]{color}
\usepackage{graphicx}
\usepackage{bm}
%\usepackage{fancyhdr}
%\usepackage{dcolumn}
\usepackage{defcolor}
\usepackage{amsmath}
\usepackage{amsfonts}
\usepackage{amssymb}
\usepackage{amscd}
\usepackage{amsthm}
\usepackage[utf8]{inputenc}
\usepackage{mathtools}
\usepackage{bbm}
\usepackage{slashed}
%\pagestyle{fancy}

\begin{document}

\title{\Large Quantum Mechanics}
\author{김현철}
\email{hchkim@inha.ac.kr}
\author{Lee Hui-Jae}
\email{hjlee6674@inha.edu}
\affiliation{Hardron Theory Group, Department of Physics, Inha University,
Incheon 22212, Republic of Korea }
\date{2021}

\maketitle

\noindent {\bf Due date:} \textbf{\color{red} April 9, 2022} \\ 
\vspace{2cm}
$\slashed{a}$
\section*{\large Problem Set 2}
\noindent \textbf{Problem 1.}
A constant electric field $\mathcal{E}$ is exerted on a charged linear
harmonic oscillator. 
\begin{itemize}
\item[(1)] Write down the corresponding Schr\"odinger equation. 
\item[(2)] Derive the eigenvalues and eigenvectors of the charged
  linear oscillators under a uniform electric field. 
\item[(3)] Discuss the change in energy levels and physics. 
  eigenstates. 
\end{itemize}
Hint: Use the operator method.

\noindent \textbf{Answer : }
\begin{itemize}
  \item[(1)] A charged particle away from the equilibrium position 
  has the potential energy when it is in the electric field. 
  Let a distance from equilibrium position to a particle is $x$.
  In the constant electric field, 
  the electric potential energy $E_p$ is
  \begin{align}
    E_p = q\mathcal{E}x.
  \end{align} 
  Then, the Hamiltonian of the charged linear
  harmonic oscillator $H$ is
  \begin{align}\label{eq:1-1}
    H = \frac{p^2}{2m}+\frac{1}{2}m\omega^2x^2-q\mathcal{E}x.
  \end{align}
  So, the Schr\"odinger equation is
  \begin{align}\label{eq:1-2}
    -\frac{\hbar}{2m}\frac{\partial^2\psi}{\partial x^2}
    +\frac{1}{2}m\omega^2x^2\psi
    -q\mathcal{E}x\psi = E\psi.
  \end{align}
  \item[(2)] First, suppose that there is no electric field.
  Then the Schr\"odinger equation and 
  the energy are
  \begin{align}
    -\frac{\hbar}{2m}\frac{\partial^2\psi_n}{\partial x^2}
    +\frac{1}{2}m\omega^2x^2\psi_n = E_n\psi_n,\,\,\, 
    E_n = \left(\frac{1}{2}+n\right)\hbar\omega.
  \end{align}
  It is the Schr\"odinger equation of the simple harmonic oscillator. 
  In the algebraic method to solve the equation, 
  we defined the operators
  \begin{align}
      a = \sqrt{\frac{m\omega}{2\hbar}}
      \left(x+i\frac{p}{m\omega}\right),\,\,\,
      a^\dagger = \sqrt{\frac{m\omega}{2\hbar}}
      \left(x-i\frac{p}{m\omega}\right),\,\,\,
      \left[a,a^\dagger\right] = \mathbbm{I},
  \end{align}
  where $\mathbbm{I}$ is the identity operator and
  \begin{align}\label{eq:1-2-1}
    x = \sqrt{\frac{2\hbar}{m\omega}}
    \left(\frac{a+a^\dagger}{2}\right),\,\,\,
    p = \sqrt{2\hbar m\omega}
    \left(\frac{a-a^\dagger}{2i}\right).
  \end{align}
 Therefore, the Hamiltonian $H_0$ can be expressed in terms of
 the ladder operators as
  \begin{align}\label{eq:1-3}
    H_0 = \frac{p^2}{2m}+\frac{1}{2}m\omega^2x^2 
    = \hbar\omega\left(a^\dagger a + \frac{1}{2}\right)
    = \hbar\omega\left(a a^\dagger - \frac{1}{2}\right).
  \end{align}
  Now, recall that there is a constant 
  electric field $\mathcal{E}$. 
  From Eq.~\eqref{eq:1-1}, \eqref{eq:1-2-1} and \eqref{eq:1-3}, 
  Hamiltonian with a constant electric field $H$ is
   \begin{align}\label{eq:1-4}
    H = \hbar\omega\left(a^\dagger a 
    + \frac{1}{2}\right)-q\mathcal{E}x
    = \hbar\omega\left(a^\dagger a 
    + \frac{1}{2}-\frac{q\mathcal{E}}{2\hbar\omega}
    \sqrt{\frac{2\hbar}{m\omega}}
    \left(a+a^\dagger\right)\right)
    =\hbar\omega\left(a^\dagger a 
    + \frac{1}{2}+\kappa\left(a+a^\dagger\right)\right),
   \end{align}  
   where
  \begin{align}\label{eq:1-8}
   \kappa = -\frac{q\mathcal{E}}{2\hbar\omega}
   \sqrt{\frac{2\hbar}{m\omega}}
   =-\frac{q\mathcal{E}}{\omega\sqrt{2\hbar m\omega}}.
  \end{align}
   To eliminate the terms of $a$ and $a^\dagger$, we define the new operator $b$.
   \begin{align}\label{eq:1-5}
    b \coloneqq a + \kappa.
   \end{align}
   $b$ satisfies following commutation relations:
   \begin{align}\label{eq:1-6}
    \begin{split}
      &\left[a,b\right] = a(a+\kappa)-(a+\kappa)a 
      = aa-aa+\kappa a-\kappa a=0, \\
      &\left[b,b^\dagger\right] 
      = \left[(a+\kappa),(a+\kappa)^\dagger\right] 
      = \mathbbm{I}.
    \end{split}
   \end{align}
   The $H$ can be written as 
   \begin{align}\label{eq:1-7}
    \begin{split}
      H &= \hbar\omega\left((a^\dagger+\kappa)(a+\kappa)
      -\kappa(a+a^\dagger)-\kappa^2
      + \frac{1}{2}+\kappa\left(a+a^\dagger\right)\right)  \\
      &=\hbar\omega
      \left(
        b^\dagger b-\kappa^2
      + \frac{1}{2}
      \right).
    \end{split}
   \end{align}
    We can write the eigenvalue equation with the new operator.
    \begin{align}\label{eq:1-11}
      &H \psi^\prime_n = \hbar\omega\left(b^\dagger b
      + \frac{1}{2}-\kappa^2\right)\psi^\prime_n 
      = E^\prime_n \psi^\prime_n.
    \end{align}
    Since $b$ and $b^\dagger$ behave as 
   $a$ and $a^\dagger$, we can treat $b^\dagger b$ as
   the number operator. So, the eigenvalue of the Hamiltonian is given by 
   \begin{align}
     E_n^\prime = \left(n +\frac{1}{2} -\kappa^2\right)\hbar\omega.
   \end{align}
   There is the ground state of $\psi_n^\prime$, that is
    \begin{align}\label{eq:1-14}
      b\psi_0^\prime=
      \left(\sqrt{\frac{m\omega}{2\hbar}}\left(x+\frac{\hbar}{m\omega}
      \frac{d}{dx}\right)+\kappa\right)
      \psi_0^\prime(x)=0.
    \end{align}
    It is ODE of the first order about $x$.
    \begin{align}
      \frac{d\psi_0^\prime}{dx}
      =-\frac{m\omega}{\hbar}
      \left(
        \sqrt{\frac{2\hbar}{m\omega}}
      \kappa^2+x
      \right)
      \psi_0^\prime
      =-\left(
        \sqrt{\frac{2m\omega}{\hbar}}
        \kappa+\frac{m\omega}{\hbar}x
        \right)
        \psi_0^\prime.
    \end{align}
     The solution of this ODE is    
    \begin{align}
      \psi_0^\prime = A\exp{\left(-\left(\frac{m\omega}{2\hbar}x^2
      +\sqrt{\frac{2m\omega}{\hbar}}\kappa x\right)\right)}
      =A\exp{\left(
        -\frac{m\omega}{2\hbar}{\left(
          x +\sqrt{\frac{2\hbar}{m\omega}}\kappa
        \right)}^2+\kappa^2
      \right)}.
    \end{align}
    % Normalization constant $A$ is 
    % \begin{align}
    %   &|A|^2\int\exp{\left(
    %     -\frac{m\omega}{\hbar}{\left(
    %       x +\sqrt{\frac{2\hbar}{m\omega}}\kappa
    %     \right)}^2+2\kappa^2
    %   \right)}\,dx
    %   =|A|^2e^{2\kappa^2}\sqrt{\frac{\pi\hbar}{m\omega}}
    %   =1, \\
    %   &A = \pm \left(\frac{m\omega}{\pi\hbar}\right)
    %   ^{\frac{1}{4}}e^{-\kappa^2}.
    % \end{align}
    Therefore the ground state of $\psi_n^\prime$ is
    \begin{align}
      \psi_0^\prime = A_0
      \exp{\left(-\frac{m\omega}{2\hbar}
      {\left(x +\sqrt{\frac{2\hbar}{m\omega}}\kappa\right)}
        ^2\right)}.
    \end{align}
    As we stated before, $b$ and $b^\dagger$ behave as the ladder operators, 
    so that $\psi_n^\prime$ can be expressed as 
    \begin{align}
      \psi_n^\prime = \frac{1}{\sqrt{n!}}(b^\dagger)^n
      \psi_0^\prime
      =A_n
      \left(
        \sqrt{\frac{m\omega}{2\hbar}}
        \left(x-\frac{\hbar}{m\omega}\frac{d}{dx}\right)
      +\kappa\right)^n
      \exp{\left(-\frac{m\omega}{2\hbar}
      {\left(x +\sqrt{\frac{2\hbar}{m\omega}}\kappa\right)}
        ^2\right)}.
    \end{align}
    $A_n$ is the normalization constant. Substituting $\xi$ as,
    \begin{align}\label{eq:1-16}
      \xi \coloneqq \sqrt{\frac{m\omega}{2\hbar}}x+\kappa,\,\,\,
      dx = \sqrt{\frac{2\hbar}{m\omega}}d\xi.
    \end{align}
    Then,
    \begin{align}
      \psi_n^\prime=A_n
      \left(
      \xi-\frac{1}{2}\frac{d}{d\xi}
      \right)^ne^{-\frac{1}{2}\xi^2}
      =A_n2^{-n}
      H_n(\xi)e^{-\xi^2}.
    \end{align}
    From Eq.\eqref{eq:1-8} and \eqref{eq:1-16},
    \begin{align}
      \xi = \sqrt{\frac{m\omega}{2\hbar}}x
      -\frac{q\mathcal{E}}{2\hbar\omega}
       \sqrt{\frac{2\hbar}{m\omega}}
      =\sqrt{\frac{m\omega}{2\hbar}}\left(x
      -\frac{q\mathcal{E}}{m\omega^2}\right).
    \end{align}
    Finally we obtain the exact form of the $n$th eigenvector $\psi^\prime$.
    \begin{align}\label{eq:1-15}
      \psi_n^\prime=A_n^\prime
      H_n\left(\sqrt{\frac{m\omega}{2\hbar}}\left(x
      -\frac{q\mathcal{E}}{m\omega^2}\right)\right)
      \exp{\left(-\frac{m\omega}{2\hbar}\left(x
      -\frac{q\mathcal{E}}{m\omega^2}\right)^2\right)}.
    \end{align}

  \item[(3)] Let us compare the energy levels 
  and eigenstate of SHO with not charged.
  \begin{align}
    &\psi_n = B_nH_n\left(\sqrt{\frac{m\omega}{2\hbar}}x\right)
    \exp{\left(-\frac{m\omega}{2\hbar}x^2\right)},\,\,\,
    E_n=\left(n+\frac{1}{2}\right)\hbar\omega \\
    &\psi_n^\prime=B^\prime_nH_n\left(\sqrt{\frac{m\omega}{2\hbar}}
    \left(x-\frac{q\mathcal{E}}{m\omega^2}\right)\right)
    \exp{\left(-\frac{m\omega}{2\hbar}\left(x
    -\frac{q\mathcal{E}}{m\omega^2}\right)^2\right)},\,\,\,
    E^\prime_n= \left(\frac{1}{2}+n
    -\frac{q^2\mathcal{E}^2}{2\hbar^2 m^2\omega^3}\right)\hbar\omega.
  \end{align}
  The energy levels with a constant electric field are shifted as 
  $-\frac{q^2\mathcal{E}^2}{2\hbar m^2\omega^2}$ and the eigenvectors
  with a constant electric field are translated as 
  $-\frac{q\mathcal{E}}{m\omega^2}$ without the change of the shape 
  for any $n$.
  \end{itemize}

\vspace{0.5cm}
\newpage
\noindent \textbf{Problem 2.} 
The generating function $S(x,t)$ for the Hermite polynomial $H_n(x)$
is defined as 
\begin{align}
S(x,t) = e^{x^2-(t-x)^2} = e^{-t^2 + 2 t x} 
= \sum_{n=0}^\infty
  \frac{H_n(x)}{n!} t^n.  
\label{eq:1}
\end{align}
\begin{itemize}
\item[(1)] Using this generating function, derive the Hermite
  differential equation. 
\item[(2)] Derive the following formula from Eq.~\eqref{eq:1}:
  \begin{align}\label{eq:2-0-1}
H_n(x) = (-1)^n e^{x^2} \frac{d^n}{dx^n} e^{-x^2}    ,
  \end{align}
which is called the Rodrigues representation of the Hermite
polynomial. 
\item[(3)] Using Eq.~\eqref{eq:1}, derive the orthogonal relation of
  the Hermite polynomials
  \begin{align}\label{eq:2-0-2}
    \int_{-\infty}^\infty e^{-x^2} H_n(x) H_m(x) dx 
    = 2^n \sqrt{\pi}
    n! \delta_{nm}.
  \end{align}
\item[(4)] Prove that
  \begin{align}\label{eq:2-0-3}
    \left(2x-\frac{d}{dx}\right)^n 1 = H_n(x),
  \end{align}
\item[(5)] Prove
  \begin{align}\label{eq:2-0-4}
    \int_{-\infty}^\infty x e^{-x^2} H_n(x) H_m(x) dx 
    = \sqrt{\pi}
    2^{n-1} n!\delta_{m,n-1} 
    + \sqrt{\pi} 2^n (n+1)! \delta_{m,n+1}.
  \end{align}
\item[(6)] Prove
  \begin{align}\label{eq:2}
    \int_{-\infty}^\infty x^2 e^{-x^2} H_n(x) H_n(x) dx = 
\sqrt{\pi} 2^n n! \left(n+\frac12\right).
  \end{align}
\end{itemize}

\noindent \textbf{Answer : }

\begin{itemize}
\item[(1)] The Hermite differential equation is
\begin{align}\label{eq:2-1}
  \frac{d^2y}{dx^2}-2x\frac{dy}{dx}+\lambda y = 0,
\end{align}
where $\lambda$ is a non-negative integer. 
The first and second derivatives of $x$ for generating function $S$ are
\begin{align}\label{eq:2-2}
  \begin{split}
    &\frac{dS}{dx} = 2tS = \sum^\infty_{n=0}
    \frac{H^\prime_n(x)}{n!}t^n  \\
    &\frac{d^2S}{dx^2} = 4t^2S=\sum^\infty_{n=0}
    \frac{H^{\prime\prime}_n(x)}{n!}t^n.
  \end{split}
\end{align}
The first derivative of $t$ is
\begin{align}\label{eq:2-3}
  \frac{dS}{dt} = 2(-t+x)S \Longrightarrow \sum^\infty_{n=0}
  \frac{H_n(x)}{n!}nt^{n-1}
  =-\frac{dS}{dx} +2xS.
\end{align}
Rearranging $S$ in the first and second deriavtive inside Eq.~\eqref{eq:2-2}, we have
\begin{align}\label{eq:2-3-1}
  \begin{split}
    \frac{dS}{dx} &= 
    \frac{1}{2t}\frac{d^2S}{dx^2}
    =\frac{1}{2}\sum^\infty_{n=0}
    \frac{H^{\prime\prime}_n(x)}{n!}t^{n-1} , \\
    2xS &= 2x\frac{1}{2t}\frac{dS}{dt} 
    =x\sum^\infty_{n=0}\frac{H^{\prime}_n(x)}{n!}t^{n-1}.
  \end{split}
\end{align}
Inserting Eq.~\eqref{eq:2-3-1} to Eq.~\eqref{eq:2-3}, we get
\begin{align}
  \sum^\infty_{n=0}\frac{H_n(x)}{n!}nt^{n-1}
  =-\frac{1}{2}\sum^\infty_{n=0}\frac{H^{\prime\prime}_n(x)}{n!}t^{n-1}
  +x\sum^\infty_{n=0}\frac{H^{\prime}_n(x)}{n!}t^{n-1},
\end{align}
or
\begin{align}
  \sum^\infty_{n=0}\left(\frac{H^{\prime\prime}_n(x)
  -2xH^{\prime}_n(x)+2nH_n(x)}{n!}t^{n-1}\right)=0.
\end{align}
It is true for any $t$ when all coefficient is zero. Finally we obtain
the hermite differential equation
\begin{align}\label{eq:2-4}
  H^{\prime\prime}_n(x)
  -2xH^{\prime}_n(x)+2nH_n(x)=0.
\end{align}
\item[(2)] Carrying out Tayler expansion of Eq.~\eqref{eq:1}, we have
\begin{align}
  S(x,t) = e^{x^2}e^{-(t-x)^2} = \left.e^{x^2}\sum^\infty_{n=0}\frac{t^n}{n!}
  \left(\frac{d^n}{dt^n}e^{-(t-x)^2}\right|_{t=0} \right)
  = \sum^\infty_{n=0}\frac{H_n(x)}{n!}t^n.
\end{align}
Since the series representation is unique,
\begin{align}\label{eq:2-5}
  H_n(x)=e^{x^2}\left.\frac{d^n}{dt^n}e^{-(t-x)^2}\right|_{t=0}.
\end{align}
If we regard $t$ as just the parameter, Eq.~\eqref{eq:2-5} is true for 
any $t$. 
A differential part of a LHS is
\begin{align}
  \left.\frac{d^n}{dt^n}e^{-(t-x)^2}\right|_{t=0}
  =\left.(-1)^n\frac{d^n}{dx^n}e^{-(t-x)^2}\right|_{t=0}
  =(-1)^n\frac{d^n}{dx^n}e^{-x^2}
\end{align}
Finally we obtain,
\begin{align}\label{eq:2-5-1}
  H_n(x) = (-1)^ne^{x^2}\frac{d^n}{dx^n}e^{-x^2}.
\end{align}
\end{itemize}
%%---------------------------------------------------------------------------------------
Let us define the new integration $I_{nkp}$.
We use this integration to solve (3), (5) and (6).
\begin{align}\label{eq:2-7-1}
  I_{nkp} = \int^{\infty}_{-\infty}H_n(x)H_k(x)e^{-x^2}x^p\,dx.
  % ,\,\,\,
  % \sum^\infty_{p=0} \sum^\infty_{n=0} \sum^\infty_{k=0}I_{nkp}
  % \frac{s^n}{n!}\frac{t^k}{k!}\frac{(2\lambda)^p}{p!}
  % =\sqrt{\pi}e^{\lambda^2+2(st+\lambda s + \lambda t)}
\end{align}
To calculate this integration, let us make every function 
in the integral into the form of the exponential.
In the case of Hermite polynomial, it can be using the generation function
and $x^p$ can be expressed to the power series of the $e^x$.
\begin{align}\label{eq:2-7-2}
  S(x,t) = \sum_{n=0}^\infty \frac{t^n}{n!}H_n(x)=e^{-t^2+2tx}
  \Longrightarrow \sum_{n=0}^\infty\sum_{k=0}^\infty \frac{t^n}{n!}\frac{s^k}{k!}
  I_{nkp}
  = \int^{\infty}_{-\infty}e^{-t^2+2tx}e^{-s^2+2sx}e^{-x^2}x^p\,dx.
\end{align}
Making the same form of the exponent, consider the $e^{2\lambda x}$. Then,
\begin{align}\label{eq:2-7-3}
  e^{2\lambda x} = \sum_{p=0}^{\infty}\frac{(2\lambda)^p}{p!}x^p
  \Longrightarrow \sum_{n=0}^\infty\sum_{k=0}^\infty\sum_{p=0}^{\infty}
  \frac{t^n}{n!}\frac{s^k}{k!}\frac{(2\lambda)^p}{p!}
  I_{nkp}
  = \int^{\infty}_{-\infty}e^{-t^2+2tx}e^{-s^2+2sx}e^{-x^2+2\lambda x}\,dx.
\end{align}
It is the gaussian integration for $x$.
\begin{align}\label{eq:2-7-4}
  \begin{split}
    \int^{\infty}_{-\infty}e^{-t^2+2tx}e^{-s^2+2sx}e^{-x^2+2\lambda x}\,dx
    &=e^{-t^2-s^2}\int^{\infty}_{-\infty}e^{-x^2+2(t+s+\lambda) x}\,dx \\
    &=e^{-t^2-s^2+(t+s+\lambda)^2}\int^{\infty}_{-\infty}e^{-(x-t-s-\lambda)^2}\,dx \\
    &=\sqrt{\pi}e^{-t^2-s^2+(t+s+\lambda)^2}.
  \end{split}
\end{align}
Therefore we obtain the result of the integration associated with $I_{nkp}$.
Using this integration, let us solve Problem 2 as well as possible.
\begin{align}\label{eq:2-7-5}
  \sum_{n=0}^\infty\sum_{k=0}^\infty\sum_{p=0}^{\infty}
  \frac{t^n}{n!}\frac{s^k}{k!}\frac{(2\lambda)^p}{p!}
  I_{nkp}
  =\sqrt{\pi}e^{-t^2-s^2+(t+s+\lambda)^2}.
\end{align}
%%---------------------------------------------------------------------------------------

\begin{itemize}
\item[(3)] Eq~\eqref{eq:2-0-2} is the case when $p=0$.
\begin{align}
  I_{nk0} = \int^{\infty}_{-\infty}H_n(x)H_k(x)e^{-x^2}\,dx.
\end{align}
From Eq~\eqref{eq:2-7-2},
\begin{align}
  \begin{split}
    \sum_{n=0}^\infty\sum_{k=0}^\infty \frac{t^n}{n!}\frac{s^k}{k!}
    I_{nk0}
    &= \int^{\infty}_{-\infty}e^{-t^2+2tx}e^{-s^2+2sx}e^{-x^2}\,dx
    =e^{-t^2-s^2}\int^\infty_{-\infty}e^{-x^2+2(t+s)x}\,dx  \\
    &=e^{-t^2-s^2+(t+s)^2}\sqrt{\pi}.
  \end{split}
\end{align}
Therefore we obtain
\begin{align}
  \sum_{n=0}^\infty\sum_{k=0}^\infty \frac{t^n}{n!}\frac{s^k}{k!}
    I_{nk0}
    =\sqrt{\pi}e^{2ts}.
\end{align}
$e^{2ts}$ can be expand to the power series.
\begin{align}
  \sum_{n=0}^\infty\sum_{k=0}^\infty \frac{t^n}{n!}\frac{s^k}{k!}
    I_{nk0}
    =\sqrt{\pi}\sum_{m=0}^{\infty}\frac{2^m}{m!}(ts)^m.
\end{align}
It is possible to express the RHS to the product of the series between $t$ and $s$.
\begin{align}
  \sum_{m=0}^{\infty}\frac{2^m}{m!}(ts)^m
  =\left(\sum_{n=0}^{\infty}2^n n!\frac{t^n}{n!}\right)
  \left(\sum_{k=0}^{\infty}\frac{s^k}{k!}\right)\delta_{nk}
  =\sum_{n=0}^\infty\sum_{k=0}^\infty 2^n n!\frac{t^n}{n!}\frac{s^k}{k!}\delta_{nk}.
\end{align}
Hence we get
\begin{align}
  \sum_{n=0}^\infty\sum_{k=0}^\infty \frac{t^n}{n!}\frac{s^k}{k!}
  I_{nk0}
  =\sqrt{\pi}\sum_{n=0}^\infty\sum_{k=0}^\infty 
  2^n n!\frac{t^n}{n!}\frac{s^k}{k!}\delta_{nk}
\end{align}
and $I_{nk0}$ is
\begin{align}
  I_{nk0}
  = \int^{\infty}_{-\infty}H_n(x)H_k(x)e^{-x^2}\,dx
  = \sqrt{\pi}2^n n!\delta_{nk}.
\end{align}
\item[(4)] {\it Proof}. We use the mathematical induction. 
If $n=0$ and $n=1$, then,
\begin{align}
  H_0(x) = 1,\,\,\,H_1(x)=2x. 
\end{align}
The statement is true. Suppose it is true:
\begin{align}
  H_k(x) = \left(2x-\frac{d}{dx}\right)^k 1.
\end{align}
Then,
\begin{align}
  \begin{split}
    H_{k+1}(x) &= \left(2x-\frac{d}{dx}\right)
    \left(2x-\frac{d}{dx}\right)^k 1
    =\left(2x-\frac{d}{dx}\right)H_k(x)
  \end{split}
\end{align}
From Eq.~\eqref{eq:2-5-1},
\begin{align}
  \begin{split}
    \left(2x-\frac{d}{dx}\right)H_k(x)
    &=\left(2x-\frac{d}{dx}\right)(-1)^ke^{x^2}
    \frac{d^k}{dx^k}e^{-x^2}  \\
    &=(-1)^k 2x e^{x^2}\frac{d^k}{dx^k}e^{-x^2}
    -(-1)^k 2x e^{x^2}\frac{d^k}{dx^k}e^{-x^2}
    -(-1)^k e^{x^2}\frac{d^{k+1}}{dx^{k+1}}e^{-x^2}  \\
    &=(-1)^{k+1}= e^{x^2}\frac{d^{k+1}}{dx^{k+1}}e^{-x^2}
    =H_{k+1}(x).
  \end{split}
\end{align}
Hence this statement is true for $n=k+1$. \\
By mathematical induction, 
this statement is true for any $n$. ~\hfill $\square$
\item[(5)] Let us consider the following expression to treat the case when $p \neq 0$.
\begin{align}
  \begin{split}
    &e^{-t^2-s^2+(t+s+\lambda)^2}
    =e^{\lambda^2+2(st+s\lambda+t\lambda)} = \sum_{k,l,q,r}\frac{1}{k!l!q!r!}
    (\lambda^2)^{k}(2st)^{l}(2s\lambda)^{q}(2t\lambda)^{r}  \\
    &\Longrightarrow \sum_{n,m,p}\frac{s^nt^m(2\lambda)^p}{n!m!p!}I_{nmp}
    =\sqrt{\pi}\sum_{k,l,q,r}\frac{2^{l+q+r}}{k!l!q!r!}\lambda^{2k+q+r}s^{l+q}t^{l+r}.
  \end{split}
\end{align}
Comparing the order of each variables $\lambda,s,t$, there are two cases to make $p=1$;
$(k,q,r) = (0, 0, 1)$ or $(k,q,r) = (0, 1, 0)$.
\begin{itemize}
  \item[a. ]$(k,q,r) = (0, 0, 1)$.
  In this case, $n=1$ and $m=l+1$. That is, $m=n+1$.
  \begin{align}
    2\lambda\sum_n \frac{s^n t^{n+1}}{n!(n+1)!}I_{n,n+1,1}
    =\sqrt{\pi}\sum_{n}\frac{2^{n+1}}{n!}\lambda s^n t^{n+1}.
  \end{align}
  Erasing $2\lambda$ on both sides,
  \begin{align}
    \sum_n \frac{s^n t^{n+1}}{n!(n+1)!}I_{n,n+1,1}
    =\sqrt{\pi}\sum_{n}\frac{2^{n}}{n!} s^n t^{n+1}.
  \end{align}
  Therefore, we obtain
  \begin{align}
    I_{n,n+1,1} = (n+1)!2^n\sqrt{\pi}.
  \end{align}
  \item[b. ]$(k,q,r) = (0, 1, 0)$.
  In this case, $n=l+1$ and $m=l$. That is, $m=n-1$.
  \begin{align}
    2\lambda\sum_n \frac{s^n t^{n-1}}{n!(n-1)!}I_{n,n-1,1}
    =\sqrt{\pi}\sum_{n}\frac{2^{n}}{(n-1)!}\lambda s^n t^{n-1}.
  \end{align}
  Erasing $2\lambda$ on both sides,
  \begin{align}
    \sum_n \frac{s^n t^{n-1}}{n!(n-1)!}I_{n,n-1,1}
    =\sqrt{\pi}\sum_{n}\frac{2^{n-1}}{(n-1)!}s^n t^{n-1}.
  \end{align}
  Therefore, we obtain
  \begin{align}
    I_{n,n-1,1} = n!2^{n-1}\sqrt{\pi}.
  \end{align}
\end{itemize}
From these two results, we can write $I_{nk1}$ as following expression.
\begin{align}
  I_{nk1}=\int_{-\infty}^\infty x e^{-x^2} H_n(x) H_k(x) dx 
  = \sqrt{\pi} 2^n (n+1)! \delta_{k,n+1}
  +\sqrt{\pi}
  2^{n-1} n!\delta_{k,n-1}.
\end{align}
\item[(6)] From Eq~\eqref{eq:2-7-1}, $I_{nnp}$ is
\begin{align}
  I_{nnp} = \int^{\infty}_{-\infty}H_n(x)H_n(x)e^{-x^2}x^p\,dx.
\end{align}
Integrating by part, we obtain
\begin{align}
  \begin{split}
    I_{nnp} &= \left.-\frac{1}{2}H_nH_n x^{p-1}e^{-x^2}\right|^{\infty}_{-\infty}
    +\frac{1}{2}\int^{\infty}_{-\infty}2H'_n H_n e^{-x^2}x^{p-1}\,dx
    +\frac{1}{2}\int^{\infty}_{-\infty}H_n H_n (p-1)e^{-x^2}x^{p-2}\,dx \\
    &=\int^{\infty}_{-\infty}H'_n H_n e^{-x^2}x^{p-1}\,dx
    +\frac{1}{2}(p-1)\int^{\infty}_{-\infty}H_n H_n e^{-x^2}x^{p-2}\,dx.
  \end{split}
\end{align}
From Eq~\eqref{eq:2-0-3}, $H'(x) = 2xH_{n}(x)-H_{n+1}(x)$. Hence,
\begin{align}
  \begin{split}
    I_{nnp} &= \int^{\infty}_{-\infty}
    2H_n H_n e^{-x^2}x^{p}\,dx
    - \int^{\infty}_{-\infty}
    H_nH_{n+1}x^{p-1}e^{-x^2}\,dx
    +\frac{1}{2}(p-1)\int^{\infty}_{-\infty}
    H_n H_n e^{-x^2}x^{p-2}\,dx \\
    &= 2I_{n,n,p}-I_{n,n+1,p-1}+\frac{1}{2}(p-1)I_{n,n,p-2}.
  \end{split}
\end{align}
We are treating the case when $p=2$. Therefore,
\begin{align}
  I_{nn2} = 2I_{n,n,2}-I_{n,n+1,1}+\frac{1}{2}I_{n,n,0}
  \Longrightarrow
  I_{n,n,2}=I_{n,n+1,1}-\frac{1}{2}I_{n,n,0}.
\end{align}
From Eq~\eqref{eq:2-0-2} and \eqref{eq:2-0-4}, we can calculate
$I_{n,n+1,1}$ and $I_{n,n,0}$.
\begin{align}
  \begin{split}
    I_{n,n,2} &= \sqrt{\pi}2^n(n+1)!-\frac{1}{2}2^n\sqrt{\pi}n!
    =\sqrt{\pi}2^n n! \left(n+1-\frac{1}{2}\right)  \\
    &=\sqrt{\pi}2^n n!\left(n+\frac{1}{2}\right).
  \end{split}
\end{align}
Finally, we get
\begin{align}
  \int^{\infty}_{-\infty}H_n(x)H_n(x)e^{-x^2}x^2\,dx
  =\sqrt{\pi}2^n n!\left(n+\frac{1}{2}\right).
\end{align}
\end{itemize}

\vspace{0.5cm}

\noindent \textbf{Problem 3.} 
Given the eigenfunctions and eigenenergies of the SHO, 
\begin{itemize}
\item[(1)] Compute the kinetic and potential energies at the $n^{th}$
  level.  Show that the results satisfy the virial theorem. 
\item[(2)] Show that the $n^{th}$ state of the SHO satisfies 
  \begin{align}
    \Delta x \Delta p = \left(n+\frac12\right)\hbar.
  \end{align}
\end{itemize}

\noindent \textbf{Answer : }
\begin{itemize}
  \item[(1)] The eigenvector and eigenfunction of the SHO are
  \begin{align}\label{eq:3}
    \psi_n(x)=\psi^*_n(x)= (n!2^n)^{-\frac{1}{2}}
    \left(\frac{m\omega}{\hbar\pi}\right)^{\frac{1}{4}}
    \exp\left(-\frac{m\omega}{2\hbar}x^2\right)
    H_n\left(\sqrt{\frac{m\omega}{\hbar}}x\right),\,\,\,
    E_n=\left( n+\frac{1}{2} \right)\hbar\omega.
  \end{align}
  The expectation value of the kinetic energy is
  \begin{align}
   \langle T_n\rangle = \frac{1}{2m}\int \psi^*_n p^2 
   \psi_n\,dx
   =\frac{\langle p^2\rangle}{2m}.
  \end{align}
  Since the expectation value of the kinetic energy is 
  an integer multiple of the square of momentum, we just 
  calculate the expectation value of the 
  square of momentum. Using the integration by part,
  \begin{align}\label{eq:3-1}
    \langle p^2\rangle = -\hbar^2\int \psi^*_n 
    \frac{\partial^2 \psi_n}{\partial x^2} \,dx
    =\hbar^2\int\frac{\partial \psi^*_n}{\partial x} 
    \frac{\partial \psi_n}{\partial x} \,dx
  \end{align}
  Changing the variable,
  \begin{align}\label{eq:3-2}
    \sqrt{\frac{m\omega}{\hbar}}x = \xi,\,\,\,
    \frac{\partial \psi_n}{\partial x}
    =\frac{\partial \psi_n}{\partial \xi}
    \frac{\partial \xi}{\partial x}
    =\sqrt{\frac{m\omega}{\hbar}}
    \frac{\partial \psi_n}{\partial \xi}
  \end{align}  
  Then,
  \begin{align}
    \frac{\partial \psi_n}{\partial \xi}
    =(n!2^n)^{-\frac{1}{2}}
    \left(\frac{m\omega}{\hbar\pi}\right)^{\frac{1}{4}}
    (-\xi H_n\left(\xi\right) 
    +H^\prime_n\left(\xi\right))
    e^{-\frac{\xi^2}{2}}.
  \end{align}
  The integration of Eq.~\eqref{eq:3-1} is
  \begin{align}
    \begin{split}
      \int\frac{\partial \psi^*_n}{\partial x} 
      \frac{\partial \psi_n}{\partial x} \,dx
      &= (n!2^n)^{-1}\sqrt{\frac{m\omega}{\hbar\pi}}
      \frac{m\omega}{\hbar}\int{(-\xi H_n\left(\xi\right) 
      +H^\prime_n\left(\xi\right))}^2 e^{-\xi^2}\,
      \sqrt{\frac{\hbar}{m\omega}}d\xi  \\
      &= (n!2^n)^{-1}\frac{m\omega}{\hbar\sqrt{\pi}}
      \int{(-\xi H_n\left(\xi\right) 
      +H^\prime_n\left(\xi\right))}^2 e^{-\xi^2}\,d\xi  \\
    \end{split}
  \end{align}
  From Eq~\eqref{eq:2-0-3}, $H'(x) = 2xH_{n}(x)-H_{n+1}(x)$. Hence,
  \begin{align}
      \int(-\xi H_n\left(\xi\right) 
      +H^\prime_n\left(\xi\right))^2 e^{-\xi^2}\,d\xi
      &=\int(-\xi H_n\left(\xi\right) 
      + 2\xi H_n(\xi)-H_{n+1}(\xi))^2 e^{-\xi^2}\,d\xi  \\
      &=\int(\xi H_n\left(\xi\right) 
      -H_{n+1}(\xi))^2 e^{-\xi^2}\,d\xi \\
      &=\int
      (\xi^2H_n H_n-2\xi H_n H_{n+1}+H_{n+1}H_{n+1})  
      e^{-\xi^2}\,d\xi.
  \end{align}
  We can use Eq.~\eqref{eq:2-0-2},~\eqref{eq:2-0-4} and 
  ~\eqref{eq:2} to calculate this integration.
  \begin{align}
    \begin{split}
      \int \xi^2 H_n H_n e^{-\xi^2}\,dx 
      &= 2^n n!\sqrt{\pi}\left( n+\frac{1}{2} \right) \\  
      \int \xi H_n H_{n+1} e^{-\xi^2}\,dx 
      &= 2^n(n+1)!\sqrt{\pi}\\  
      \int  H_{n+1} H_{n+1} e^{-\xi^2}\,dx 
      &= 2^{n+1}(n+1)!\sqrt{\pi}.
    \end{split}
  \end{align}
  Then,
  \begin{align}
      \int(-\xi H_n\left(\xi\right) 
      +H^\prime_n\left(\xi\right))^2 e^{-\xi^2}\,d\xi
      &= \sqrt{\pi}2^n n!
      \left(n+\frac{1}{2}-2(n+1)+2(n+1) \right) \\
      &= \sqrt{\pi}2^n n!\left( n+\frac{1}{2} \right).
  \end{align}
  Therefore the expectation value of the square of the momentum is
  \begin{align}\label{eq:3-ps}
    \langle p^2\rangle = \hbar^2\,(n!2^n)^{-1}\frac{m\omega}
    {\hbar\sqrt{\pi}}\,
    \sqrt{\pi}2^n n!\left( n+\frac{1}{2} \right)
    =\hbar m\omega\left( n+\frac{1}{2} \right).
  \end{align}
  We obtain the expectation value of the kinetic energy.
  \begin{align}\label{eq:3-3}
    \langle T_n\rangle=\frac{\langle p^2\rangle}{2m}
    =\frac{1}{2}\hbar \omega\left( n+\frac{1}{2} \right).
  \end{align}
  The expectation value of the potential energy is
  \begin{align}
    \langle V_n\rangle = \int \psi^*_n \frac{1}{2}m\omega^2x^2 
    \psi_n \,dx
    =\frac{1}{2}m\omega^2\int \psi^*_n x^2 \psi_n \,dx
    =\frac{1}{2}m\omega^2\langle x^2\rangle.
  \end{align}
  From Eq.~\eqref{eq:3-2}, the expectation value of the square 
  of $x$ is
  \begin{align}
      \langle x^2\rangle &= \int \psi^*_n x^2 \psi_n \,dx
      =\langle x^2\rangle ={\left(\frac{\hbar}{m\omega}\right)}
      ^{\frac{3}{2}} 
      \int \psi^*_n(\xi) \xi^2 \psi_n(\xi) \,d\xi \\
      &= (n!2^n)^{-1}\sqrt{\frac{m\omega}{\hbar\pi}}
      {\left(\frac{\hbar}{m\omega}\right)}^{\frac{3}{2}} 
      \int \xi^2 H_n(\xi) H_n(\xi) e^{-\xi^2}\,d\xi  
  \end{align}
  The integration part can be calculated by Eq.~\eqref{eq:2}.
  \begin{align}
    \int \xi^2 H_n(\xi) H_n(\xi) e^{-\xi^2}\,d\xi
    =\sqrt{\pi}2^n n!\left( n+\frac{1}{2} \right).
  \end{align}
  So, $\langle x^2\rangle$ is
  \begin{align}\label{eq:3-xs}
    \langle x^2\rangle = 
    \frac{\hbar}{m\omega}
    \left( n+\frac{1}{2}\right).
  \end{align}
  Finally we obtain the expectation value of the potential energy.
  \begin{align}\label{eq:3-4}
    \langle V_n\rangle=\frac{1}{2}m\omega^2(n!2^n)^{-1}
    \sqrt{\frac{m\omega}{\hbar\pi}}
    {\left(\frac{\hbar}{m\omega}\right)}^{\frac{3}{2}} 
    \sqrt{\pi}2^n n!\left( n+\frac{1}{2} \right)
    =\frac{1}{2}\hbar\omega
    \left( n+\frac{1}{2} \right).
  \end{align}
  Let us confirm that the results satisfy the virial theorem. 
  In this condition the virial theorem is
  \begin{align}
    \left\langle x\frac{\partial V}{\partial x}\right\rangle 
    = 2\left\langle T\right\rangle .
  \end{align}
  Substituting Eq.~\eqref{eq:3-3} and~\eqref{eq:3-4},
  \begin{align}
  \left\langle x\frac{\partial V}{\partial x}\right\rangle 
  =m\omega^2\int\psi^*_n x^2\psi_n\,dx = 2\left\langle 
  V_n\right\rangle 
  =2\left\langle T_n\right\rangle.
  \end{align}
  The results satisfy the virial theorem.
  \item[(2)] Let us calculate $\Delta x$ and $\Delta p$. 
  From the definition, $\Delta x$ and $\Delta p$ are
  \begin{align}
      \Delta x = \sqrt{\left\langle x^2\right\rangle 
      - {\langle x\rangle}^2  },\,\,\,
      \Delta p = \sqrt{\left\langle p^2\right\rangle 
      - {\langle p\rangle}^2  }.
  \end{align}
  $\langle x\rangle$ is
  \begin{align}
      \langle x\rangle = \int x\psi^*\psi\,dx 
      = (n!2^n)^{-1}\sqrt{\frac{m\omega}{\hbar\pi}}
      \left(\frac{\hbar}{m\omega}\right)
      \int \xi H_n H_n e^{-\xi^2}\,d\xi.
  \end{align}
  Since the integrated term is an even function and the integration 
  interval is symmetric,
  the integration is a zero. Therefore,
  \begin{align}
    \langle x\rangle = 0.
  \end{align}
  From Eq.~\eqref{eq:3-xs}, $\Delta x$ is
  \begin{align}
    \Delta x = \sqrt{\left\langle x^2\right\rangle}
    = \sqrt{\frac{\hbar}{m\omega}
    \left( n+\frac{1}{2}\right)}.
  \end{align}
  To calculate $\Delta p$, let us find the expectation value of $p$.
  $\langle p\rangle$ is
  \begin{align}
    \langle p\rangle = \int \psi^*p\psi\,dx
    = -i\hbar\int\psi^*_n
    \frac{\partial \psi_n}{\partial x} \,dx.
  \end{align}
  The integration by part of $\langle p\rangle$ is
  \begin{align}
      \langle p\rangle=-i\hbar\int\psi^*_n
      \frac{\partial \psi_n}{\partial x} \,dx
      =i\hbar\int\frac{\partial \psi^*_n}{\partial x}
      \psi_n \,dx.
  \end{align}
  From Eq.~\eqref{eq:3}, $\psi = \psi^*$. So,
  \begin{align}
    \langle p\rangle=i\hbar\int\frac{\partial \psi^*_n}{\partial x}
    \psi_n \,dx=i\hbar\int\psi^*_n
    \frac{\partial \psi_n}{\partial x}\,dx = -\langle p\rangle.
  \end{align}
  Therefore, we get
  \begin{align}
    \langle p\rangle = 0.
  \end{align}
  From Eq.~\eqref{eq:3-ps}, $\Delta p$ is
  \begin{align}
    \Delta p = \sqrt{\left\langle p^2\right\rangle}
    = \sqrt{\hbar m\omega\left( n+\frac{1}{2} \right)}.
  \end{align}
  Finally we obtain $\Delta x\Delta p$.
  \begin{align}
    \Delta x\Delta p = \sqrt{\frac{\hbar}{m\omega}
    \left(n+\frac{1}{2}\right)}\sqrt{\hbar m\omega
    \left(n+\frac{1}{2}\right)}
    =\hbar\left(n+\frac{1}{2}\right).
  \end{align}
\end{itemize}

\vspace{1cm}

\noindent \textbf{Problem 4.}
If a wavefunction desribes a mixed state of the eigenstates of the SHO
given as 
\begin{align}
  \psi(x,t) = \frac1{\sqrt{2}}[\psi_0(x,t) + \psi_1(x,t)] ,
\end{align}
\begin{itemize}
\item[(1)] Investigate how the probability density changes in time. 
\item[(2)] Prove the following relations 
  \begin{align}
\langle E\rangle &= \langle H\rangle =\hbar \omega    ,\cr
\langle x \rangle &= \frac1{\sqrt{2}\alpha}\cos\omega t,\cr
\langle p \rangle &= -\frac{\alpha}{\sqrt{2}}\hbar \sin\omega t,
 \end{align}
where $\alpha = \sqrt{m\omega/\hbar}$.
 \item[(3)] If 
   \begin{align}
 \psi(x,t) = \frac1{\sqrt{2}}[e^{i\delta_0} \psi_0(x,t) + e^{i\delta}
     \psi_1(x,t)],     
   \end{align}
discuss the effects of the phase factors $\delta_0$ and $\delta$ on
$\langle x\rangle$ and $\langle p\rangle$.
\end{itemize}

\noindent \textbf{Answer : }
\begin{itemize}
  \item[(1)] The probability density of this wavefunction is
  \begin{align}\label{eq:4-1}
    \rho = |\psi(x,t)|^2 =\frac{1}{2}\left[
      |\psi_0(x,t)|^2+|\psi_1(x,t)|^2+\psi_0^*(x,t)
      \psi_1(x,t)+\psi_0(x,t)\psi_1^*(x,t)
    \right].
  \end{align}
  To consider the time factor $\exp{\left(-\frac{iE_n}{\hbar}t\right)}$, 
  we have to know the energy of the SHO.
  The Schr\"odinger equation of the SHO is
  \begin{align}\label{eq:4-2}
      &H\psi_n(x,0)=E_n\psi_n(x,0)=\left(\frac{1}{2}+n\right)
      \hbar\omega\psi_n(x,0).
  \end{align}
  And,
  \begin{align}%\label{eq:4-3}
    \psi_0(x,t) = \psi_0(x,0)\exp{\left(-\frac{iE_0}{\hbar}t\right)},\,\,\,
    \psi_1(x,t) = \psi_1(x,0)\exp{\left(-\frac{iE_1}{\hbar}t\right)}
  \end{align}
  Therefore energys of $\psi_0$ and $\psi_1$ are
  \begin{align}
    \begin{split}\label{eq:4-4}
      E_0 = \frac{1}{2}\hbar\omega,\,\,\,\psi_0(x,t)
      &=\psi_0(x,0)e^{-\frac{1}{2}i\omega t},  \\
      E_1 = \frac{3}{2}\hbar\omega,\,\,\,\psi_1(x,t)
      &=\psi_1(x,0)e^{-\frac{3}{2}i\omega t}.
    \end{split}
  \end{align}
  Then Eq.~\eqref{eq:4-1} is
  \begin{align}%\label{eq:4-5}
    \rho =\frac{1}{2}\left[
      |\psi_0(x,0)|^2+|\psi_1(x,0)|^2+\psi_0^*(x,0)\psi_1(x,0)e^{-i\omega t}
      +\psi_0(x,0)\psi_1^*(x,0)e^{i\omega t}
    \right]
  \end{align}
  Since $\psi_0(x,0)$ and $\psi_0(x,0)$ are the eigenstate of the SHO,
  \begin{align}\label{eq:4-6}
    \psi_0(x,0)=\psi_0^*(x,0),\,\,\,\psi_1(x,0)=\psi_1^*(x,0)
  \end{align}
  Then the last two terms are
  \begin{align}%\label{eq:4-7}
    \psi_0^*(x,0)\psi_1(x,0)e^{-i\omega t}
      +\psi_0(x,0)\psi_1^*(x,0)e^{i\omega t}
      =2\psi_0(x,0)\psi_1(x,0)\cos{\omega t}.
  \end{align}
  The probability density is
  \begin{align}\label{eq:4-8}
    \rho =\frac{1}{2}\left[
      |\psi_0(x,0)|^2+|\psi_1(x,0)|^2
      +2\psi_0(x,0)\psi_1(x,0)\cos{\omega t}\right]
  \end{align}
  Because $-1\leq \cos{\omega t}\leq 1$, the probability density oscillates
  having the amplitude between $\rho_{\mathrm{min}}$ and $\rho_{\mathrm{max}}$.
  \begin{align}
    \rho_{\min} = \frac{1}{2}\left(\psi_0(x,0)-\psi_1(x,0)
      \right)^2,\,\,\,
      \rho_{\max} = \frac{1}{2}\left(\psi_0(x,0)+\psi_1(x,0)
      \right)^2.
  \end{align}
  \item[(2)]
   From Eq.~\eqref{eq:4-2}, Eq.~\eqref{eq:4-4} and Eq.~\eqref{eq:4-6}, 
   the expectation value of the Hamiltonian is
  \begin{align}
    \begin{split}
      \langle H \rangle &= \int \psi^*(x,t)H\psi(x,t)\,dx
      = \int \psi^*(x,t)E\psi(x,t)\,dx
      =\langle E \rangle  \\
      &=\frac{1}{2}\int
      [\psi_0^*(x,t) + \psi_1^*(x,t)][H\psi_0(x,t) + H\psi_1(x,t)]
      \,dx \\
      &=\frac{1}{2}\int
      \left[e^{\frac{1}{2}i\omega t}\psi_0(x,0) 
      + e^{\frac{3}{2}i\omega t}\psi_1(x,0)\right]
      \left[\frac{1}{2}\hbar\omega e^{-\frac{1}{2}i\omega t}\psi_0(x,0) 
      + \frac{3}{2}\hbar\omega 
      e^{-\frac{3}{2}i\omega t}\psi_1(x,0)\right]
      \,dx.
    \end{split}
  \end{align}
  Since $\psi_0(x,0)$ and $\psi_1(x,0)$ are orthogonal to each other,
  the term of $\psi_0(x,0)\psi_1(x,0)$ can be canceled out.
  \begin{align}\label{eq:4-9}
    \begin{split}
      \langle H \rangle = \frac{1}{2}\int
      \left[\frac{1}{2}\hbar\omega |\psi_0(x,0)|^2 
      + \frac{3}{2}\hbar\omega |\psi_1(x,0)|^2\right]
      \,dx
      =\frac{1}{2}\left[\frac{1}{2}\hbar\omega
      +\frac{3}{2}\hbar\omega\right]
      =\hbar\omega.
    \end{split}
  \end{align}
  The expectation value of the $x$ is
  \begin{align}
    \langle x \rangle = \int \psi^*(x,t)x\psi(x,t)\,dx
    =\int x|\psi(x,t)|^2\,dx=\int x\rho\,dx
  \end{align}
  From the Eq.~\eqref{eq:4-8},
  \begin{align}\label{eq:4-11}
    \langle x \rangle=\frac{1}{2}\int x\left[
      |\psi_0(x,0)|^2+|\psi_1(x,0)|^2
      +2\psi_0(x,0)\psi_1(x,0)\cos{\omega t}\right]\,dx.
  \end{align}
  Since the first two terms in the braket are the even functions, 
  these terms can be canceled out.
  \begin{align}\label{eq:4-12}
    \langle x \rangle=\int x\psi_0(x,0)\psi_1(x,0)\cos{\omega t}\,dx.
  \end{align}
  $\psi_0(x,0)$ and $\psi_1(x,0)$ are the eigenstate of the SHO. Therefore,
  \begin{align}
      \psi_0(x,0)&=\left(\frac{m\omega}{\pi\hbar}\right)
      ^{\frac{1}{4}}
      \exp{\left(-\frac{m\omega}{2\hbar}x^2\right)} \\
      \psi_1(x,0)&=\sqrt{2}\left(\frac{m\omega}{\pi\hbar}\right)
      ^{\frac{1}{4}}
      \sqrt{\frac{m\omega}{\hbar}}x\exp{\left(-\frac{m\omega}{2\hbar}
      x^2\right)}.
  \end{align}
  Then the expectation value of $x$ is
  \begin{align}
      \langle x \rangle
      &=\sqrt{\frac{2}{\pi}}
      \frac{m\omega}{\hbar}\cos{\omega t}
      \int x^2\exp{\left(-\frac{m\omega}{\hbar}x^2\right)}
      \,dx.
  \end{align}
  Substituting $\alpha = \sqrt{m\omega/\hbar}$, we get
  \begin{align}
      \langle x \rangle
      &=\sqrt{\frac{2}{\pi}}
      \alpha^2\cos{\omega t}
      \left(-\frac{1}{2\alpha} \right)
      \left(\frac{d}{d\alpha}\right)\int 
      e^{-\alpha^2x^2}\,dx 
      = -\sqrt{\frac{1}{2\pi}}
      \alpha\cos{\omega t}
      \left(\frac{d}{d\alpha}\right)\frac{\sqrt{\pi}}{\alpha} \\
      &= \sqrt{\frac{1}{2\pi}}
      \alpha\cos{\omega t}
      \left(\frac{\sqrt{\pi}}{\alpha^2}\right)
      =\frac{1}{\sqrt{2}\alpha}\cos{\omega t}.
  \end{align}
  Before finding expectation value of $p$, let us show that
  \begin{align}
    \langle p \rangle = m\frac{d}{dt}\langle x \rangle.
  \end{align}
  From the Generalized Ehrenfest's Theorem,
  \begin{align}
    i\hbar\frac{d}{dt}\langle x \rangle
    =\langle\left[x,H\right]\rangle
    +i\hbar\left\langle
    \frac{\partial x}{\partial t}\right\rangle
    =\left\langle
    \left[x,\frac{p^2}{2m}+\frac{1}{2}m\omega^2x^2\right]
    \right\rangle
    =\frac{1}{2m}\left\langle
    \left[x,p^2\right]
    \right\rangle.
  \end{align}
  Since $\left[x,p^2\right]=2i\hbar p$,
  \begin{align}
    i\hbar\frac{d}{dt}\langle x \rangle
    =\frac{i\hbar}{m}\langle
     p\rangle.
  \end{align}
  Hence,
  \begin{align}\label{eq:4-13}
    \langle p\rangle
    =m\frac{d}{dt}\langle x \rangle
    =-\frac{m\omega}{\sqrt{2}\alpha}\sin{\omega t}
    =-\frac{\alpha}{\sqrt{2}}\hbar\sin{\omega t}.
  \end{align}
  \item[(3)] 
  If there are the phase factors $\delta_0$ and $\delta$,
  Eq.~\eqref{eq:4-12} changes into
  \begin{align}
      \langle x \rangle&=\frac{1}{2}
      \left(e^{i(\delta_0-\delta)}+e^{-i(\delta_0-\delta)}\right)
      \int x\left[\psi_0(x,0)\psi_1(x,0)\cos{\omega t}\right]\,dx  \\
      &=\cos{(\delta_0-\delta)}
      \int x\left[\psi_0(x,0)\psi_1(x,0)\cos{\omega t}\right]\,dx.
  \end{align}
  Let define $\Delta\delta$ as $\Delta\delta=\delta_0-\delta$. Then 
  $\langle x \rangle$ is
  \begin{align}
    \langle x \rangle=\cos{\Delta\delta}
    \int x\left[\psi_0(x,0)\psi_1(x,0)\cos{\omega t}\right]\,dx.
  \end{align}
\end{itemize}
The integration part equals the expectation value of $x$ 
without the phase factors. Therefore,
\begin{align}
  \langle x \rangle=
  \frac{1}{\sqrt{2}\alpha}\cos{\Delta\delta}\cos{\omega t}.
\end{align}
From Eq.~\eqref{eq:4-13}, the expectation value of $p$ is
\begin{align}
  \langle p \rangle=
  -\frac{\alpha}{\sqrt{2}}\hbar\cos{\Delta\delta}\sin{\omega t}.
\end{align}
If $\Delta\delta=\left(n+\frac{1}{2}\right)\pi$, the expectation value of 
$x$ and $p$ are zeros. And when $\Delta\delta=n\pi$,  the expectation value of 
$x$ and $p$ has the maximum value.
 \vspace{1cm}

\noindent \textbf{Problem 5.}
Derive the wavefunction in momentum space, which corresponds to the
eigenfunctions for the SHO in coordinates, $\psi_n(x)$. 

\noindent \textbf{Answer : }
The solution of the Schr\"odinger equation in the 
coordinate space is

 \begin{align}\label{eq:5-1}
   \psi_n(x)=(n!2^n)^{-\frac{1}{2}}
   \left(\frac{m\omega}{\hbar\pi}\right)^{\frac{1}{4}}
   \exp\left(-\frac{m\omega}{2\hbar}x^2\right)
   H_n\left(\sqrt{\frac{m\omega}{\hbar}}x\right).
 \end{align}
 It satisfies the equation,
\begin{align}\label{eq:5-2}
  -\frac{\hbar^2}{2m}\frac{\partial^2\psi_n}{\partial x^2}
  +\frac{1}{2}m\omega^2x^2\psi_n = E_n\psi_n,\,\,\,
  E_n = \left(n+\frac{1}{2}\right)\hbar\omega.
\end{align}
The wavefunction in the momentum space $\phi_n(p)$ 
is the Inverse Fourier Transformation 
of the wavefunction in the coordinate space $\psi_n(x)$.
\begin{align}
  \begin{split}
    \phi_n(p) &= \frac{1}{\sqrt{2\pi\hbar}}
    \int \psi_n(x) e^{-\frac{i}{\hbar} px} \, dx \\
    &= \frac{1}{\sqrt{2\pi\hbar}}(n!2^n)^{-\frac{1}{2}}
    \left(\frac{m\omega}{\hbar\pi}\right)^{\frac{1}{4}}
    \int H_n\left(\sqrt{\frac{m\omega}{\hbar}}x\right)
    \exp{\left( -\frac{m\omega}{2\hbar}x^2
    -\frac{i}{\hbar} px \right)}\, dx.
  \end{split}
\end{align}
Changing $x$ and $p$ into the dimensionless variables 
$\xi$ and $p_{\xi}$.
\begin{align}
  \xi = \sqrt{\frac{m\omega }{\hbar }}x,\,\,\,
  p_\xi = \frac{1}{\sqrt{\hbar m \omega}}p.
\end{align}
Then $\phi_n(p)$ is
\begin{align}
  \phi_n(p) = \frac{1}{\sqrt{2\pi\hbar}}(n!2^n)^{-\frac{1}{2}}
  \left(\frac{m\omega}{\hbar\pi}\right)^{\frac{1}{4}}
  \sqrt{\frac{\hbar }{m\omega }}
  \int H_n\left( \xi \right)
  e^{-\frac{1}{2}\xi^2 -i p_\xi\xi }\, d\xi.
\end{align}
To find the Inverse Fourier Transformation of the Hermite polynomial,
let us consider the generating function of the Hermite polynomial.
\begin{align}\label{eq:5-3}
  \int e^{-t^2 +2\xi t-\frac{1}{2}\xi^2 -i p_\xi\xi } \,d\xi
  =\sum_n \frac{t^n}{n!}\int H_n(\xi)
  e^{-\frac{1}{2}\xi^2 -i p_\xi\xi }\, d\xi.
\end{align}
The LHS is the gaussian integration of $\xi$.
\begin{align}
  \begin{split}
    \int e^{-t^2 +2\xi t-\frac{1}{2}\xi^2 -i p_\xi\xi } \,d\xi
    &=e^{-t^2 }\int e^{-\frac{1}{2}\xi^2+\xi(2t -i p_\xi) } \,d\xi
    =e^{-t^2 }\int e^{-\frac{1}{2}(\xi-(2t -i p_\xi))^2
    +\frac{1}{2}(2t-ip_\xi)^2 } \,d\xi  \\
    &=e^{-t^2 +\frac{1}{2}(2t-ip_\xi)^2}\sqrt{2\pi}
    =\sqrt{2\pi}e^{t^2-2ip_\xi-\frac{1}{2}p^2_\xi}.
  \end{split}
\end{align}
We can obtain the generating function about $-it$ and $p_\xi$.
\begin{align}
  \int e^{-t^2 +2\xi t-\frac{1}{2}\xi^2 -i p_\xi\xi } \,d\xi
  =\sqrt{2\pi}e^{t^2-2ip_\xi-\frac{1}{2}p^2_\xi}
  =\sqrt{2\pi}e^{-\frac{1}{2}p^2_\xi}
  \sum_m\frac{\left(-it\right)^m}{m!}H_m(p_\xi).
\end{align}
From Eq.~\eqref{eq:5-3}, the coefficients of $n$th order 
have to be the same.
\begin{align}
  \begin{split}
    &\sum_n \frac{t^n}{n!}\int H_n(\xi)
    e^{-\frac{1}{2}\xi^2 -i p_\xi\xi }\, d\xi
    =\sum_m\frac{\left(-it\right)^m}{m!}H_m(p_\xi)
    \sqrt{2\pi}e^{-\frac{1}{2}p^2_\xi},  \\
    &\int H_n(\xi)
    e^{-\frac{1}{2}\xi^2 -i p_\xi\xi }\, d\xi
    =(-i)^mH_m(p_\xi)
    \sqrt{2\pi}e^{-\frac{1}{2}p^2_\xi}.
  \end{split}
\end{align}
Therefore the wavefunction in the momentum space $\phi_n(p)$ is
\begin{align}
  \phi_n(p) = (n!2^n)^{-\frac{1}{2}}
  \left(\frac{m\omega}{\hbar\pi}\right)^{\frac{1}{4}}
  \sqrt{\frac{1}{m\omega }}(-i)^nH_n(p_\xi)
  e^{-\frac{1}{2}p^2_\xi}.
\end{align}
\vspace{1cm}

\noindent \textbf{Problem 6.}
At $t=0$, the wavefunction for a state is described by
\begin{align}\label{eq:6-1}
\psi(x,0) = \sum_n A_n u_n(x) =
  \left(\frac{\alpha^2}{\pi}\right)^{1/4} e^{-\alpha^2(x-a)^2/2}  .
\end{align}
show that after some time $t$, the probability density changes in time
as 
\begin{align}\label{eq:6-2}
|\psi(x,t)|^2  =  \left(\frac{\alpha^2}{\pi}\right)^{1/4}e^{-\alpha^2
  (x-a\cos\omega t)^2}
\end{align}
and discuss the result. 


\noindent \textbf{Answer : }
First we change the form of the exponential more similarly with 
the generating function for the Hermite polynomial Eq.~\eqref{eq:1}.
\begin{align}
  \begin{split}\label{eq:6-3}
    \exp{\left(-\frac{\alpha^2}{2}(x-a)^2\right)}
    &= \exp{\left(-\frac{1}{2}\alpha^2x^2+\alpha^2 ax
    -\frac{1}{2}\alpha^2a^2\right)}
    =\exp{\left(-\frac{1}{2}\alpha^2x^2\right)
    \exp\left(\alpha^2 ax-\frac{1}{2}\alpha^2a^2\right)}  \\
    &= \exp{\left(-\frac{1}{2}\alpha^2x^2\right)
    \exp\left(2\left(\frac{\alpha a}{2}\right)(\alpha x)
    -\left(\frac{\alpha a}{2}\right)^2\right)}
    \exp \left(-\left(\frac{\alpha a}{2}\right)^2\right).
  \end{split}
\end{align}
The second exponential of the RHS is the generating function
$S\left(\alpha x, \frac{\alpha a}{2}\right)$.
\begin{align}%\label{eq:6-4}
  \exp{\left(-\frac{\alpha^2}{2}(x-a)^2\right)}
  =\exp{\left(-\frac{1}{2}\alpha^2\left(x^2+\frac{1}{2}a^2\right)
  \right)}
  \sum_{m=0}^{\infty}\frac{H_m(\alpha x)}{m!}\left(\frac{\alpha a}{2}
  \right)^n.
\end{align}
Hence, $\psi(x,0)$ is
\begin{align}\label{eq:6-5}
    \psi(x,0) &= \sum_n A_n u_n(x) =
    \left(\frac{\alpha^2}{\pi}\right)^{\frac{1}{4}}
    \exp{\left(-\frac{1}{2}\alpha^2\left(x^2+\frac{1}{2}a^2\right)
    \right)}
    \sum_n\frac{H_n(\alpha x)}{n!}\left(\frac{\alpha a}{2}
    \right)^n \\
    \label{eq:6-5-1}  &=\sum_n
    \left(\left(\frac{\alpha^2}{\pi}\right)^{\frac{1}{4}}
    \exp{\left(-\frac{1}{4}\alpha^2a^2\right)}
    \frac{1}{n!}\left(\frac{\alpha a}{2}\right)^n\right)
    \left(\exp{\left(-\frac{1}{2}\alpha^2x^2\right)}
    H_n(\alpha x)\right).
\end{align}
From Eq.~\eqref{eq:6-1}, the wavefunction at the time $t$ is
\begin{align}
    \psi(x,t) &= \sum_n A_nu_n(x)e^{-i\frac{E_n}{\hbar}t}
    =\sum_n A_nu_n(x)e^{-i\left(n+\frac{1}{2}\right)\omega t}  \\
    &=e^{-\frac{1}{2}i\omega t}
    \sum_n A_nu_n(x)e^{-in\omega t}.
\end{align}
Substituting Eq.~\eqref{eq:6-5}, we get
\begin{align}
    \psi(x,t) &=
    \left(\frac{\alpha^2}{\pi}\right)^{\frac{1}{4}}
    e^{-\frac{1}{2}\alpha^2\left(x^2+\frac{1}{2}a^2\right)}
    e^{-\frac{1}{2}i\omega t}
    %\exp{\left(-\frac{1}{2}\alpha^2\left(x^2+\frac{1}{2}a^2\right)\right)}
    \sum_n\left(\frac{H_n(\alpha x)}{n!}
    \left(\frac{\alpha a}{2}\right)^n
    e^{-in\omega t} \right) \\
    &=\left(\frac{\alpha^2}{\pi}\right)^{\frac{1}{4}}
    e^{-\frac{1}{2}\alpha^2\left(x^2+\frac{1}{2}a^2\right)}
    e^{-\frac{1}{2}i\omega t}
    \sum_n\left(\frac{H_n(\alpha x)}{n!}
    \left(\frac{\alpha a}{2}e^{-i\omega t} \right)^n
    \right).
\end{align}
The summation can be expressed as the generating function for the Hermite polynomial.
\begin{align}
  &\sum_n\left(\frac{H_n(\alpha x)}{n!}
  \left( \frac{\alpha a}{2}e^{-i\omega t} \right)^n
  \right) = S\left(\alpha x,\frac{\alpha a}{2}
  e^{-i\omega t}\right)
  =\exp{\left(-\frac{1}{4}\alpha^2 a^2e^{-2i\omega t}+\alpha^2ax
  e^{-i\omega t}\right)},  \\
  \label{eq:6-6}&\psi(x,t)
  =\left(\frac{\alpha^2}{\pi}\right)
  ^{\frac{1}{4}}
  e^{-\frac{1}{2}\alpha^2
  \left(x^2+\frac{1}{2}a^2\right)}
  e^{-\frac{1}{2}i\omega t}
  \exp{\left(-\frac{1}{4}\alpha^2 
  a^2e^{-2i\omega t}
  +\alpha^2axe^{-i\omega t}\right)}
\end{align}
From Eq.~\eqref{eq:6-6}, we obtain the probability density at the time $t$.
\begin{align}
    |\psi(x,t)|^2
    &=\left(\frac{\alpha^2}{\pi}\right)^{\frac{1}{2}}
    e^{-\alpha^2\left(x^2+\frac{1}{2}a^2\right)}
    \exp{\left(-\frac{1}{4}\alpha^2 a^2
    (e^{2i\omega t}+e^{-2i\omega t})
    +\alpha^2ax(e^{i\omega t}+e^{-i\omega t})\right)} \\
    &=\left(\frac{\alpha^2}{\pi}\right)^{\frac{1}{2}}
    \exp{\left(-\alpha^2x^2
    -\frac{1}{2}\alpha^2 a^2
    (1+\cos{2\omega t})
    +2\alpha^2ax\cos{\omega t}\right)}.
    %&= \left(\frac{\alpha^2}{\pi}\right)^{\frac{1}{2}}
    %e^{-\alpha^2\left(x^2+\frac{1}{2}a^2\right)}
    %\exp{\left(-\alpha^2a
    %\left(\frac{1}{2} a(2\cos^2{\omega t}-1)
    %-2x\cos{\omega t}
    %\right)\right)}
\end{align}
From the half angle identity,
$1+\cos{2\omega t}=2\cos^2{\omega t}$. So,
\begin{align}
  \begin{split}
    |\psi(x,t)|^2&=\left(\frac{\alpha^2}{\pi}\right)^{\frac{1}{2}}
    \exp{\left(-\alpha^2x^2
    -\alpha^2 a^2\cos^2{\omega t}
    +2\alpha^2ax\cos{\omega t}\right)}  \\
    &=\left(\frac{\alpha^2}{\pi}\right)^{\frac{1}{2}}
    \exp{\left(-\alpha^2(x^2
    +a^2\cos^2{\omega t}
    -2ax\cos{\omega t})\right)} \\
    &=\left(\frac{\alpha^2}{\pi}\right)^{\frac{1}{2}}
    \exp{\left(-\alpha^2(x
    -a\cos{\omega t})\right)^2}.
  \end{split}
\end{align}
Finally the probability density is
\begin{align}
  |\psi(x,t)|^2
  =\left(\frac{\alpha^2}{\pi}\right)^{\frac{1}{2}}
  e^{-\alpha^2(x-a\cos{\omega t})^2} .
\end{align}
With this probability density, the wavefunction moves between 
intervals of $2a$ periodically without change in the shape and energy.
Since, the expectation value of the position and the momentum oscillate 
with the probability density.
\vspace{1cm}

\noindent \textbf{Problem 7.} THe Einstein model for a solid
assumes that it consists of many SHOs. If the $N$ atoms are similar
each other and oscillate similarly in average, the solid can be
explained in terms of $N$ SHOs. At a given temperature $T$, $N$ atoms
are in thermal equilibrium. Then, the Boltzmann distribution is given
by 
\begin{align}\label{eq:7-1}
P_n = \frac1{Z} e^{-E_n/kT}   
\end{align}
with
\begin{align}\label{eq:7-2}
  Z = \sum_n e^{-E_n/kT}, 
\end{align}
where
\begin{align}\label{eq:7-3}
E_n = \left(n+\frac12\right)\hbar \omega .
\end{align}
\begin{itemize}
\item[(1)] Derive the mean energy per an SHO 
  \begin{align}
    \langle E\rangle = \frac{\hbar\omega}{e^{\hbar\omega/kT}-1} +
    \frac12 \hbar \omega.
  \end{align}
\item[(2)] If $U$ is the internal energy of the solid, derive the
  specific heat with constant volume 
  \begin{align}
    C_V = \frac{\partial U}{\partial T}.
  \end{align}
Show that when $T$ is large, $C_V=3R$.
\item[(3)] Discuss the physics related to this problem as far as you
  can. 
\end{itemize}

\noindent \textbf{Answer : }
\begin{itemize}
  \item[(1)] By the definition, the expectation value of the energy is
  \begin{align}
    \langle E\rangle = \sum_n E_nP_n 
    = \frac{1}{Z}\sum_n E_n e^{-E_n/kT},   \,\,\,
    Z = \sum_n P_n.
  \end{align}
  Define $\beta$ as
  \begin{align}\label{eq:7-4}
    \beta =\frac{1}{kT}.
  \end{align}
  Then,
  \begin{align}
    \langle E\rangle = \frac{1}{Z}\sum_n E_n e^{-\beta E_n},   \,\,\,
    Z = \sum_n e^{-\beta E_n}.
  \end{align}
  The summation term is regraded as the deriavtive for $\beta$.
  \begin{align}\label{eq:7-5}
    \langle E\rangle = -\frac{1}{Z}\frac{\partial Z}{\partial \beta}
    = -\frac{\partial (\ln{Z})}{\partial \beta}.
  \end{align}
  From Eq.~\eqref{eq:7-3}, $Z$ is
  \begin{align}
    Z = \sum_n e^{-\beta\hbar\omega(\frac{1}{2}+n)} 
    = e^{-\frac{1}{2}\beta\hbar\omega}\sum_n 
    e^{-n\beta\hbar\omega}.
  \end{align}
  It is power series with a common ratio 
  $e^{-\beta\hbar\omega}$ 
  and first term $e^{-\frac{1}{2}\beta\hbar\omega}$.
  \begin{align}
    Z = \frac{e^{-\frac{1}{2}\beta\hbar\omega}}
    {1-e^{-\beta\hbar\omega}}.
  \end{align}
  Then $\ln{Z}$ is
  \begin{align}
    \ln{Z} = \ln{\left(e^{-\frac{1}{2}\beta\hbar\omega}
    \right)}
    -\ln{\left(1-e^{-\beta\hbar\omega}\right)}
    =-\frac{1}{2}\beta\hbar\omega
    -\ln{\left(1-e^{-\beta\hbar\omega}\right)}.
  \end{align}
  And Eq.~\eqref{eq:7-5} is
  \begin{align}
      \langle E\rangle = -\frac{\partial (\ln{Z})}
      {\partial \beta}
      = \frac{1}{2}\hbar\omega
      +\frac{\hbar\omega e^{-\beta\hbar\omega}}
      {1-e^{-\beta\hbar\omega}}  
  \end{align}
  Multiplying $e^{\beta\hbar\omega}$ to the second term of the RHS,
  \begin{align}
    \begin{split}\label{eq:7-6}
      \langle E\rangle &=\frac{1}{2}\hbar\omega
      +\frac{\hbar\omega}{e^{\beta\hbar\omega}-1} \\
      &=\frac{\hbar\omega}{e^{\hbar\omega/kT}-1} 
      + \frac{1}{2}\hbar\omega.
    \end{split}
  \end{align}
  \item[(2)] The first law of the thermodynamics is
  \begin{align}
    dU = dQ - dW.
  \end{align} 
  $Q$ and $W$ are the heat supplied to the system and 
  the work done on the system respectively. In the case of
  the solid, the volume is constant and the work is a zero.
  \begin{align}
    dW = PdV = 0,\,\,\, dU = dQ.
  \end{align}
  By the definition of the specific heat with constant volume,
  \begin{align}
    C_V = \left(\frac{\partial Q}{\partial T}\right)_V 
    =\frac{\partial U}{\partial T}.
  \end{align}
  Since $U$ is the total energy of the solid and there are the $N$
  atoms,
  \begin{align}
    U = N\langle E\rangle=\frac{N\hbar\omega}{e^{\hbar\omega/kT}-1} 
    + \frac{1}{2}N\hbar\omega.
  \end{align}
  Hence the specific heat $C_V$ is
  \begin{align}\label{eq:7-8}
    C_V = \frac{\partial U}{\partial T}
    = \frac{\partial }{\partial T}
    \left(\frac{N\hbar\omega}{e^{\hbar\omega/kT}-1}\right) 
    =\frac{N\hbar^2\omega^2e^{\hbar\omega/kT}}
    {kT^2\left(e^{\hbar\omega/kT}-1\right)^2}.
  \end{align}
  Considering that the degree of freedom is 3, we get
  \begin{align}\label{eq:7-9}
    C_V = \frac{3N\hbar^2\omega^2e^{\hbar\omega/kT}}
    {kT^2\left(e^{\hbar\omega/kT}-1\right)^2}.
  \end{align}
  From Eq.~\eqref{eq:7-4}, Eq.~\eqref{eq:7-8} can be rewritten as
  \begin{align}
    C_V=\frac{3Nk\beta^2\hbar^2\omega^2e^{\beta\hbar\omega}}
    {\left(e^{\beta\hbar\omega}-1\right)^2}.
  \end{align}
  When $T$ is large, $\beta$ is converged to a zero and 
  $e^{-\beta\hbar\omega}$ is converged to 1. Then,
  \begin{align}
    \lim_{\beta\rightarrow 0}\frac{\beta}
    {e^{\beta\hbar\omega}-1} = \frac{1}{\hbar\omega}.
  \end{align}
  Finally $C_V$ is
  \begin{align}
    C_V = \frac{3Nk\hbar^2\omega^2}
    {\hbar^2\omega^2}=3Nk = 3nR.
  \end{align}
  \item[(3)] Let us consider when the temperature is very small.
  From Eq.~\eqref{eq:7-9},
  \begin{align}
    \lim_{T\rightarrow 0}\frac{3N\hbar^2\omega^2e^{\hbar\omega/kT}}
    {kT^2\left(e^{\hbar\omega/kT}-1\right)^2} = 0.
  \end{align} 
  $C_v$ is converged to a zero.
  From Eq.~\eqref{eq:7-6}, when the temperature is large,
  the mean energy per an SHO behaves approximately as the linear function.
   \begin{align}
    \begin{split}
      \langle E\rangle &=\frac{1}{2}\hbar\omega
      +\frac{\hbar\omega}{e^{\beta\hbar\omega}-1} 
      \approx \frac{1}{2}\hbar\omega
      +\frac{1}{\beta}-\frac{1}{2}\hbar\omega
      +\frac{1}{12}\hbar\omega^2\beta+\cdots  \\
      &\approx \frac{1}{\beta} = kT.
    \end{split}
   \end{align} 
   In high temperatures, the mean energy per an SHO by The Einstein model is
   directly proportional to temperature.
\end{itemize}
\end{document}