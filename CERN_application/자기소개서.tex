\documentclass[aps,superscriptaddress,11pt]{revtex4-2}
\usepackage{kotex}
\usepackage[HWP]{dhucs-interword}
\usepackage[dvips]{color}
\usepackage{graphicx}
\usepackage{bm}
\usepackage{amsmath}
\usepackage{tikz}
\usepackage{mhchem}
\usepackage{booktabs}
\usepackage{array}
\usepackage{tikz}

\setlength{\arrayrulewidth}{0.4mm}
\setlength{\tabcolsep}{12pt}
\renewcommand{\arraystretch}{1.8}
\begin{document}
\title{자기소개서}
% \author{이휘재}
% \affiliation{Physics Department, Inha University}
% \email{hjlee6674@inha.edu}
% \date{\today}
 \maketitle
 \begin{table}[ht]
    \fontsize{8.5}{1}
    % \scriptsize
    \begin{tabular}{|p{1.3cm}|p{5.5cm}|p{1.3cm}|p{5.5cm}|}
    \hline
    \multicolumn{4}{|p{3.5cm}|}{\textbf{\large 인적사항}} \\
    \hline
    \hline
    성명    & 이휘재 & 생년월일     & 1999.07.30 \\
    \hline
    소속    & 인하대학교 물리학과 강입자이론 연구실& 입학년도 & 2018.03 \\
    \hline
    학번    & 12181978 & 현 학년   & 4학년 1학기(예정) \\
    \hline
    e-mail    & hjlee6674@inha.edu & 전화번호 & 010-6674-4887 \\
    \hline
    \end{tabular}
    \end{table}
% \begin{table}[ht]

%     \begin{tabular}{l  c | l  c }

%     \multicolumn{4}{p{3.5cm}}{\textbf{\large 인적사항}} \\
%     \hline
%     \hline
%     성명    :& 이휘재 & 생년월일     :& 1999.07.30 \\
%     \hline
%     소속    :& 인하대학교 물리학과 & 입학년도 :& 2018.03 \\
%     \hline
%     학번    :& 12181978 & 현 학년   :& 4학년 1학기(예정) \\
%     \hline
%     \end{tabular}
    
%     \end{table}
\section{소개 및 관심 분야}\small\noindent
Arthur C. Clarke의 유명한 문구
"Any sufficiently advanced technology is indistinguishable from magic."처럼,
2013년 LHC에서 힉스 입자가 존재한다는 증거가 발견되었다는 소식은 중학교 2학년인
저에게 마법처럼 들리기에 충분했고 이를 계기로 물리학에 이끌려 2018년에 인하대학교 물리학과에 
진학하게 되었습니다. 물리학과 학부생의 신분으로 대학교에서 전공 강의와 여러 세미나를 통해
물리학의 세부 분야 중, 핵자의 내부 구조를 연구하는 강입자 이론 물리 분야를 알게되어
현재의 강입자이론 연구실에 들어오게 되었습니다. 현재는 연구실 내에서 주로 다루는
Chiral-quark soliton model을 이용해 저에너지 영역에서 발생하는 현상을 연구함에 목적을
두고 이를 위해 공부 중에 있습니다.

\section{지원 동기 및 목적}\noindent
자연과학은 실험과 이론의 상호작용과 함께 발전해왔습니다. 
디랙이 그의 방정식으로 반입자를 예건하고 앤더슨이 안개상자를 통해 실험적으로 양전자의 존재를
검증한 것 처럼 실험이 이론의 근거가 되기도 하고, 러더퍼드가 알파 입자 산란 실험의 결과를 보고
새로운 원자 모형을 제시함으로써 보다 발전된 이론을 내놓았듯이 이론으로 실험의 결과를 설명하기도
하였습니다. 현대의 핵, 입자 물리학 분야는 실험이 이론을 앞서있는 상태입니다.
1946년 로런스가 개발한 사이클로트론은 알파 입자를 380MeV까지 가속할 수 있었지만 2008년 가동을
시작한 LHC는 양성자-양성자 충돌의 경우 빔을 7TeV까지 가속하여 충돌 에너지를 14TeV까지 올릴 수 있습니다.
단순하게 본다면 62년만에 빔의 에너지를 18000배 증가시킨 셈입니다. {\color{red}실험 기기의 눈부신 발전에 힘입어
물리학자들은 힉스 입자의 존재를 간접적으로 보여주는 증거를 발견하였고 표준 모형을 한번 더 검증하였습니다.}

%%%%%%%%%%%%%%%%%%%%%%%%%%%%%%%%%%%%%%%%%%%%%%%
1. 이론과 실험의 연관성 : 이론의 완성은 실험과 얼마나 일치하는가
\\
2. 동기부여 : Higgs particle, Pentaquark, 중력파의 발견
\vfill
\end{document}