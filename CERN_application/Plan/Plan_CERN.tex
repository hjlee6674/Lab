\documentclass[aps,superscriptaddress,11pt]{revtex4-2}
\usepackage{kotex}
\usepackage[HWP]{dhucs-interword}
\usepackage[dvips]{color}
\usepackage{graphicx}
\usepackage{bm}
\usepackage{amsmath}
\usepackage{xcolor}

\begin{document}
\title{Plan for CERN Summer Student Program}

\author{Hui-Jae Lee}

\affiliation{Physics Department, Inha University}
\email{hjlee6674@inha.edu}

\date{\today}

 \maketitle
 
\section{Introduction}
Conseil Européen pour la Recherche Nucléaire(CERN) is the world's largest particle 
physics research organization, employing cutting-edge science and technology. 
CERN has contributed to remarkable advances such as the discovery of the W, Z bosons, 
and CP violation in particle physics and has produced macroscopic results from a 
collection of human technologies such as the LHC, ALICE, ATLAS, and CMS. For young 
students who watched these achievements as children, it has become one of the most 
exciting dreams to research particle physics at CERN, either theoretically or 
experimentally. Because I'm one of them, I hope that I can learn a lot and get 
strong motivation from this summer program.


While writing this plan, I looked for activities that 
I expected to be able to participate in during the summer program.
The most anticipated activity in this summer program is the 
lecture program. 
Today's particle and nuclear physics are based on quantum field theory, 
which takes into account both relativistic effects and quantum mechanics.
Since most of the effective collision experiments in these fields are proceeding 
or scheduled in the high energy region, such as in the LHC or EIC, undergraduate 
students need to understand quantum field theory. Not only quantum field theory, 
but it is also crucial to study the Standard Model.

이유: QFT를 공부하고 있는 나에게 직접적인 도움이 될 것
CERN은 QFT와 SM의 전문가 집단
VISITS: 물리학을 연구함에 있어서 실험과 이론은 뗄 수 없는 관계에 있다. 
향후에 둘 중 어느 분야로 나아가던, 세계 최대의 실험 장비를 operating하는 
전문가들로부터 듣는 설명은 공부와 연구에 최고의 경험이 될 것이다.


%\begin{thebibliography}{9}
%\end{thebibliography}

\vfill
\end{document}