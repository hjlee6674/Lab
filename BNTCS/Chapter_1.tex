%\documentclass[preprint,tightenlines,showpacs,showkeys,floatfix,
%nofootinbib,superscriptaddress,fleqn]{revtex4} 
\documentclass[tightenlines,floatfix,nofootinbib,superscriptaddress,fleqn]{revtex4-2} 
%\documentclass[aps,epsfig,tightlines,fleqn]{revtex4}
\usepackage{kotex}
\usepackage[HWP]{dhucs-interword}
\usepackage[dvips]{color}
\usepackage{graphicx}
\usepackage{bm}
%\usepackage{fancyhdr}
%\usepackage{dcolumn}
\usepackage{slashed}
\usepackage{amsmath}
\usepackage{amsfonts}
\usepackage{amssymb}
\usepackage{amscd}
\usepackage{amsthm}
\usepackage[utf8]{inputenc}
\usepackage{xcolor}
%\pagestyle{fancy}

\begin{document}

\title{\Large SU(2) NJL model: vacuum sector and meson properties}
\author{Hui-Jae Lee} 
\email{hjlee6674@inha.edu}
\affiliation{Hadron Theory Group, Department of Physics,
  Inha  University, Incheon 22212, Republic of Korea }
  \author{Ho-Yeon Won}
  \affiliation{Hadron Theory Group, Department of Physics,
  Inha  University, Incheon 22212, Republic of Korea }
  \author{Yu-Son Jun}
  \affiliation{Hadron Theory Group, Department of Physics,
  Inha  University, Incheon 22212, Republic of Korea }
\maketitle

\section[2.4]{Mesonic Properties}

The action $S$ in this process is
\begin{align}\label{1-1}
  S &= -N_c\mathrm{Tr}\left[\log D\right]+\int d^4x\,\frac{1}{2G}\left(\sigma^2+\vec{\pi}^2\right)  \\
  &= -\frac{1}{2}N_c\mathrm{Tr}\left[\log D^\dagger D\right]
  +\int d^4x\,\frac{1}{2G}\left(\sigma^2+\vec{\pi}^2\right)
\end{align}
where $D^\dagger D$ is
\begin{align}\label{1-2}
  &D^\dagger D = -\partial^2+\sigma^2+\vec{\pi}^2
  -\left\{\slashed{\partial}(\sigma+i\vec{\pi}\cdot\vec{\tau}\gamma_5)\right\} 
  =V^{-1}+A
\end{align}
and $V^{-1}= -\partial^2+M^2,\,\,\, A=\sigma^2+\vec{\pi}^2-M^2
-\left\{\slashed{\partial}(\sigma+i\vec{\pi}\cdot\vec{\tau}\gamma_5)\right\}$.
By the expansion for log,
\begin{align}\label{1-3}
  \begin{split}
    &\mathrm{Tr}\left[\log\left(V^{-1}+A\right)\right] \approx 
    \mathrm{Tr}\left[\log\left(V^{-1}\right)\right]
    -\frac{1}{2}\mathrm{Tr}\left[VAVA\right]
    +8I_1(M)\int d^4x\left(\sigma^2+\vec{\pi}^2-M^2\right) \\
    &\Longrightarrow
    S=-\frac{1}{2}N_c\left[\mathrm{Tr}
    \left[\log\left(V^{-1}\right)\right]
    -\frac{1}{2}\mathrm{Tr}\left[VAVA\right]
    +8I_1(M)\int d^4x\left(\sigma^2+\vec{\pi}^2-M^2\right)\right]
    +\int d^4x\,\frac{1}{2G}\left(\left(\sigma-\hat{m}\right)^2+\vec{\pi}^2\right)
  \end{split}
\end{align}
We have to calculate $\mathrm{Tr}\left[VAVA\right]$ hand by hand.

\subsection{Calculation of $\mathrm{Tr}\left[VAVA\right]$}
\begin{align}\label{1-1-1}
  \begin{split}
    \mathrm{Tr}\left[VAVA\right]
    &=\mathrm{Tr}\left[V\left(\sigma^2+\vec{\pi}^2-M^2
  -\left\{\slashed{\partial}(\sigma+i\vec{\pi}\cdot\vec{\tau}\gamma_5)
  \right\}\right)V\left(\sigma^2+\vec{\pi}^2-M^2
  -\left\{\slashed{\partial}(\sigma+i\vec{\pi}\cdot\vec{\tau}\gamma_5)
  \right\}\right)\right]  \\
  &=\mathrm{Tr}\left[V\left(2\sigma_0\tilde{\sigma}+2\vec{\pi}_0\tilde{\vec{\pi}}
  +\tilde{\sigma}^2+\tilde{\vec{\pi}}^2
  -\left\{\slashed{\partial}(\tilde{\sigma}+i\tilde{\vec{\pi}}\cdot\vec{\tau}\gamma_5)
  \right\}\right)V\left(2\sigma_0\tilde{\sigma}+2\vec{\pi}_0\tilde{\vec{\pi}}
  +\tilde{\sigma}^2+\tilde{\vec{\pi}}^2
  -\left\{\slashed{\partial}(\tilde{\sigma}+i\tilde{\vec{\pi}}\cdot\vec{\tau}\gamma_5)
  \right\}\right)\right]  \\
  &=\mathrm{Tr}\left[V\left(2\sigma_0\tilde{\sigma}+2\vec{\pi}_0\tilde{\vec{\pi}}
  +\tilde{\sigma}^2+\tilde{\vec{\pi}}^2
  -\left\{\slashed{\partial}(\tilde{\sigma}+i\tilde{\vec{\pi}}\cdot\vec{\tau}\gamma_5)
  \right\}\right)V\left(2\sigma_0\tilde{\sigma}+2\vec{\pi}_0\tilde{\vec{\pi}}
  +\tilde{\sigma}^2+\tilde{\vec{\pi}}^2
  -\left\{\slashed{\partial}(\tilde{\sigma}+i\tilde{\vec{\pi}}\cdot\vec{\tau}\gamma_5)
  \right\}\right)\right]  
  \\
  &=\mathrm{Tr}\left[{\color{red}4\sigma^2_0V\tilde{\sigma}V\tilde{\sigma}}
  +2i\sigma_0V\tilde{\sigma}V\left\{\slashed{\partial}(\tilde{\sigma}+i\tilde{\vec{\pi}}
  \cdot\vec{\tau}\gamma_5)\right\}
  +2i\sigma_0V\left\{\slashed{\partial}(\tilde{\sigma}+i\tilde{\vec{\pi}}
  \cdot\vec{\tau}\gamma_5)\right\}V\tilde{\sigma}
  {\color{red}-V\left\{\slashed{\partial}(\tilde{\sigma}+i\tilde{\vec{\pi}}\cdot\vec{\tau}\gamma_5)
  \right\}}\right.  \\
  &\left.\,\,\,\,\,{\color{red}\times V\left\{\slashed{\partial}(\tilde{\sigma}+i\tilde{\vec{\pi}}\cdot\vec{\tau}
  \gamma_5)\right\}
  +2\sigma_0V\tilde{\sigma}V\left(\tilde{\sigma}^2+\tilde{\vec{\pi}}^2\right)
  +2\sigma_0V\left(\tilde{\sigma}^2+\tilde{\vec{\pi}}^2\right)V\tilde{\sigma}}
  +iV\left(\tilde{\sigma}^2+\tilde{\vec{\pi}}^2\right)V\left\{\slashed{\partial}
  (\tilde{\sigma}+i\tilde{\vec{\pi}}\cdot\vec{\tau}\gamma_5)\right\}\right.  \\
  &\left.\,\,\,\,\,+iV\left\{\slashed{\partial}(\tilde{\sigma}+i\tilde{\vec{\pi}}\cdot\vec{\tau}
  \gamma_5)\right\}V\left(\tilde{\sigma}^2+\tilde{\vec{\pi}}^2\right)
  {\color{red}+V\left(\tilde{\sigma}^2+\tilde{\vec{\pi}}^2\right)V\left(\tilde{\sigma}^2+\tilde{\vec{\pi}}^2\right)}
  \right]  \\
\end{split}
\end{align}
We can erase terms using relations tr$[\gamma^\mu]=0$ and tr$[\gamma^\mu\gamma_5]=0$ in the spin space. 
\begin{align}\label{1-1-2}
  \begin{split}
    \mathrm{Tr}\left[VAVA\right]
  &=\mathrm{Tr}\left[
    4\sigma^2_0V\tilde{\sigma}V\tilde{\sigma}
  -V\left\{\slashed{\partial}(\tilde{\sigma}+i\tilde{\vec{\pi}}\cdot\vec{\tau}\gamma_5)
  \right\}V\left\{\slashed{\partial}(\tilde{\sigma}+i\tilde{\vec{\pi}}\cdot\vec{\tau}
  \gamma_5)\right\}+2\sigma_0\left\{V\tilde{\sigma},V\left(\tilde{\sigma}^2+\tilde{\vec{\pi}}^2\right)
  \right\}\right.\\
  &\left.\,\,\,\,\,+V\left(\tilde{\sigma}^2+\tilde{\vec{\pi}}^2\right)V\left(\tilde{\sigma}^2
  +\tilde{\vec{\pi}}^2\right)\right].
\end{split}
\end{align}
To treat second term, following relation is needed.
\begin{align}\label{1-1-3}
  \begin{split}
    \gamma_5\slashed{\partial}(\tilde{\vec{\pi}}\cdot\vec{\tau})
    \gamma_5\slashed{\partial}(\tilde{\vec{\pi}}\cdot\vec{\tau})
    &=-\slashed{\partial}(\tilde{\vec{\pi}}\cdot\vec{\tau})\slashed{\partial}(\tilde{\vec{\pi}}\cdot\vec{\tau})
    =-\partial_\mu\gamma^\mu\tilde{\pi}_i\tau_i\partial_\nu\gamma^\nu\tilde{\pi}_j\tau_j
    =-\partial_\mu\gamma^\mu\tilde{\pi}_i\partial_\nu\gamma^\nu\tilde{\pi}_j(\delta_{ij}+i\epsilon_{ijk}\tau_k)  \\
    &=-\partial_\mu\gamma^\mu\tilde{\pi}_i\partial_\nu\gamma^\nu\tilde{\pi}_i
      -\epsilon_{ijk}\partial_\mu\gamma^\mu\tilde{\pi}_i\partial_\nu\gamma^\nu\tilde{\pi}_j\tau_k
  \end{split}
\end{align}
Because this is in the trace and tr[$\tau_i$]=0, last term is a zero. Therefore
\begin{align}\label{1-1-3-1}
  \begin{split}
    \gamma_5\slashed{\partial}(\tilde{\vec{\pi}}\cdot\vec{\tau})
    \gamma_5\slashed{\partial}(\tilde{\vec{\pi}}\cdot\vec{\tau})
    =-\slashed{\partial}\tilde{\vec{\pi}}\cdot
    \slashed{\partial}\tilde{\vec{\pi}}
  \end{split}
\end{align}
\begin{align}\label{1-1-4}
  \begin{split}
    \mathrm{Tr}\left[VAVA\right]
    &=\mathrm{Tr}\left[4\sigma^2_0V\tilde{\sigma}V\tilde{\sigma}
    -V\slashed{\partial}\tilde{\sigma}V\slashed{\partial}\tilde{\sigma}
    -V\slashed{\partial}\tilde{\vec{\pi}}\cdot V\slashed{\partial}\tilde{\vec{\pi}}
    -2i\gamma_5V\slashed{\partial}\tilde{\sigma}\cdot\slashed{\partial}\left(\tilde{\vec{\pi}}
    \cdot\vec{\tau}\right)+2\sigma_0\left\{V\tilde{\sigma},V\left(\tilde{\sigma}^2+\tilde{\vec{\pi}}^2\right)
    \right\}\right.\\
    &\left.\,\,\,\,\,+V\left(\tilde{\sigma}^2+\tilde{\vec{\pi}}^2\right)V\left(\tilde{\sigma}^2
    +\tilde{\vec{\pi}}^2\right)
    \right].
  \end{split}
\end{align}
Since tr$[\gamma_5]=0$,
\begin{align}\label{1-1-5}
  \begin{split}
    \mathrm{Tr}\left[VAVA\right]
    &=\mathrm{Tr}\left[4\sigma^2_0V\tilde{\sigma}V\tilde{\sigma}
    -V\slashed{\partial}\tilde{\sigma}V\slashed{\partial}\tilde{\sigma}
    -V\slashed{\partial}\tilde{\vec{\pi}}\cdot V\slashed{\partial}\tilde{\vec{\pi}}
    +2\sigma_0\left\{V\tilde{\sigma},V\left(\tilde{\sigma}^2+\tilde{\vec{\pi}}^2\right)
    \right\}
    \right.\\
    &\left.\,\,\,\,\,
    +V\left(\tilde{\sigma}^2+\tilde{\vec{\pi}}^2\right)V\left(\tilde{\sigma}^2
    +\tilde{\vec{\pi}}^2\right)
    \right].
  \end{split}
\end{align}
From $\phi=(\sigma,\vec{\pi})$,
\begin{align}\label{1-1-6}
  \begin{split}
    &\phi=\phi_0+\tilde{\phi}
    =(\sigma_c=M,0)+(\tilde{\sigma},\tilde{\vec{\pi}}) \\
    &\Longrightarrow \phi_0\tilde{\phi}=\sigma_0\tilde{\sigma},\,\,\,
    \tilde{\phi}\tilde{\phi}=\tilde{\sigma}^2+\tilde{\vec{\pi}}^2
  \end{split}
\end{align}
substituting Eq~\eqref{1-1-5},
\begin{align}\label{1-1-7}
  \begin{split}
    \mathrm{Tr}\left[VAVA\right]
    &=\mathrm{Tr}\left[
    4\phi_0^2V\tilde{\phi}V\tilde{\phi}
    -V\slashed{\partial}\tilde{\phi} 
     V\slashed{\partial}\tilde{\phi}
    +2\left\{V\phi_0\tilde{\phi},V\tilde{\phi}\tilde{\phi}
    \right\}
    +V\tilde{\phi}\tilde{\phi}V\tilde{\phi}\tilde{\phi}
    \right].
  \end{split}
\end{align}
\subsection{Action and Propagator}
Therefore the action Eq~\eqref{1-3} is
\begin{align}\label{1-1-8}
  \begin{split}
    S=-\frac{1}{2}N_c\left[\mathrm{Tr}
    \left[\log\left(V^{-1}\right)\right]
    -\frac{1}{2}\mathrm{Tr}\left[
      4\phi_0^2V\tilde{\phi}V\tilde{\phi}
      -V\slashed{\partial}\tilde{\phi}
       V\slashed{\partial}\tilde{\phi}
      +2\left\{V\phi_0\tilde{\phi},V\tilde{\phi}\tilde{\phi}
      \right\}
      +V\tilde{\phi}\tilde{\phi}V\tilde{\phi}\tilde{\phi}
      \right]\right.\\
      \,\,\,\,\,\left.
        +8I_1(M)\int d^4x\left(\sigma^2+\vec{\pi}^2-M^2\right)\right]
        +\int d^4x\,\frac{1}{2G}\left(\left(\sigma-\hat{m}\right)^2+\vec{\pi}^2\right)
      \end{split}
\end{align}
We already know from the previous calculation in the paper that
\begin{align}
  \begin{split}
    8\sigma_c G N_cI_1(\sigma_c)=\sigma_c-\hat{m}.
  \end{split}
\end{align}
Then last two terms are
\begin{align}
  \begin{split}
    &-4N_cI_1(M)\int d^4x\left(\sigma^2+\vec{\pi}^2-M^2\right)
        +\int d^4x\,\frac{1}{2G}\left(\left(\sigma-\hat{m}\right)^2+\vec{\pi}^2\right)  \\
    &=-\frac{M-\hat{m}}{2MG}\int d^4x\left(\phi^2-M^2\right)
    +\int d^4x\,\frac{1}{2G}\left(\phi^2 -2\sigma\hat{m}+\hat{m}^2\right) \\
    &=\left(\frac{\hat{m}}{2MG}-\frac{1}{2G}\right)\int d^4x\left(
      \phi^2+2\phi\tilde{\phi}+\tilde{\phi}^2-M^2\right)
    +\frac{1}{2G}\int d^4x\,\left(\phi^2 -2\sigma\hat{m}+\hat{m}^2\right) \\
    &=\left(\frac{\hat{m}}{2MG}-\frac{1}{2G}\right)\int d^4x\left(
      2M\tilde{\sigma}+\tilde{\phi}^2\right)
    +\frac{1}{2G}\int d^4x\,\left(M^2+2M\tilde{\sigma}+\tilde{\phi}^2 
    -2\sigma\hat{m}+\hat{m}^2\right) \\
    &=\int d^4x\left(
      \frac{\hat{m}\tilde{\sigma}}{G}+\frac{\hat{m}}{2MG}\tilde{\phi}^2\right)
      -\int d^4x\left(\frac{M\tilde{\sigma}}{G}+\frac{1}{2G}\tilde{\phi}^2\right)
    +\int d^4x\,\left(\frac{M^2}{2G}+\frac{M\tilde{\sigma}}{G}+\frac{1}{2G}\tilde{\phi}^2 
    -\frac{M\hat{m}}{G}+\frac{\tilde{\sigma}\hat{m}}{G}+\frac{\hat{m}^2}{2G}\right) \\
    &=\int d^4x\left(
      \frac{2\hat{m}\tilde{\sigma}-M\hat{m}}{G}
      +\frac{M^2+\hat{m}^2}{2G}\right)
      +\int d^4x\frac{\hat{m}}{2MG}\tilde{\phi}^2.
  \end{split}
\end{align}
We have to keep in mind the last term which will be included the second derivative of the action.
Let us write the action one more time:
\begin{align}
  \begin{split}
    S=-\frac{1}{2}N_c\left[\mathrm{Tr}
    \left[\log\left(V^{-1}\right)\right]
    -\frac{1}{2}\mathrm{Tr}\left[
    4\phi_0^2V\tilde{\phi}V\tilde{\phi}
    -V\slashed{\partial}\tilde{\phi}
     V\slashed{\partial}\tilde{\phi}
    +2\left\{V\phi_0\tilde{\phi},V\tilde{\phi}\tilde{\phi}
    \right\}
    +V\tilde{\phi}\tilde{\phi}V\tilde{\phi}\tilde{\phi}
    \right]\right]\\
    \,\,\,\,\,
    +\int d^4x\left(
    \frac{2\hat{m}\tilde{\sigma}-M\hat{m}}{G}
    +\frac{M^2+\hat{m}^2}{2G}\right)
    +\int d^4x\frac{\hat{m}}{2MG}\tilde{\phi}^2,
  \end{split}
\end{align}
\begin{align}\label{1-1-R}
  S=-\frac{1}{2}N_c\left[\mathrm{Tr}
  \left[\log\left(V^{-1}\right)\right]
  -\frac{1}{2}\mathrm{Tr}\left[
  4\phi_0^2V\tilde{\phi}V\tilde{\phi}
  -V\slashed{\partial}\tilde{\phi}
   V\slashed{\partial}\tilde{\phi}
  +2\left\{V\phi_0\tilde{\phi},V\tilde{\phi}\tilde{\phi}
  \right\}
  +V\tilde{\phi}\tilde{\phi}V\tilde{\phi}\tilde{\phi}
  \right]\right]+R_1+R_2\tilde{\phi}^2.
\end{align}
Since we use saddle-point approximation, the action can be expressed the expansion for $\phi$
with a zero first derivative:
\begin{align}\label{1-1-9}
  \begin{split}
    S[\phi] &\approx S[\phi_0]+\frac{\delta}{\delta\phi}S[\phi_0]\tilde{\phi}
    +\frac{1}{2}\frac{\delta^2}{\delta\phi^2}S[\phi_0]\tilde{\phi}^2\\
    &=S[\phi_0]+\frac{1}{2}\frac{\delta^2}{\delta\phi^2}S[\phi_0]\tilde{\phi}^2.
  \end{split}
\end{align}
Comparing with Eq~\eqref{1-1-R}, we can find the second derivative term
\begin{align}\label{1-1-10}
  \begin{split}
    \frac{1}{2}\frac{\delta^2}{\delta\phi^2}S[\phi_0]\tilde{\phi}^2
    &=-\frac{1}{2}N_c\left[
      -\frac{1}{2}\mathrm{Tr}\left[
      4\phi_0^2V\tilde{\phi}V\tilde{\phi}
      -V\slashed{\partial}\tilde{\phi}
       V\slashed{\partial}\tilde{\phi}
      \right]
    \right]+R_2\tilde{\phi}^2\\
    &=\frac{N_c}{4}\int\frac{d^4p}{(2\pi)^4}
    \mathrm{tr}\left<p\left|4\phi_0^2V\tilde{\phi}V\tilde{\phi}
    -V\slashed{\partial}\tilde{\phi}
     V\slashed{\partial}\tilde{\phi}\right|p\right> +R_2\tilde{\phi}^2\\
     &=2N_c\int\frac{d^4p}{(2\pi)^4}
     \left<p\left|4\phi_0^2V\tilde{\phi}V\tilde{\phi}
     -V\slashed{\partial}\tilde{\phi}
      V\slashed{\partial}\tilde{\phi}\right|p\right>+R_2\tilde{\phi}^2.
  \end{split}
\end{align}
Now we will use the completeness relation:
\begin{align}\label{1-1-11}
  &\mathbb{I} = \frac{d^4p'}{(2\pi)^4}|p'\left>\right<p'|\\
  \begin{split}
    &\Longrightarrow
    \int\frac{d^4p}{(2\pi)^4}
     \left<p\left|4\phi_0^2V\tilde{\phi}V\tilde{\phi}
     -V\slashed{\partial}\tilde{\phi}
      V\slashed{\partial}\tilde{\phi}\right|p\right>
      =\int\frac{d^4p}{(2\pi)^4}\int\frac{d^4p'}{(2\pi)^4}
      \left(
                \left<p
                \left|4\phi_0^2V\tilde{\phi}
                \right|p'
                \right>
              \left<p'
              \left|V\tilde{\phi}
              \right|
              p\right>
      \right.\\
      &\qquad\qquad\qquad\qquad\qquad\qquad\qquad\qquad\qquad
      \qquad\qquad\qquad\qquad\qquad\qquad\qquad
      -\left.
                      \left<p
                      \left|V\slashed{\partial}\tilde{\phi}
                      \right|p'
                      \right>
              \left<p'
              \left|V\slashed{\partial}\tilde{\phi}
              \right|p
              \right>
        \right)
  \end{split}
\end{align}
In the momentum space representation, $V = 1/(p^2+M^2)$. So
\begin{align}\label{1-1-12}
  \begin{split}
    \frac{1}{2}\frac{\delta^2}{\delta\phi^2}S[\phi_0]\tilde{\phi}^2
    &=2N_c\int\frac{d^4p}{(2\pi)^4}\int\frac{d^4p'}{(2\pi)^4}
    \frac{1}{(p^2+M^2)(p'^2+M^2)}
    \left(\left<p\left|4\phi_0^2\tilde{\phi}\right|p'\right>
    \left<p'\left|\tilde{\phi}\right|p\right>
    -\left<p\left|\slashed{\partial}\tilde{\phi}\right|p'\right>
    \left<p'\left|\slashed{\partial}\tilde{\phi}\right|p\right>
    \right)+R_2\tilde{\phi}^2
  \end{split}
\end{align}
Let us calculate term by term.
\begin{align}\label{1-1-13}
  \begin{split}
    \left<p\left|4\phi_0^2\tilde{\phi}\right|p'\right>
    &=4M^2\int d^4x\left<p\left|\tilde{\phi}
    \right|x\right>\left<x\left| \right.p'\right>
    =4M^2\int d^4x\,\tilde{\phi}(x)\left<p\left|
    \right.x\right>\left<x\left| \right.p'\right>
    =4M^2\int d^4x\,\tilde{\phi}(x)e^{ix(p-p')} \\
    &= 4M^2\tilde{\phi}(p-p'), \\
    %%%%%%%%%%%%%%%%%%%%%%%%%%%%%%%%%%%%%%%%%%%%%%%%
    \left<p'\left|\tilde{\phi}\right|p\right>&=\tilde{\phi}(p'-p),  \\
    %%%%%%%%%%%%%%%%%%%%%%%%%%%%%%%%%%%%%%%%%%%%%%%%
    \left<p\left|\slashed{\partial}\tilde{\phi}\right|p'\right>&= 
    \int d^4x\left<p\left|\slashed{\partial}\tilde{\phi}
    \right|x\right>\left<x\left| \right.p'\right>
    =\int d^4x\,\slashed{\partial}\tilde{\phi}(x)\left<p\left|
    \right.x\right>\left<x\left| \right.p'\right>
    =\int d^4x\,\slashed{\partial}\tilde{\phi}(x)e^{ix(p-p')} \\
    &=\int d^4x\,\slashed{\partial}\left(\tilde{\phi}(x)e^{ix(p-p')}\right)
    -\int d^4x\,\tilde{\phi}(x)\slashed{\partial}e^{ix(p-p')}
    =-\int d^4x\,\tilde{\phi}(x)i(p-p')e^{ix(p-p')} \\
    &=-i(p-p')\tilde{\phi}(p-p'),
    % \int d^4x\left<p\right|\slashed{\partial}\left(\tilde{\phi}
    % \left|x\right>\right)-\tilde{\phi}\slashed{\partial}
    % \left|x\right>
    \\
    %%%%%%%%%%%%%%%%%%%%%%%%%%%%%%%%%%%%%%%%%%%%%%%%
    \left<p'\left|\slashed{\partial}\tilde{\phi}\right|p\right>&=  
    -i(p'-p)\tilde{\phi}(p'-p).
  \end{split}
\end{align}
Then Eq~\eqref{1-1-12} is
\begin{align}\label{1-1-14}
  \begin{split}
    \frac{1}{2}\frac{\delta^2}{\delta\phi^2}S[\phi_0]\tilde{\phi}^2
    &=2N_c\int\frac{d^4p}{(2\pi)^4}\int\frac{d^4p'}{(2\pi)^4}
    \frac{4M^2\tilde{\phi}(p-p')
    \tilde{\phi}(p'-p)
  -(p-p')(p'-p)\tilde{\phi}(p-p')
   \tilde{\phi}(p'-p)}{(p^2+M^2)(p'^2+M^2)}+R_2\tilde{\phi}^2.
  \end{split}
\end{align}
For convenience, let us introduce changing variable:
\begin{align}
  p'+p=2k,\,\,\,p'-p=q
\end{align}
and
\begin{align}
  p'=k+\frac{q}{2},\,\,\, p =k-\frac{q}{2}.
\end{align}
Since we change two variables $(p',p)$ to $(k,q)$, 4-integral variables becomes
$d^4kd^4q$. Hence 
\begin{align}\label{1-1-15}
  \begin{split}
    \frac{1}{2}\frac{\delta^2}{\delta\phi^2}S[\phi_0]\tilde{\phi}^2
    &=2N_c\int\frac{d^4k}{(2\pi)^4}\int\frac{d^4q}{(2\pi)^4}
    \frac{4M^2\tilde{\phi}(-q)
    \tilde{\phi}(q)
  +q^2\tilde{\phi}(-q)
   \tilde{\phi}(q)}{\left\{\left(k-\frac{q}{2}\right)^2+M^2\right\}
   \left\{\left(k+\frac{q}{2}\right)^2+M^2\right\}}+R_2\tilde{\phi}^2 \\
   &=2N_c\int\frac{d^4k}{(2\pi)^4}\int\frac{d^4q}{(2\pi)^4}
    \frac{4M^2+q^2}{\left\{\left(k-\frac{q}{2}\right)^2+M^2\right\}
   \left\{\left(k+\frac{q}{2}\right)^2+M^2\right\}}\tilde{\phi}(-q)
   \tilde{\phi}(q)+R_2\tilde{\phi}^2  \\
   &= 2N_c\int\frac{d^4q}{(2\pi)^4}
   (4M^2+q^2)f(q)\tilde{\phi}(-q)\tilde{\phi}(q)+R_2\tilde{\phi}^2,\,\,\,
   f(q) = \int\frac{d^4k}{(2\pi)^4}\frac{1}
   {\left\{\left(k-\frac{q}{2}\right)^2+M^2\right\}
   \left\{\left(k+\frac{q}{2}\right)^2+M^2\right\}}.
  \end{split}
\end{align}
$\tilde{\phi}(q)$ can be expressed into the Fourier transformation
\begin{align}\label{1-1-16}
  \begin{split}
    &\tilde{\phi}(-q)\tilde{\phi}(q)
    =\int d^4x\int d^4y\,
    \tilde{\phi}(y)e^{iq(x-y)}\tilde{\phi}(x)  \\
    &\Longrightarrow 
    \frac{1}{2}\frac{\delta^2}{\delta\phi^2}S[\phi_0]\tilde{\phi}^2
    =2N_c
    \int d^4x\int d^4y
    \int\frac{d^4q}{(2\pi)^4}(4M^2+q^2)f(q)
    \tilde{\phi}(y)e^{iq(x-y)}\tilde{\phi}(x)\\
    &\qquad\qquad\qquad\qquad\qquad\qquad\qquad\qquad\qquad\qquad\qquad\qquad\qquad\qquad
    +\int d^4x\int d^4y\frac{\hat{m}}{2MG}\delta^{(4)}(x-y)\tilde{\phi}(y)\tilde{\phi}(x).
    \end{split}
\end{align}
Actually, the exact expression of the second derivative is
\begin{align}\label{1-1-17}
  \frac{1}{2}\frac{\delta^2}{\delta\phi^2}S[\phi_0]\tilde{\phi}^2
  =\frac{1}{2}\int d^4x\int d^4y\left.\frac{\delta^2S}{\delta\phi(x)\delta\phi(y)}\right|_{\phi^0}
  \tilde{\phi}(x)\tilde{\phi}(y) 
\end{align}
and this form is seem to be comparable with the LHS of Eq~\eqref{1-1-16}. Therefore
\begin{align}
  \begin{split}
    \frac{1}{2}\left.\frac{\delta^2S}{\delta\phi(x)\delta\phi(y)}\right|_{\phi^0}
    &=2N_c
    \int\frac{d^4q}{(2\pi)^4}(4M^2+q^2)f(q)e^{iq(x-y)}
    +\frac{\hat{m}}{2MG}\delta^{(4)}(x-y)\\
    &=\frac{1}{2}\int\frac{d^4q}{(2\pi)^4}
    \left[
      4N_c(4M^2+q^2)f(q)
      +\frac{\hat{m}}{MG}
    \right]e^{iq(x-y)}.
  \end{split}
\end{align}
Finally the meson propagator $K_\phi(q^2)$ is
\begin{align}
  \begin{split}
  &\left.\frac{\delta^2S}{\delta\phi(x)\delta\phi(y)}\right|_{\phi^0}
    =\int\frac{d^4q}{(2\pi)^4}K^{-1}(q^2)e^{iq(x-y)}
    =\int\frac{d^4q}{(2\pi)^4}
    \left[
      4N_c(4M^2+q^2)f(q)
      +\frac{\hat{m}}{MG}
    \right]e^{iq(x-y)}  \\
    &= K(q^2)=\frac{1}{ 4N_c(4M^2+q^2)f(q)
    +\frac{\hat{m}}{MG}}
    =\frac{1}{Z_q(q^2)}\frac{1}{4M^2+q^2
    +\frac{\hat{m}}{MGZ_q(q^2)}},\,\,
    Z_q(q^2)=4N_cf(q).
  \end{split}
\end{align}
\subsection{$Z_q(q^2)$ Calculation}
Now only one process is remain: calculating $Z_q(q^2)$.
From Eq~\eqref{1-1-15},
\begin{align}
  \begin{split}
    Z_q(q^2)&=4N_c\int\frac{d^4k}{(2\pi)^4}\frac{1}
    {\left\{\left(k-\frac{q}{2}\right)^2+M^2\right\}
    \left\{\left(k+\frac{q}{2}\right)^2+M^2\right\}}  \\
    &=4N_c\int\frac{d^4k}{(2\pi)^4}\frac{1}
    {\left(k^2+M^2+\frac{q^2}{4}-qk\right)
     \left(k^2+M^2+\frac{q^2}{4}+qk\right)}.
    \\
  \end{split}
\end{align}
There are useful integral rules:
\begin{align}
  \frac{1}{AB}=\int^1_0dx\,\frac{1}{[xA+(1-x)B]^2},\,\,\,
  \frac{1}{Q^2}=\int_0^\infty du\,ue^{-uQ}.
\end{align}
Let us $A = \left(k^2+M^2+\frac{q^2}{4}-qk\right)$
and $B = \left(k^2+M^2+\frac{q^2}{4}+qk\right)$. substituting
to the first rule,
\begin{align}
  \begin{split}
    Z_q(q^2)
    &=4N_c\int\frac{d^4k}{(2\pi)^4}
    \int^1_0dx
    \left[
      x\left(k^2+M^2+\frac{q^2}{4}-qk\right)
      +(1-x)\left(k^2+M^2+\frac{q^2}{4}+qk\right)
      \right]^{-2}  \\
      &=4N_c\int\frac{d^4k}{(2\pi)^4}
    \int^1_0dx
    \left[
      -2qkx
      +\left(k^2+M^2+\frac{q^2}{4}+qk\right)
      \right]^{-2}  \\
      &= 4N_c\int\frac{d^4k}{(2\pi)^4}
      \int^1_0dx
      \frac{1}{(Kx+B)^2}
      = 4N_c\int\frac{d^4k}{(2\pi)^4}
      \int^1_0dx
      \frac{1}{(Kx+B)^2}
  \end{split}
\end{align}
where $K = -2qk$ and $B = k^2+M^2+\frac{q^2}{4}+qk$. By the second rule,
\begin{align}
  \begin{split}
    Z_q(q^2)
    &= 4N_c\int\frac{d^4k}{(2\pi)^4}
    \int^1_0dx
    \int^\infty_0du\,u\exp\left[-u\left(-2qkx
    +k^2+M^2+\frac{q^2}{4}+qk\right)\right]  \\
    &= 4N_c
    \int^1_0dx
    \int^\infty_0du\,u
    \exp\left[-\frac{u}{4}\left\{
      1-(1-2x)^2
    \right\}q^2-uM^2\right]
    \int\frac{d^4k}{(2\pi)^4}
    \exp\left[-u\left(k+\frac{1}{2}q(1-2x)\right)^2\right].
  \end{split}
\end{align}
For convenience, let $1-2x=y$,  and $-2dx=dy$
\begin{align}
  \begin{split}
    Z_q(q^2)
    &= 4N_c
    \int^1_0dx
    \int^\infty_0du\,u
    \exp\left[-\frac{u}{4}\left\{
      1-(1-2x)^2
    \right\}q^2-uM^2\right]
    \int\frac{d^4k}{(2\pi)^4}
    \exp\left[-u\left(k+\frac{1}{2}q(1-2x)\right)^2\right]  \\
    &=-2N_c
    \int^{-1}_1dy
    \int^\infty_0du\,u
    \exp\left[-\frac{u}{4}\left(
      1-y^2
      \right)q^2-uM^2\right]
      \int\frac{d^4k}{(2\pi)^4}
      \exp\left[-u\left(k+\frac{1}{2}qy\right)^2\right] \\
    &= 2N_c
    \int_{-1}^1dy
    \int^\infty_0du\,u
    \exp\left[-\frac{u}{4}\left(
      1-y^2
      \right)q^2-uM^2\right]
      \int\frac{d^4k}{(2\pi)^4}
      \exp\left[-u\left(k+\frac{1}{2}qy\right)^2\right].
  \end{split}
\end{align}
The last integral is the 4-dimensional spherical integral. 
Moving the center of the gaussian function to the zero-point,
\begin{align}
  \begin{split}
    \int\frac{d^4k}{(2\pi)^4}
      \exp\left[-u\left(k+\frac{1}{2}qy\right)^2\right]
      &=\int\frac{d^4k}{(2\pi)^4}
      e^{-uk^2} 
      = 2\pi^2\int^\infty_0\frac{dk}{(2\pi)^4}
      k^3e^{-uk^2}
      = \frac{2}{(4\pi)^2}\frac{1}{2u}.
      \end{split}
\end{align}

Hence
\begin{align}
  \begin{split}
    Z_q(q^2)&=\frac{2N_c}{(4\pi)^2}
    \int_{-1}^1dy
    \int^\infty_0du\,u
    \exp\left[-\frac{u}{4}\left(
      1-y^2
      \right)q^2-uM^2\right]
      \frac{1}{(4\pi)^2}\frac{1}{u}  \\
      &=\frac{2N_c}{(4\pi)^2}
      \int_{-1}^1dy
      \int^\infty_0du\,
        \left(\frac{1}{u}e^{-u\left(\frac{q^2}{4}\left(
          1-y^2
          \right)+M^2\right)}\right)  \\
          &=\frac{4N_c}{(4\pi)^2}
          \int_{0}^1dy
          \int^\infty_0du\,
            \left(\frac{1}{u}e^{-u\left(\frac{q^2}{4}\left(
              1-y^2
              \right)+M^2\right)}\right).
  \end{split}
\end{align}

This is the proper-time expression.


\end{document}

