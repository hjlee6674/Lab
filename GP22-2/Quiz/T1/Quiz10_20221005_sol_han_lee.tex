%\documentclass[preprint,tightenlines,showpacs,showkeys,floatfix,
%nofootinbib,superscriptaddress,fleqn]{revtex4} 
\documentclass[tightenlines,floatfix,nofootinbib,superscriptaddress,fleqn]{revtex4}  
%\documentclass[aps,epsfig,tightlines,fleqn]{revtex4}
\usepackage{kotex}
\usepackage[HWP]{dhucs-interword}
\usepackage[dvips]{color}
\usepackage{graphicx}
\usepackage{bm}
%\usepackage{fancyhdr}
%\usepackage{dcolumn}
\usepackage{defcolor}
\usepackage{amsmath}
\usepackage{amsfonts}
\usepackage{amssymb}
\usepackage{amscd}
\usepackage{amsthm}
\usepackage[utf8]{inputenc}
%\pagestyle{fancy}

\begin{document}

\title{\Large 2022년 2학기 물리학 II}
\author{Byeong-woo Han}
\email{12191964@inha.edu}
\affiliation{Hadron Theory Group, Department of Physics,
  Inha  University, Incheon 22212, Republic of Korea }
  \author{Hui-Jae Lee}
\email{hjlee6674@inha.edu}
\affiliation{Hadron Theory Group, Department of Physics,
  Inha  University, Incheon 22212, Republic of Korea }
\author{김현철\footnote{Office: 5S-436D (면담시간 매주
    수요일-16:15$\sim$19:00)}} 
\email{hchkim@inha.ac.kr}
\affiliation{Hadron Theory Group, Department of Physics,
  Inha  University, Incheon 22212, Republic of Korea }
\date{Autumn Semester, 2022}

\maketitle

\section*{\large Mock test 1 }
\noindent {\bf 1번 풀이 :} 쿨롱의 법칙에 의하면 두 전하 간 작용하는 전기력의 크기는
\begin{align}
  F = \frac{1}{4\pi \epsilon_0}\frac{q_1q_2}{r^2}
\end{align}
이다.
\\

\noindent {\bf 2번 풀이 :} 
\begin{itemize}
  \item[(1)] 전하에는 $+$전하와 $-$전하 총 2종류가 있다.
  \item[(2)] 두 전하 사이에 작용하는 전기력은 두 전하를 잇는 직선 상에서 작용한다.
  \item[(3)] 쿨롱의 법칙에 의해 전기력은 전하 사이 거리의 제곱에 반비례 한다.
  \item[(4)] 쿨롱의 법칙에 의해 전기력은 전하량의 곱에 비례한다.
  \item[(5)] 중첩의 원리에 의해 둘 이상의 전하가 존재할 때 각 전하에 의한 전기력을 합하여
  합력을 구할 수 있다.
  \item[(6)] 전기장 벡터 $\vec{E}$와 전위 $V$는
  \begin{align}
    \vec{E} = -\nabla V
  \end{align}
  이고 전기력은 전기장에 평행하므로 전기력은 등전위선의 접선 방향에 평행하다.
\end{itemize}

\noindent {\bf 3번 풀이 :} 
\begin{itemize}
  \item[(ㄱ)] 도체관 사이 전기장은 $x$축에 평행하고 $-$극으로 대전된 도체관을 향한다.
  따라서 $A$에서 $B$로 전하를 이동시키면 $+$로 대전된 도체관에 대해 평행하게 멀어지므로 
  전하의 전기적 위치 에너지가 감소한다. 즉, 전기 위치에너지의 변화량의 부호는 $-$이다.
  \item[(ㄴ)] $C$에서 $D$로 전하를 이동시키면 $+$로 대전된 도체관과의 거리가 변하지 않으므로
  전하의 전기적 위치 에너지가 변하지 않는다. 즉, 전기 위치에너지의 변화량의 부호는 $0$이다.
  \item[(ㄷ)]  $B$에서 $D$로 전하를 이동시키면 $+$로 대전된 도체관에 가까워지므로
  전하의 전기적 위치 에너지는 증가한다. 즉, 전기 위치에너지의 변화량의 부호는 $+$이다.
\end{itemize}

\noindent {\bf 4번 풀이 :} 
계의 전기 위치에너지는 거리가 무한대인 지점으로부터 전하들을 끌어올 때 필요한 에너지로 
정의한다. 아무 전하도 없는 공간에서 전하를 끌어오는데 필요한 에너지는 $0$이다. 그리고 다른 
전하를 또 끌어오는데 필요한 에너지 $E_1$는
\begin{align}
  E_1 = -\int _\infty^d k\frac{q^2}{r^2}\,dr 
  = \left.\left(k\frac{q^2}{r}\right)\right| _\infty^d = k\frac{q^2}{d}
\end{align}
이다. 이제 마지막 전하를 끌어오는데 필요한 에너지 $E_2$를 구해보자. 이미 전하 2개가 있으므로
\begin{align}
  E_2 = -2\int _\infty^d k\frac{q^2}{r^2}\,dr  = 2k\frac{q^2}{d}
\end{align}
이고 이 계의 전기 위치에너지 $E$는 전하들을 끌어오는데 필요한 에너지들의 합이므로
\begin{align}
  E=E_1+E_2 = 3k\frac{q^2}{d}
\end{align}
이다.
\\

\noindent {\bf 5번 풀이 :} 
가우스 법칙을 수식으로 정리하면
\begin{align}
  \int \vec{E}\cdot d\vec{a} =\frac{q}{\epsilon_0}
\end{align}
이다. 좌항은 폐곡면을 지나는 전기선속의 합이고 우항은 폐곡면 내부에 존재하는 총 전하량에 대한
항이다.
\\

\noindent {\bf 6번 풀이 :} 
전기장의 크기를 구하기 위해 가우스 법칙을 이용하자. 가우스 곡면을 반지름이 $r$인 구의 표면으로
하면 가우스 곡면 내부 총 전하량 $q$는 전체 전하량 중 가우스 곡면 내부 부피에 존재하는 전하량에
해당하므로
\begin{align}
 q = Q\frac{\frac{4}{3}\pi r^3}{\frac{4}{3}\pi R^3}=Q\frac{r^3}{R^3}
\end{align}
이고 구 대칭성에 의해 미소 면벡터와 전기장의 방향이 일치한다고 생각할 수 있다. 
\begin{align}
  \int \vec{E}\cdot d\vec{a} = \left|\vec{E}\right|\int d\vec{a}.
  =4\pi r^2\left|\vec{E}\right|
\end{align}
따라서 전기장의 크기 $\left|\vec{E}\right|$는
\begin{align}
  \left|\vec{E}\right| = \frac{Qr}{4\pi \epsilon_0 R^3}
\end{align}
이다.
\\

\noindent {\bf 7번 풀이 :} 
전기장의 크기는 각 도체 평면에 의한 전기장의 크기의 합이다. 무한히 넓은 도체 평면에 의한
전기장은 거리에 무관하게 일정하므로 도체 평면 1과 2에 의한 전기장은 서로 상쇄된다. 도체 평면
3에 의한 전기장 $E$는
\begin{align}
  E = \frac{\sigma}{\epsilon_0}
\end{align}
이고 이것이 평면 2와 3 사이 영역에서의 전기장의 크기이다.
\\

\noindent {\bf 8번 풀이 :} 
평행판 축전기에 충전된 전하량을 $Q$, 축전기 사이 전위차를 $V$라 하면 전기용량 $C$는
\begin{align}
  Q = CV,\,\,\,C = \epsilon_0\frac{A}{d}
\end{align}
이다. $A$는 평행판 축전기의 면적이고 $d$는 평행판 축전기의 간격이다.
\\

\noindent {\bf 9번 풀이 :} 
\\

\noindent {\bf 10번 풀이 :} 
\\

\noindent {\bf 11번 풀이 :} 
\\

\noindent {\bf 12번 풀이 :} 
\\

\noindent {\bf 서술형 1번 풀이 :} 
\\

\noindent {\bf 서술혈 2번 풀이 :} 
\\

\noindent {\bf 서술횽 3번 풀이 :} 

\end{document}

