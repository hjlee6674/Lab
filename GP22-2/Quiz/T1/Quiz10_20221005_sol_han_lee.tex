%\documentclass[preprint,tightenlines,showpacs,showkeys,floatfix,
%nofootinbib,superscriptaddress,fleqn]{revtex4} 
\documentclass[tightenlines,floatfix,nofootinbib,superscriptaddress,fleqn]{revtex4}  
%\documentclass[aps,epsfig,tightlines,fleqn]{revtex4}
\usepackage{kotex}
\usepackage[HWP]{dhucs-interword}
\usepackage[dvips]{color}
\usepackage{graphicx}
\usepackage{bm}
%\usepackage{fancyhdr}
%\usepackage{dcolumn}
\usepackage{defcolor}
\usepackage{amsmath}
\usepackage{amsfonts}
\usepackage{amssymb}
\usepackage{amscd}
\usepackage{amsthm}
\usepackage[utf8]{inputenc}
%\pagestyle{fancy}

\begin{document}

\title{\Large 2022년 2학기 물리학 II}
\author{Byeong-woo Han}
\email{12191964@inha.edu}
\affiliation{Hadron Theory Group, Department of Physics,
  Inha  University, Incheon 22212, Republic of Korea }
  \author{Hui-Jae Lee}
\email{hjlee6674@inha.edu}
\affiliation{Hadron Theory Group, Department of Physics,
  Inha  University, Incheon 22212, Republic of Korea }
\author{김현철\footnote{Office: 5S-436D (면담시간 매주
    수요일-16:15$\sim$19:00)}} 
\email{hchkim@inha.ac.kr}
\affiliation{Hadron Theory Group, Department of Physics,
  Inha  University, Incheon 22212, Republic of Korea }
\date{Autumn Semester, 2022}

\maketitle

\section*{\large Mock test 1 }
\noindent {\bf 1번 풀이 :} 쿨롱의 법칙에 의하면 두 전하 간 작용하는 전기력의 크기는
\begin{align}
  F = \frac{1}{4\pi \epsilon_0}\frac{q_1q_2}{r^2}
\end{align}
이다.
\\

\noindent {\bf 2번 풀이 :} 
\begin{itemize}
  \item[(1)] 전하에는 $+$전하와 $-$전하 총 2종류가 있다.
  \item[(2)] 두 전하 사이에 작용하는 전기력은 두 전하를 잇는 직선 상에서 작용한다.
  \item[(3)] 쿨롱의 법칙에 의해 전기력은 전하 사이 거리의 제곱에 반비례 한다.
  \item[(4)] 쿨롱의 법칙에 의해 전기력은 전하량의 곱에 비례한다.
  \item[(5)] 중첩의 원리에 의해 둘 이상의 전하가 존재할 때 각 전하에 의한 전기력을 합하여
  합력을 구할 수 있다.
  \item[(6)] 전기장 벡터 $\vec{E}$와 전위 $V$는
  \begin{align}
    \vec{E} = -\nabla V
  \end{align}
  이고 전기력은 전기장에 평행하므로 전기력은 등전위선의 접선 방향에 평행하다.
\end{itemize}

\noindent {\bf 3번 풀이 :} 
\begin{itemize}
  \item[(ㄱ)] 도체관 사이 전기장은 $x$축에 평행하고 $-$극으로 대전된 도체관을 향한다.
  따라서 $A$에서 $B$로 전하를 이동시키면 $+$로 대전된 도체관에 대해 평행하게 멀어지므로 
  전하의 전기적 위치 에너지가 감소한다. 즉, 전기 위치에너지의 변화량의 부호는 $-$이다.
  \item[(ㄴ)] $C$에서 $D$로 전하를 이동시키면 $+$로 대전된 도체관과의 거리가 변하지 않으므로
  전하의 전기적 위치 에너지가 변하지 않는다. 즉, 전기 위치에너지의 변화량의 부호는 $0$이다.
  \item[(ㄷ)]  $B$에서 $D$로 전하를 이동시키면 $+$로 대전된 도체관에 가까워지므로
  전하의 전기적 위치 에너지는 증가한다. 즉, 전기 위치에너지의 변화량의 부호는 $+$이다.
\end{itemize}

\noindent {\bf 4번 풀이 :} 
계의 전기 위치에너지는 거리가 무한대인 지점으로부터 전하들을 끌어올 때 필요한 에너지로 
정의한다. 아무 전하도 없는 공간에서 전하를 끌어오는데 필요한 에너지는 $0$이다. 그리고 다른 
전하를 또 끌어오는데 필요한 에너지 $E_1$는
\begin{align}
  E_1 = -\int _\infty^d k\frac{q^2}{r^2}\,dr 
  = \left.\left(k\frac{q^2}{r}\right)\right| _\infty^d = k\frac{q^2}{d}
\end{align}
이다. 이제 마지막 전하를 끌어오는데 필요한 에너지 $E_2$를 구해보자. 이미 전하 2개가 있으므로
\begin{align}
  E_2 = -2\int _\infty^d k\frac{q^2}{r^2}\,dr  = 2k\frac{q^2}{d}
\end{align}
이고 이 계의 전기 위치에너지 $E$는 전하들을 끌어오는데 필요한 에너지들의 합이므로
\begin{align}
  E=E_1+E_2 = 3k\frac{q^2}{d}
\end{align}
이다.
\\

\noindent {\bf 5번 풀이 :} 
가우스 법칙을 수식으로 정리하면
\begin{align}
  \int \vec{E}\cdot d\vec{a} =\frac{q}{\epsilon_0}
\end{align}
이다. 좌항은 폐곡면을 지나는 전기선속의 합이고 우항은 폐곡면 내부에 존재하는 총 전하량에 대한
항이다.
\\

\noindent {\bf 6번 풀이 :} 
전기장의 크기를 구하기 위해 가우스 법칙을 이용하자. 가우스 곡면을 반지름이 $r$인 구의 표면으로
하면 가우스 곡면 내부 총 전하량 $q$는 전체 전하량 중 가우스 곡면 내부 부피에 존재하는 전하량에
해당하므로
\begin{align}
 q = Q\frac{\frac{4}{3}\pi r^3}{\frac{4}{3}\pi R^3}=Q\frac{r^3}{R^3}
\end{align}
이고 구 대칭성에 의해 미소 면벡터와 전기장의 방향이 일치한다고 생각할 수 있다. 
\begin{align}
  \int \vec{E}\cdot d\vec{a} = \left|\vec{E}\right|\int d\vec{a}.
  =4\pi r^2\left|\vec{E}\right|
\end{align}
따라서 전기장의 크기 $\left|\vec{E}\right|$는
\begin{align}
  \left|\vec{E}\right| = \frac{Qr}{4\pi \epsilon_0 R^3}
\end{align}
이다.
\\

\noindent {\bf 7번 풀이 :} 
전기장의 크기는 각 도체 평면에 의한 전기장의 크기의 합이다. 무한히 넓은 도체 평면에 의한
전기장은 거리에 무관하게 일정하므로 도체 평면 1과 2에 의한 전기장은 서로 상쇄된다. 도체 평면
3에 의한 전기장 $E$는
\begin{align}
  E = \frac{\sigma}{\epsilon_0}
\end{align}
이고 이것이 평면 2와 3 사이 영역에서의 전기장의 크기이다.
\\

\noindent {\bf 8번 풀이 :} 
평행판 축전기에 충전된 전하량을 $Q$, 축전기 사이 전위차를 $V$라 하면 전기용량 $C$는
\begin{align}
  Q = CV,\,\,\,C = \epsilon_0\frac{A}{d}
\end{align}
이다. $A$는 평행판 축전기의 면적이고 $d$는 평행판 축전기의 간격이다. 평행판 축전기의 전기적
위치에너지 $U$는
\begin{align}
  E = \frac{1}{2}QV
\end{align}
이다. 거리 $d$를 4배로 늘릴 경우의 전기용량, 전위차, 전기적 위치에너지를 각각 $C'$, 
$V'$, $U'$라고 하자. 거리를 늘리더라도 면적과 전하량은 변화가 없으므로 전하 $Q$와 
전하밀도 $\sigma$는 일정하다. $C'$, $V'$, $U'$는
\begin{align}
  \begin{split}
    C'& = \epsilon_0\frac{A}{4d} = \frac{1}{4}C \\
    V'& = \frac{Q}{C'} = 4\frac{Q}{C} = 4V \\
    U'& = \frac{1}{2}QV' = 4\left(\frac{1}{2}QV\right) = 4U\\ 
  \end{split}
\end{align}
이다. 따라서 전기용량은 $\frac{1}{4}$배, 전위차는 4배, 전하밀도는 1배, 저장된 에너지는 4배가
된다.
\\

\noindent {\bf 9번 풀이 :} 
우선 줄의 법칙을 이용해 전구의 저항 $R$을 구하자. 전구에 걸린 전압을 $V$, 전구가 소모하는 
전력을 $P$라고 하면 줄의 법칙에 의해
\begin{align}
  P = \frac{V^2}{R} \Longrightarrow R = \frac{V^2}{P} 
  = \frac{(220~\mathrm{V})^2}{440~\mathrm{W}} = 110~\mathrm{\Omega}
\end{align}
전구의 저항은 $110~\mathrm{\Omega}$이다.
전구에 전압 $110$ V를 연결하면 전구가 소모하는 전력 $P$는
\begin{align}
  P = \frac{V^2}{R} = \frac{(110~\mathrm{V})^2}{110~\mathrm{\Omega}}
  =110~\mathrm{W}
\end{align}
와 같다. 따라서 전구는 전력 110 W를 소모한다.
\\

\noindent {\bf 10번 풀이 :} 점전하 $+q$가 정지해있으므로 점전하에 작용하는 합력은 $0$이고
위쪽으로 전기력, 아래쪽으로 중력이 작용한다. 점전하에 작용하는 전기력을 $F_q$, 중력을 $F_g$
라고 하면 점전하에 대한 운동방정식으로부터
\begin{align}\label{eq:10-1}
  \sum F = F_q - F_g = qE - mg =0\Longrightarrow E= \frac{mg}{q}
\end{align}
를 얻는다. $E$는 도체판 사이에 생성되는 전기장의 크기이다. 도체판 사이의 간격을 $d$라 하면 
전위차 $V$는
\begin{align}
  V = Ed
\end{align}
로 주어지므로 식~\eqref{eq:10-1}를 대입하면
\begin{align}
  V = \frac{mg}{q}d
\end{align}
와 같이 쓸 수 있다. $m = 4\times 10^{-13}$ kg, $q = 4.9 \times 10^{-18}$ C, 
$d = 2\times 10^{-2}$ m이므로 $V$는 
\begin{align}
  V = \frac{(4\times 10^{-13}~\mathrm{kg})(9.8~\mathrm{m/s^2})}
  {4.9 \times 10^{-18}~\mathrm{C}}(2\times 10^{-2}~\mathrm{m})
  =1.6\times 10^4~\mathrm{V}
\end{align}
이다.
\\

\noindent {\bf 11번 풀이 :} 
두개의 평행한 도선에 같은 방향으로 전류가 흐를 때 도선에 작용하는 힘의 크기 $F$는
\begin{align}
  F = \frac{\mu_0 I^2}{2\pi d}
\end{align}
이다. $d$는 도선 사이 간격이다. 두 도선에 흐르는 전류량이 각각 2배로 늘어나면
힘은 4배로 증가한다. 따라서 도선에 작용하는 힘의 변화가 없으려면 도선 사이 거리가
4배로 늘어나야 한다.
\\

\noindent {\bf 12번 풀이 :} 
원형도선의 중심에서는 직선도선에 의한 자기장의 크기 $B_1$과 원형도선에 의한 자기장의 크기 
$B_2$가 모두 같은 방향으로 작용하므로 두 자기장을 모두 구한 후 합해주어야 한다. 원형도선의
반지름은 $R$이고 원형도선의 중심은 직선도선으로부터 $R$만큼 떨어져 있으므로
직선도선에 의한 자기장의 크기 $B_1$은
\begin{align}
  B_1 = \frac{\mu_0 i}{2\pi R}
\end{align}
이고 원형도선에 의한 자기장의 크기 $B_2$는
\begin{align}
  B_2 = \frac{\mu_0 i}{2R}
\end{align}
이므로 총 자기장의 크기 $B$는
\begin{align}
  B = B_1+B_2 = \left(1+\frac{1}{\pi}\right)\frac{\mu_0 i}{2R}
\end{align}
이다.
\\

\noindent {\bf 서술형 1번 풀이 :} 
\begin{itemize}
  \item[(가)]
  암페어의 법칙은 임의의 폐곡선을 따라 흐르는 자기장은 폐곡선 내 전류의 합과 비례한다는 
  법칙이다.
  \begin{align}
    \int \vec{B}\cdot d\vec{l} = \mu_0 I
  \end{align}
  폐곡선을 직선 도선의 중심을 중심으로 하고 반지름이 $r$인 원으로 잡자. 직선 도선에 의한 
  자기장 $\vec{B}$는 폐곡선을 따르는 미소길이 $d\vec{l}$과 평행하게 생성되므로 좌항은
  \begin{align}
    \int \vec{B}\cdot d\vec{l} =B(r)\int d\vec{l} = 2\pi r B(r)
  \end{align}
  이 되고 폐곡선 내부에 전류는 $I$만큼 흐르므로 앙페르 법칙에 의해
  \begin{align}
  2\pi r B(r) = \mu_0 I\Longrightarrow B(r) = \frac{\mu_0 I}{2\pi r}
  \end{align}
  을 얻는다.
  \item[(나)]
  폐곡선의 반지름이 $R$보다 작으므로 이 경우 폐곡선 내부에 흐르는 전류 $I'$는 
  \begin{align}
    I' = \frac{\pi r^2}{\pi R^2} I = \frac{r^2}{R^2} I
  \end{align}
  이고 앙페르 법칙으로부터
  \begin{align}
    2\pi r B(r) = \mu_0 I' = \frac{r^2\mu_0 I}{R^2}
    \Longrightarrow B(r) = \frac{r\mu_0 I}{2\pi R^2}
  \end{align}
  를 얻는다.
  \item[(다)]
  도선의 길이를 $L$이라 하고 도선의 전류가 $z$방향에 평행하게 흐르며 두 도선이  가정하자. 
  
  

\end{itemize}

\noindent {\bf 서술혈 2번 풀이 :} 

\\

\noindent {\bf 서술횽 3번 풀이 :} 

\end{document}

