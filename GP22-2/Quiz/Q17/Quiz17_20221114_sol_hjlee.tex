%\documentclass[preprint,tightenlines,showpacs,showkeys,floatfix,
%nofootinbib,superscriptaddress,fleqn]{revtex4} 
\documentclass[tightenlines,floatfix,nofootinbib,superscriptaddress,fleqn]{revtex4} 
%\documentclass[aps,epsfig,tightlines,fleqn]{revtex4}
\usepackage{kotex}
\usepackage[HWP]{dhucs-interword}
\usepackage[dvips]{color}
\usepackage{graphicx}
\usepackage{bm}
%\usepackage{fancyhdr}
%\usepackage{dcolumn}
\usepackage{defcolor}
\usepackage{amsmath}
\usepackage{amsfonts}
\usepackage{amssymb}
\usepackage{amscd}
\usepackage{amsthm}
\usepackage[utf8]{inputenc}
\usepackage{tikz}
%\pagestyle{fancy}

\begin{document}

\title{\Large 2022년 2학기 물리학 II}
\author{Hui-Jae Lee} 
\email{hjlee6674@inha.edu}
\affiliation{Hadron Theory Group, Department of Physics,
  Inha  University, Incheon 22212, Republic of Korea }
\date{Autumn Semester, 2022}
\author{김현철\footnote{Office: 5S-436D (면담시간 매주
    수요일-16:15$\sim$19:00)}} 
\email{hchkim@inha.ac.kr}
\affiliation{Hadron Theory Group, Department of Physics,
  Inha  University, Incheon 22212, Republic of Korea }
\date{Autumn Semester, 2022}
\maketitle


\section*{\large Quiz 17}
\noindent {\bf 문제 1 [30pt].}
$S$ 관성좌표계에서 $\vec{v}$의 속도로 움직이고 있는 입자가 있다.
\begin{itemize}
\item[(가)]  $S$ 좌표계에 대해 $x$ 방향으로 상대속도 $u$로 움직이고
  있는 관성좌표계 $S'$에서 이 입자의 속도의 각 성분을 구하여라. 
\item[(나)] $v_x=c$일 때, $v_x'=c$임을 보여라. 
\end{itemize}

\noindent {\bf 풀이. }
\begin{itemize}
  \item[(가)] $S'$ 관성좌표계에서 관측한 입자의 속도를 $\vec{v}'$라고 하자. 이 좌표계는
  $S$ 관성좌표계에 대해 $x$축 방향으로 $u$의 상대속도로 움직이고 있으므로 로렌츠의 속도
  변환 공식에 의해 속도 $\vec{v}$, $\vec{v}'$의 성분들은 다음과 같은 관계에 있다.
  \begin{align}\label{eq:1-1}
    \begin{split}
      v'_x &=\frac{v_x-u}{1-v_x u/c^2},  \\
      v'_y &=\frac{v_y}{\gamma(1-v_x u/c^2)}, \\
      v'_z &=\frac{v_z}{\gamma(1-v_x u/c^2)} .\\
    \end{split}
  \end{align}
  여기서 $\gamma$는 로렌츠 인자로 $\gamma=(1-v_x^2/c^2)^{-1/2}$이다.
  \item[(나)]
  식~\eqref{eq:1-1}의 $x$방향 성분 변환 관계로부터 $v_x=c$이면
  \begin{align}
    v'_x &=\frac{c-u}{1-c u/c^2}=\frac{c-u}{1-u/c}=c\frac{1-u/c}{1-u/c}=c
  \end{align}
  $v_x'=c$임을 알 수 있다.
\end{itemize}

\vspace{1cm}

\noindent {\bf 문제 2 [70pt].}
로렌츠 변환을 자세히 유도하여라. 즉,
\begin{align}
  \label{eq:1}
  x' = \frac{x-ut}{\sqrt{1-u^2/c^2}},\;\;\; y'=y,\;\;\;z'=z,\;\;\;
  t' = \frac{t-ux/c^2}{\sqrt{1-u^2/c^2}}
\end{align}
임을 자세히 보여라. 

\noindent {\bf 풀이. }
\begin{figure}[htp]
  \centering
  \begin{tikzpicture}
    
    \coordinate (S) at (0,0);
    \coordinate (S') at (3,0);
    \coordinate (A) at (5,3);
    \draw[-latex] (S) -- +(0,4) node[above,left=2]{$y$};
    \draw[-latex] (S) -- +(4,0) node[above]{$x$};
    \draw[-latex] (S') -- +(0,4) node[above,left=2]{$y'$};;
    \draw[-latex] (S') -- +(4,0) node[above]{$x'$};;
    \draw node[left=5,below] at (0,0) {$S$};
    \draw node[left=5,below] at (3,0) {$S'$};
    \filldraw[black] (A) circle (1pt) node[right] {$A$};
    \draw[-latex] (S) -- (A)  node[left=75,below=25]{$\vec{r}$}; 
    \draw[-latex] (S') -- (A) node[left=40,below=35]{$\vec{r}'$}; 
  \end{tikzpicture}
  \caption{두 관성좌표계 $S$, $S'$}
  \label{fig:2-1}
\end{figure}
두 관성좌표계 $S$와 $S'$으로부터 각각 $(x,y,z,t)$, $(x',y',z',t')$만큼 떨어진 지점에
사건 $A$가 발생하였다고 하자. 사건 $A$의 발생이 $S$의 원점에 위치한 관측자에 도달하는 시간
$t$와 $S'$의 원점에 위치한 관측자에 도달하는 시간 $t'$는 다음과 같다.
\begin{align}\label{eq:2-1}
  c^2t^2 = x^2+y^2+z^2,\,\,\,  c^2t'^2 = x'^2+y'^2+z'^2.
\end{align}
이 때 $y=y'$, $z=z'$이다. 관성좌표계 $S'$이 $S$에 대해 $x$축 방향으로
상대속도 $u$로 움직인다고 가정하자. 로렌츠 변환은 다음과 같이 $(x,t)$와 $(x',t')$의
변환을 가정하면서 시작한다.
\begin{align}\label{eq:2-2}
  \begin{split}
    \begin{cases}
      x'=\gamma(x-ut) \\
      t'=a(t-bx)
    \end{cases}
  \end{split}
\end{align}
식~\eqref{eq:2-1}에 대입하면 $y=y'$, $z=z'$이므로
\begin{align}
  x'^2+y'^2+z'^2 = \gamma^2(x-ut)^2+y^2+z^2 = c^2a^2(t-bx)^2
\end{align}
으로 쓸 수 있고 오른쪽 두 항을 풀어 쓰면
\begin{align}
  \gamma^2(x^2-2uxt+u^2t^2)+y^2+z^2 = c^2a^2(t^2-2xbt+b^2x^2)
\end{align}
이다. $x^2$, $xt$, $t^2$끼리 묶으면
\begin{align}
  (\gamma^2-c^2a^2b^2)x^2-2(\gamma^2u-a^2bc^2)xt+y^2+z^2
  =(c^2a^2-\gamma^2u^2)t^2
\end{align}
로 식~\eqref{eq:2-1}와 비교할 수 있는 형태을 얻는다.
\begin{align}
  \begin{split}
    \begin{cases}
      (\gamma^2-c^2a^2b^2)x^2-2(\gamma^2u-a^2bc^2)xt+y^2+z^2
      =(c^2a^2-\gamma^2u^2)t^2 \\
      x^2+y^2+z^2 =  c^2t^2
    \end{cases}
  \end{split}
\end{align}
각 계수가 같아야 하므로
\begin{align}
    \begin{cases}
    \gamma^2-c^2a^2b^2 = 1 \\
    \gamma^2u-a^2bc^2 = 0 \\
    c^2a^2-\gamma^2u^2 = c^2
  \end{cases}
\end{align}
이다. 두번째 식으로부터 얻을 수 있는
\begin{align}\label{eq:2-4}
  a^2 = \frac{\gamma^2u}{bc^2},\,\,\,b = \frac{\gamma^2u}{a^2c^2}
\end{align}
를 첫번째 식과 세번째 식에 대입해 $a$와 $b$를 소거하자. 세번째 식에 
$a^2$을 대입하면 
\begin{align}\label{eq:2-5}
  b = \frac{\gamma^2u}{c^2+u^2\gamma^2}
\end{align}
를 얻고 이를 첫번째 식에 대입하여
\begin{align}
  \gamma^2c^2=c^2+u^2\gamma^2
\end{align}
$\gamma$를 다음과 같이 구할 수 있다.
\begin{align}
  \gamma^2(c^2-u^2)=c^2
  \Longrightarrow \gamma = \frac{1}{\sqrt{1-u^2/c^2}}.
\end{align}
이를 식~\eqref{eq:2-5}에 대입하여
\begin{align}
  b = \frac{u}{c^2}
\end{align}
$b$를 구할 수 있고 식~\eqref{eq:2-4}에 대입하여
\begin{align}
  a^2=\gamma^2\Longrightarrow a= \gamma
\end{align}
를 얻는다. 따라서 식~\eqref{eq:2-2}는
\begin{align}
  x'=\frac{x-ut}{\sqrt{1-u^2/c^2}},\,\,\,t'=a\frac{t-ux/c^2}{\sqrt{1-u^2/c^2}}
\end{align}
인 로렌츠 변환이 된다.


\vspace{1cm}
\noindent {\bf 문제 3 [30pt].}
정지길이가 30.0 cm인 막대자가 진행방향인 $x$축에 대해 $30.0^\circ$
기울어진 채 $x$방향으로 $v=0.990c$의 속도로 움직이고 있다. 정지해 있는
관찰자가 측정한 이 자의 길이는 얼마인가? 

\noindent {\bf 풀이. }
막대자가 기울어져 있으므로 길이 수축은 $x$축 방향으로의 길이에만 일어난다.
막대자의 $x$축 방향 길이 $L_x$는
\begin{align}
  L_x = L\cos30.0^\circ
\end{align}
이고 수축이 일어난 $x$축 방향 길이 $L_x'$는
\begin{align}
  L_x' = \sqrt{1-\frac{v^2}{c^2}}L_x
\end{align}
이다. $L_x=$30.0 cm, $v=0.990c$이므로
\begin{align}
  L_x' = \sqrt{1-(0.990)^2}(30.0~\mathrm{cm})\cos30.0^\circ
  =3.67~\mathrm{cm}
\end{align}
로 막대자의 수축된 $x$축 방향 길이를 구할 수 있다. 따라서 정지해 있는 관찰자가 측정한 
막대자의 총 길이 $L'$은
\begin{align}
  {L'} = \sqrt{{L'}_x^2+{L'}_y^2} 
  =\sqrt{(3.67~\mathrm{cm})^2+((30.0~\mathrm{cm})\sin30.0^\circ)^2} 
  =15.4~\mathrm{cm}
\end{align}
$15.4$ cm이다.

\vspace{1cm}
\noindent {\bf 문제 4 [30pt].}
정지상태에서 중간자는 생성 후 2.0 $\mu s$만에 소멸된다. 이 중간자가
실험실에서 $0.990c$의 속력으로 움직이면, 실험실 시계로 중간자 수명은
얼마인가? 

\noindent {\bf 풀이. }
중간자가 소멸되는데 걸리는 시간을 $t$라 하면 시간 팽창을 고려하여 실험실에서 관측한
시간 $t'$를 구할 수 있다. 시간 팽창에 의해
\begin{align}
  t' = \frac{t}{\sqrt{1-u^2/c^2}}
\end{align}
$t=2.0~\mu s$이고 $u=0.990c$이므로
\begin{align}
  t' = \frac{2.0~\mu s}{\sqrt{1-(0.990)^2}}
  =14~\mu s
\end{align}
실험실 시계로 관측한 중간자 수명은 14 $\mu s$이다.
\end{document}

