%\documentclass[preprint,tightenlines,showpacs,showkeys,floatfix,
%nofootinbib,superscriptaddress,fleqn]{revtex4} 
\documentclass[tightenlines,floatfix,nofootinbib,superscriptaddress,fleqn]{revtex4} 
%\documentclass[aps,epsfig,tightlines,fleqn]{revtex4}
\usepackage{kotex}
\usepackage[HWP]{dhucs-interword}
\usepackage[dvips]{color}
\usepackage{graphicx}
\usepackage{bm}
%\usepackage{fancyhdr}
%\usepackage{dcolumn}
\usepackage{defcolor}
\usepackage{amsmath}
\usepackage{amsfonts}
\usepackage{amssymb}
\usepackage{amscd}
\usepackage{amsthm}
\usepackage[utf8]{inputenc}
%\pagestyle{fancy}

\begin{document}

\title{\Large 2022년 2학기 물리학 II}
\author{김현철\footnote{Office: 5S-436D (면담시간 매주
    수요일-16:15$\sim$19:00)}} 
\email{hchkim@inha.ac.kr}
\affiliation{Hadron Theory Group, Department of Physics,
  Inha  University, Incheon 22212, Republic of Korea }
\date{Autumn Semester, 2022}

\maketitle

\section*{\large Quiz 17}
\noindent {\bf 문제 1 [30pt].}
$S$ 관성좌표계에서 $\vec{v}$의 속도로 움직이고 있는 입자가 있다.
\begin{itemize}
\item[(가)]  $S$ 좌표계에 대해 $x$ 방향으로 상대속도 $u$로 움직이고
  있는 관성좌표계 $S'$에서 이 입자의 속도의 각 성분을 구하여라. 
\item[(나)] $v_x=c$일 때, $v_x'=c$임을 보여라. 
\end{itemize}
\newpage

\noindent {\bf 문제 2 [70pt].}
로렌츠 변환을 자세히 유도하여라. 즉,
\begin{align}
  \label{eq:1}
  x' = \frac{x-ut}{\sqrt{1-u^2/c^2}},\;\;\; y'=y,\;\;\;z'=z,\;\;\;
  t' = \frac{t-ux/c^2}{\sqrt{1-u^2/c^2}}
\end{align}
임을 자세히 보여라. 
\newpage

\noindent {\bf 문제 3 [30pt].}
정지길이가 30.0 cm인 막대자가 진행방향인 $x$축에 대해 $30.0^\circ$
기울어진 채 $x$방향으로 $v=0.990c$의 속도로 움직이고 있다. 정지해 있는
관찰자가 측정한 이 자의 길이는 얼마인가? 

\newpage

\noindent {\bf 문제 4 [30pt].}
정지상태에서 중간자는 생성 후 2.0 $\mu s$만에 소멸된다. 이 중간자가
실험실에서 $0.990c$의 속력으로 움직이면, 실험실 시계로 중간자 수명은
얼마인가? 

\end{document}

