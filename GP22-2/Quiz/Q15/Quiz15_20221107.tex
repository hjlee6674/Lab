%\documentclass[preprint,tightenlines,showpacs,showkeys,floatfix,
%nofootinbib,superscriptaddress,fleqn]{revtex4} 
\documentclass[tightenlines,floatfix,nofootinbib,superscriptaddress,fleqn]{revtex4} 
%\documentclass[aps,epsfig,tightlines,fleqn]{revtex4}
\usepackage{kotex}
\usepackage[HWP]{dhucs-interword}
\usepackage[dvips]{color}
\usepackage{graphicx}
\usepackage{bm}
%\usepackage{fancyhdr}
%\usepackage{dcolumn}
\usepackage{defcolor}
\usepackage{amsmath}
\usepackage{amsfonts}
\usepackage{amssymb}
\usepackage{amscd}
\usepackage{amsthm}
\usepackage[utf8]{inputenc}
%\pagestyle{fancy}

\begin{document}

\title{\Large 2022년 2학기 물리학 II}
\author{김현철\footnote{Office: 5S-436D (면담시간 매주
    수요일-16:15$\sim$19:00)}} 
\email{hchkim@inha.ac.kr}
\affiliation{Hadron Theory Group, Department of Physics,
  Inha  University, Incheon 22212, Republic of Korea }
\date{Autumn Semester, 2022}

\maketitle

{\color{red} {\bf Due date:} 2022년 11월 07일  15:00-16:15 }
\vspace{1.cm}

\noindent \textbf{ 주의: \color{blue} 단 한 번의 부정행위도 절대
  용납하지 않습니다. 적발 시, 학점은 F를 받게 됨은 물론이고,
  징계위원회에 회부합니다. One strike out임을 명심하세요.} 
\\
\\

{\bf 학번:} \hspace{4cm}
{\bf 이름:} 

\section*{\large Quiz 15}
\noindent {\bf 문제 1 [20pt].}
어떤 렌즈 앞에 물체를 놓았더니 4배 크기의 실상이 생겼고, 이 물체를
렌즈에서 4.00 cm 더 멀리 하였더니 2배 크기의 실상이 생겼다. 이 렌즈의
초점거리는 얼마인가? 
\newpage

\noindent {\bf 문제 2 [30pt].}
굴절율이 1.62인 유리로 만든 얇은 렌즈가 있다. 이 렌즈의 한쪽 면은
오목하며 곡률 반지름이 100 cm이고, 다른 한쪽 면은 볼록하며
곡률 반지름이 40.0 cm이다. 이 렌즈의 초점거리를 구하여라. 
\newpage

\noindent {\bf 문제 3 [50pt].}
너비 $a$와 간격이 $d$인 이중실틈(doubleslit)에 파장이 $\lambda$인
결맞은 빛을 비춘다. 실틈에서부터 $D$만큼 떨어진 화면에 나타나는 밟은
간섭무늬 사이의 거리는 얼마인가?

\newpage

\noindent {\bf 문제 4 [50pt].}
폭이 $a$인 단일슬릿에서부터 $L$만큼 떨어진 곳에 스크린을
두었다. 단일슬릿 앞에서 파장이 $\lambda$인 빛을 쪼였다. $a\ll L$
이라고 하자. 만약에 회절 무늬에서 어두운 부분을 나타내는 두 최소점
$m=m_1$과 $m=m_2$ 사이의 거리를 $\Delta y$라고 둔다면, 이 슬릿의 폭
$a$는 얼마인가? 

\end{document}

