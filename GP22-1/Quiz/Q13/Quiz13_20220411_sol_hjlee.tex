\documentclass[floatfix,nofootinbib,superscriptaddress,fleqn]{revtex4-2} 
%\documentclass[aps,epsfig,tightlines,fleqn]{revtex4}
\usepackage[utf]{kotex}
\usepackage[HWP]{dhucs-interword}
\usepackage[dvips]{color}
\usepackage{graphicx}
\usepackage{bm}
%\usepackage{fancyhdr}
%\usepackage{dcolumn}
\usepackage{defcolor}
\usepackage{amsmath}
\usepackage{amsfonts}
\usepackage{amssymb}
\usepackage{amscd}
\usepackage{amsthm}
\usepackage[utf8]{inputenc}
 \usepackage{setspace}
 \usepackage{tikz}
%\pagestyle{fancy}

\begin{document}

\title{\Large 2022년 1학기 물리학 I: Quiz 12}
\author{김현철\footnote{Office: 5S-436D (면담시간 매주
    화요일-16:00$\sim$20:00)}} 
\email{hchkim@inha.ac.kr}
\author{Lee Hui-Jae} 
\email{hjlee6674@inha.edu}
\affiliation{Hadron Theory Group, Department of Physics,
Inha University, Incheon 22212, Republic of Korea }
\date{Spring semester, 2022}

\vspace{1.cm}

\maketitle


\noindent {\bf 문제 1. (20 pt)}



\noindent {\bf 풀이 : }
우선 지구 탈출속력을 구해보자. 탈출속력을 $v_e$, 지구의 질량을 $M_\oplus$, 
발사체의 질량을 $m$이라고 하면 발사체가 지구 표면에 있을 때의 역학적 에너지 $E$는,
\begin{align}
    E = -\frac{GM_\oplus m}{R_\oplus} + \frac{1}{2}mv^2_e=0.
\end{align}
따라서 탈출속력 $v_e$는 다음과 같다.
\begin{align}
    v_e = \sqrt{\frac{2GM_\oplus}{R_\oplus}}.
\end{align}
발사체가 탈출속력의 1/2배의 속력으로 지구를 떠난다면 발사체가 지구 표면에 있을 때의 
역학적 에너지 $E_1$는,
\begin{align}
    E_1 = -\frac{GM_\oplus m}{R_\oplus} + \frac{1}{8}mv_e^2
    =-\frac{3GM_\oplus m}{4R_\oplus}.
\end{align}
발사체가 최대높이 $h$에 있을 때 발사체의 속력은 0이다. 따라서 역학적 에너지 $E_2$는,
\begin{align}
    E_2 = -\frac{GM_\oplus m}{h}.
\end{align}
에너지 보존 법칙에 의해,
\begin{align}
    E_1 =E_2,\,\,\,
    -\frac{3GM_\oplus m}{4R_\oplus}=-\frac{GM_\oplus m}{h}.
\end{align}
그러므로 최대높이 $h$는 다음과 같다.
\begin{align}
    h = \frac{4}{3}R_\oplus.
\end{align}
\vspace{1.cm}

\noindent {\bf 문제 2. (40 pt)}

\noindent {\bf 풀이 : }
중첩 원리에 의해 물체 $m$에 작용하는 중력의 크기는 속이 찼을 때 
작용하는 중력에 공동 만큼의 물체에 의한 중력을 뺀 값과 같다.
물체 $M$의 밀도를 $\rho$라고 하면,
\begin{align}
    M = \frac{4}{3}R^3\rho ,\,\,\,\rho = \frac{3M}{4R^3}.
\end{align}
공동 만큼의 물체의 질량을 $M^\prime$이라고 하면,
\begin{align}
    M^\prime = \frac{4}{3}{\left(\frac{1}{2}R\right)}^3\rho=\frac{1}{8}M.
\end{align}
속이 찼을 때 중력을 $F_1$라고 하자. $F_1$은,
\begin{align}
    F_1 = \frac{GMm}{d}
\end{align}
공동 만큼의 중력을 $F_2$라고 하면 $F_2$는,
\begin{align}
    F_2 = \frac{GM^\prime m}{d-\frac{1}{2}R}=\frac{GMm}
    {8d-4R}.
\end{align}
위에서 말했듯이 실제 중력 $F$은 속이 찼을 때의 중력 $F_1$에 공동 만큼의 물체에 의한 중력 $F_2$를
뺀 값과 같다.
\begin{align}
    F= F_1 - F_2.
\end{align}
따라서,
\begin{align}
    F = GMm\left(\frac{1}{d}-\frac{1}{8d-4R}\right).
\end{align}
 G = $6.67430\times 10^{-11}\,\mathrm{N\cdot m^2/kg^2}$ 이므로 수치들을 모두 대입하고 계산해보자.
\begin{align}
    \begin{split}
        F &= \left(6.67430\times 10^{-11}\,\mathrm{N\cdot m^2/kg^2}\right)
        (2.95\,\mathrm{kg})(0.431\,\mathrm{kg})
        \left(\frac{1}{9.00\,\mathrm{cm}}
        -\frac{1}{8(9.00\,\mathrm{cm})-4(4.00\,\mathrm{cm})}\right) \\
        &=  7.91\times 10^{-10}\,\mathrm{N}.  
    \end{split}
\end{align}
물체 $m$에 미치는 중력은 $7.91\times 10^{-10}\,\mathrm{N}$이다.
\vspace{1.cm}

\noindent {\bf 문제 3. (30pt)}

\noindent {\bf 풀이 : }
상대론적 효과는 고려하지 않는다고 하자. 중성자별 질량을 $m$, 반지름을 $r$, 
별이 서로 떨어진 거리를 $d$라 하자. 한 별 위에서 관측한다고 하면 즉,
두 별 중 하나에 대해 정지해 있는 좌표계에서 서술한다고 하면 다른 별의 운동에 대해서만 생각하면 된다. 
편의상 정지해 있는 별을 별1, 움직이는 별을 별2라 하자.
별2가 가진 처음 역학적 에너지를 $E_1$라고 하면,
\begin{align}
    E_1 = -\frac{Gm^2}{d}.
\end{align}
거리가 처음의 반일 때 별2의 속력을 $v_2$라 하면 별2의 역학적 에너지 $E_2$는,
\begin{align}
    E_2 = -\frac{2Gm^2}{d}+ \frac{1}{2}mv_2^2.
\end{align}
에너지 보존 법칙에 의해 별2의 역학적 에너지는 보존되므로,
\begin{align}
    E_1=E_2,\,\,\,-\frac{Gm^2}{d}=-\frac{2Gm^2}{d}+ \frac{1}{2}mv_2^2.
\end{align}
속력 $v_2$에 대해 정리하여 속력 $v_2$를 구할 수 있다.
\begin{align}
    \begin{split}
        v_2 &= \sqrt{\frac{2Gm}{d}} 
        =\sqrt{\frac{2(6.67430\times 10^{-11}\,\mathrm{N\cdot m^2/kg^2})
        (1.0\times 10^{30}\,\mathrm{kg})}{(1.0\times 10^{10}\,\mathrm{m})}} \\
        &=1.2\times 10^5\mathrm{m/s}.
    \end{split}
\end{align}
별 사이 거리가 처음의 반일 때 별2의 속력은 $1.2\times 10^5\mathrm{m/s}$이다. 
맨 처음 두 별에게 정지해 있던 좌표계에서 관측하면 각 별의 속력은 $5.8\times 10^4\mathrm{m/s}$이다.    \\
-----------------------------------------------------------------------------   \\
충돌 직전이면 두 별 사이 거리가 $2r$인 상태이다. 이 때 별2의 속력을 $v_3$라 하면 
별2의 역학적 에너지 $E_3$는,
\begin{align}
    E_3 =-\frac{Gm^2}{2r} + \frac{1}{2}mv_3^2
\end{align}
에너지 보존 법칙에 의해 처음 역학적 에너지와 나중 역학적 에너지가 같으므로,
\begin{align}
    E_1 = E_3,\,\,\, -\frac{Gm^2}{d}=-\frac{Gm^2}{2r} + \frac{1}{2}mv_3^2
\end{align}
속력 $v_3$는 다음과 같이 구할 수 있다.
\begin{align}
    \begin{split}
        v_3&=\sqrt{Gm\left(\frac{d-2r}{rd}\right)}  \\ 
        &=\sqrt{(6.6743\times 10^{-11}\,\mathrm{N\cdot m^2/kg^2})
        (1.0\times 10^{30}\,\mathrm{kg})
        \left(\frac{(1.0\times 10^{10}\,\mathrm{m})-2(1.0\times 10^{5}\,\mathrm{m})}
        {(1.0\times 10^{5}\,\mathrm{m})(1.0\times 10^{10}\,\mathrm{m})}\right)} \\
        &=2.6\times 10^7\,\mathrm{m/s}.
    \end{split}
\end{align}
별끼리 충돌 직전일 때 별2의 속력은 $2.6\times 10^7\,\mathrm{m/s}$이다. 
맨 처음 두 별에게 정지해 있던 좌표계에서 관측하면 각 별의 속력은 $1.3\times 10^7\mathrm{m/s}$이다.
\vspace{1.cm}

\noindent {\bf 문제 4. (40pt)}


\noindent {\bf 풀이 : }
각 물체에 대한 자유 물체 다이어그램을 그리고 운동 방정식을 세워보자.
\begin{figure}[htp]
    \centering
    \begin{tikzpicture}
      \draw (-1.5,2.5) node {$m_1$ :} ;
      \draw (-2,0) -- (2,0) ;
      \draw (0,-2.5) -- (0,2.5) ;
      \draw [red,very thick,-latex] (0,0.1) -- (0,1.6) 
      node [left,black] {$T1$}; 
      \draw [blue,very thick,-latex] (0,-0.1) -- (0,-1.4) 
      node [left,black] {$m_1g$};
    \end{tikzpicture}
    \begin{tikzpicture}
      \draw (-1.5,2.5) node {$m_2$ :} ;
      \draw (-2,0) -- (2,0) ;
      \draw (0,-2.5) -- (0,2.5) ;
      \draw [red,very thick,-latex] (0,0.1) -- (0,1.8) 
      node [left,black] {$T_2$};
      \draw [blue,very thick,-latex] (0,-0.1) -- (0,-2.2) 
      node [left,black] {$m_2g$};
    \end{tikzpicture}
    \caption{자유 물체 다이어그램}
    \label{fig:2}
  \end{figure}
\begin{align}\label{eq:4-1}
    \begin{split}
        m_1:\sum F &= m_1a_1 = T_1 - m_1g,   \\
        m_2:\sum F &= m_2a_2 = m_2g - T_2.
    \end{split}
\end{align}
두 상자는 한 줄로 연결되어 있으므로,
\begin{align}\label{eq:4-2}
    a_1=a_2.
\end{align}
식 \eqref{eq:4-1}로 부터 각 상자의 가속도는 다음과 같다.
    \begin{align}\label{eq:4-3}
        a_1 = \frac{T_1-m_1g}{m_1},\,\,\,a_2 = \frac{m_2g-T_2}{m_2}
    \end{align}
\begin{itemize}
    \item[(가)] 
    상자 1은 이동거리가 $s=$ 75.0 cm, 걸린 시간 $t=$ 5.00 초, 
    초기 속력 $v_0=$ 0 m/s인 등가속도 운동을 하였으므로 다음과 같이 가속도를 구할 수 있다.
    \begin{align}\label{eq:4-4}
        s = \frac{1}{2}a_1t^2,\,\,\,a_1 = \frac{2s}{t^2}.
    \end{align}
    따라서 상자1 의 가속도 $a_1$은,
    \begin{align}
        \begin{split}
            a_1 &= \frac{2(75.0\,\mathrm{cm})}{(5.00\,\mathrm{s})^2} \\
            &= 6.00\,\mathrm{cm/s^2} = 6.00\times 10^{-2}\,\mathrm{m/s^2}.
        \end{split}
    \end{align}
    \item[(나)]
    식 \eqref{eq:4-1}으로 부터 다음과 같이 장력을 구할 수 있다.
    \begin{align}\label{eq:4-5}
       T_1 =m_1\left( a_1+g \right),\,\,\,
       T_2 =m_2\left( g-a_2 \right).
    \end{align}
    $m_1 = 460\,\mathrm{g}$이므로,
    \begin{align}
        \begin{split}
            T_1 &= (460\,\mathrm{g})
            ((6.00\times 10^{-2}\,\mathrm{m/s^2})+(9.80\,\mathrm{m/s^2})) \\
            &=(0.460\,\mathrm{kg})
            (9.86\,\mathrm{m/s^2}) \\
            &=4.54\,\mathrm{N}.
        \end{split}
    \end{align} 
    $m_2 = 500\,\mathrm{g}$이므로,
    \begin{align}
        \begin{split}
            T_2 &= (500\,\mathrm{g})
            ((9.80\,\mathrm{m/s^2})-(6.00\times 10^{-2}\,\mathrm{m/s^2})) \\
            &=(0.500\,\mathrm{kg})
            (9.74\,\mathrm{m/s^2}) \\
            &=4.87\,\mathrm{N}.
        \end{split}
    \end{align} 
    \item[(다)] 줄의 가속도를 $a$라고 하자. 
    줄이 미끄러지지 않으므로 도르래는 줄과 함께 회전하므로 도르래의 각가속도 $\alpha$는 
    다음과 같이 표현할 수 있다.
    \begin{align}\label{eq:4-6}
        \alpha=\frac{a_1}{R}.
    \end{align}
    따라서,
    \begin{align}
        \begin{split}
            \alpha&=\frac{(6.00\times 10^{-2}\,\mathrm{m/s^2})}
            {(5.00 \,\mathrm{cm})}
            = \frac{(6.00\times 10^{-2}\,\mathrm{m/s^2})}
            {(5.00\times 10^{-2} \,\mathrm{m})} \\
            &= 1.20\,\mathrm{rad/s^2}.
        \end{split}        
    \end{align}
    \item[(라)] 도르래에 감긴 줄에 작용하는 합력을 구하자. 자유 물체 다이어그램을 그려보면 다음과 같다.
    \begin{figure}[htp]
        \centering
        \begin{tikzpicture}
          \draw (-1.5,2.5) node {줄에 작용하는 합력 :} ;
          \draw (-3,0) -- (3,0) ;
          \draw (0,-1.5) -- (0,1.5) ;
          \draw [red,very thick,-latex] (0.1,0) -- (2.4,0) 
          node [above,black] {$T_2$};
          \draw [red,very thick,-latex] (-0.1,0) -- (-1.6,0) 
          node [above,black] {$T_1$};
        \end{tikzpicture}
        \caption{자유 물체 다이어그램}
    \end{figure}
    줄에 작용하는 합력 $T$는,
    \begin{align}\label{eq:4-7}
        T = T_2-T_1.
    \end{align}
    도르래에 작용하는 돌림힘 $\tau$은,
    \begin{align}
        \tau  = TR.
    \end{align}
    돌림힘과 각가속도를 알고 있으므로 도르래의 회전관성을 구할 수 있다. 도르래의 회전관성 $I$는,
    \begin{align}
        \tau = I\alpha,\,\,\, I=\frac{\tau}{\alpha}=\frac{TR}{\alpha}.
    \end{align}
    식 \eqref{eq:4-5}, \eqref{eq:4-6}과 식 \eqref{eq:4-7}로 부터,
    \begin{align}
        \begin{split}
            I &= \frac{(T_2-T_1)R^2}{a_1} =  \frac{(m_2\left( g-a_2 \right)
            -m_1\left( a_1+g \right))R^2}{a_1}  
            = \frac{((m_2-m_1)g-(m_2+m_1)a_1)R^2}{a_1} \\
            &=  \frac{(((500\,\mathrm{g})-(460\,\mathrm{g}))(9.80\,\mathrm{m/s^2})
            -((500\,\mathrm{g})+(460\,\mathrm{g}))
            (6.00\times 10^{-2}\,\mathrm{m/s^2}))(5.00\times 10^{-2}\,\mathrm{m})^2}
            {6.00\times 10^{-2}\,\mathrm{m/s^2}}    \\
            &= 13.9\,\mathrm{g\cdot m^2}=1.39\times 10^{-2}\,\mathrm{kg\cdot m^2}.
        \end{split}
    \end{align}
    도르래의 회전관성은 $1.39\times 10^{-2}\,\mathrm{kg\cdot m^2}$이다.
\end{itemize}

\end{document}