\documentclass[floatfix,nofootinbib,superscriptaddress,fleqn]{revtex4-2}  
%\documentclass[aps,epsfig,tightlines,fleqn]{revtex4}
\usepackage[utf]{kotex}
\usepackage[HWP]{dhucs-interword}
\usepackage[dvips]{color}
\usepackage{graphicx}
\usepackage{bm}
%\usepackage{fancyhdr}
%\usepackage{dcolumn}
\usepackage{defcolor}
\usepackage{amsmath}
\usepackage{amsfonts}
\usepackage{amssymb}
\usepackage{amscd}
\usepackage{amsthm}
\usepackage[utf8]{inputenc}
 \usepackage{setspace}
 \usepackage{tikz}
 \usetikzlibrary{decorations,decorations.markings,decorations.text}
%\pagestyle{fancy}

\begin{document}
\pgfkeys{/pgf/decoration/.cd,
distance/.initial=10pt
}  

\pgfdeclaredecoration{add dim}{final}{
\state{final}{% 
\pgfmathsetmacro{\dist}{5pt*\pgfkeysvalueof{/pgf/decoration/distance}/abs(\pgfkeysvalueof{/pgf/decoration/distance})} 
          \pgfpathmoveto{\pgfpoint{0pt}{0pt}}             
          \pgfpathlineto{\pgfpoint{0pt}{2*\dist}}   
          \pgfpathmoveto{\pgfpoint{\pgfdecoratedpathlength}{0pt}} 
          \pgfpathlineto{\pgfpoint{(\pgfdecoratedpathlength}{2*\dist}}
           \pgfusepath{stroke} 
          \pgfsetdash{{0.1cm}{0.1cm}{0.1cm}{0.1cm}}{0cm}     
          \pgfsetarrowsstart{latex}
          \pgfsetarrowsend{latex}  
          \pgfpathmoveto{\pgfpoint{0pt}{\dist}}
          \pgfpathlineto{\pgfpoint{\pgfdecoratedpathlength}{\dist}} 
          \pgfusepath{stroke} 
          \pgfsetdash{}{0pt}
          \pgfpathmoveto{\pgfpoint{0pt}{0pt}}
          \pgfpathlineto{\pgfpoint{\pgfdecoratedpathlength}{0pt}}
}}

\tikzset{dim/.style args={#1,#2}{decoration={add dim,distance=#2},
                decorate,
                postaction={decorate,decoration={text along path,
                                                 raise=#2,
                                                 text align={align=center},
                                                 text={#1}}}}}


\title{\Large 2022년 1학기 물리학 I: Quiz 16}
\author{김현철\footnote{Office: 5S-436D (면담시간 매주
    화요일-16:00$\sim$20:00)}} 
\email{hchkim@inha.ac.kr}
\author{Lee Hui-Jae} 
\email{hjlee6674@inha.edu}
\affiliation{Hadron Theory Group, Department of Physics,
Inha University, Incheon 22212, Republic of Korea }
\date{Spring semester, 2022}

\vspace{1.cm}

\maketitle

\noindent {\bf 문제 1. (40 pt)}
높이가 5 m인 큰 수족관에 2.00 m의
깊이로 민물이 채워져 있다. 폭이 8.00 m인 수족관의 한쪽 벽은 두꺼운
플라스틱으로 만들어져 있다. 물을 더 채워 수심이 4.00 m가 되었다면,
벽에 가해지는 전체 힘은 얼마나 증가하겠는가? 

\noindent {\bf 풀이 : }
벽이 받는 전체 힘을 $F$, 민물에 의한 압력을 $P$, 
민물과 벽이 닿는 면적을 $A$라 하자.
$F$는
\begin{align}
  F = \int P\,dA,\,\,\,P = \rho g h + P_0
\end{align}
이다. $h$는 민물의 깊이이고 $P_0$는 대기압이다. 대기압은 벽에 가해지는 전체 힘의
증가에 영향을 주지 않으므로 무시할 수 있다.
폭을 $a$라 하면 미소면적 $dA$는 다음과 같이 표현할 수 있다.
\begin{align}
  dA = a\,dh
\end{align}
처음 높이를 $h_1$라 하면 처음 힘 $F_1$은
\begin{align}
  F_1 = \int_0^{h_1}a\rho g h\,dh = \frac{1}{2}a\rho g h^2_1
\end{align}
이다. 같은 방법으로
나중 높이 $h_2$라 하면 나중 힘 $F_2$은
\begin{align}
  F_2 = \int_0^{h_2}a\rho g h\,dh = \frac{1}{2}a\rho g h^2_2
\end{align}
이다. 수심이 바뀌었을 때 힘의 변화량을 $\Delta F$은 다음과 같이 구할 수 있다,
\begin{align}
  \Delta F = F_2 - F_1 = \frac{1}{2}a\rho g \left(h^2_2-h^2_1\right).
\end{align}
수치를 넣어 계산하면,
\begin{align}
  \begin{split}
    \Delta F &=\frac{1}{2}(8.00\,\mathrm{m})
    (998\,\mathrm{kg/m^3}) (9.80\,\mathrm{m/s^2}) 
    \left((4.00\,\mathrm{m})^2-(2.00\,\mathrm{m})^2\right)  \\
    &= 4.69\times 10^5\,\mathrm{N}
  \end{split}
\end{align}
이다. 벽에 가해지는 전체 힘은 $4.69\times 10^5\,\mathrm{N}$만큼 증가한다.
\vspace{1.cm}


\noindent {\bf 문제 2. (100 pt) {\color{red} 난이도 상}:}
그림~\ref{fig:1}에서처럼 폭이 $W=314$
m인 댐의 상류 쪽에 깊이 $S=35.0$ m만큼 물이 차 있다. 
\begin{figure}[htp]
  \centering
  \includegraphics[scale=0.65]{Qfig16-1.pdf}
    \caption{문제 2}
  \label{fig:1}
\end{figure}
\begin{itemize}
\item[(가)] 물의 계기압력(gauge pressure of the water)으로부터 댐에
  가해지는 알짜힘을 구하라.
\item[(나)] 그 힘에 의해 생기는 알짜 돌림힘을 점 $O$를 지나고 댐의
  폭에 형행인 축에 대해 구하여라.
\item[(다)] 돌림힘의 모멘트팔을 구하여라.   
\end{itemize}

\noindent {\bf 풀이 : }
\begin{itemize}
  \item[(가)]
  깊이 $h$에서 물의 계기압력 $P_{\mathrm{gauge}}$는
  \begin{align}
    P_{\mathrm{gauge}} = \rho g h
  \end{align}
  이다.
  물과 댐이 닿는 면적을 $A$, 계기압력에 의해 댐에 가해지는 힘을 
  $F_{gauge}$라 하면 $F_{gauge}$는
  \begin{align}
    F_{\mathrm{gauge}} = \int P_{\mathrm{gauge}}\,dA
  \end{align}
  이다. 폭을 $W$라 하면 미소면적 $dA$는
  \begin{align}
    A = Wh,\,\,\, dA = Wdh
  \end{align}
  이고 힘 $F_{gauge}$는
  \begin{align}\label{eq:2-1}
    F_{\mathrm{gauge}} = \int^S_0 P_{\mathrm{gauge}}\,dA
    =\int^S_0 \rho g W h\,dh = \frac{1}{2}\rho g WS^2
  \end{align}
  이다. 따라서,
  \begin{align}
    \begin{split}
      F_{\mathrm{gauge}} &= \frac{1}{2}(998\,\mathrm{kg/m^3})
      (9.80\,\mathrm{m/s^2}) 
      (314\,\mathrm{m})(35.0\,\mathrm{m})^2  \\
      &= 1.88\times 10^9\,\mathrm{N}
    \end{split}
  \end{align}
  을 얻는다. 계기압력으로부터 댐에 가해지는 알짜힘은 
  $ 1.88\times 10^9\,\mathrm{N}$이다.
  \item[(나)] 
  물에 닿는 면적에 대해 힘이 연속적으로 작용하므로 
  각 높이에 작용하는 미소힘에 의한 미소돌림힘 $d\tau$를 찾으면
  \begin{align}
    d\tau = r\sin\theta dF,\,\,\,dF=\rho g Wh\,dh
  \end{align}
  이고 $r\sin\theta=S-h$ 이므로 
  \begin{align}
    d\tau = r\sin\theta dF
    =\rho g W (S-h)h \,dh
  \end{align}
  이다. 
  \begin{figure}[htbp]
    \begin{tikzpicture}
    \centering
    \coordinate (O) at (4,0);
    \coordinate (A) at (0,0);
    \coordinate (A') at (-1,0);
    \coordinate (S') at (-1,3);
    \coordinate (h) at (0,2);
    \coordinate (h') at (-1,2);
    \coordinate (S) at (0,3);
    \draw[fill] (O) circle (2pt) node [right,black] {$O$};
    \draw[dim={$l$,-15pt}]  (A) --  (O); 
    \draw[dim={$r$, 15pt}]  (h) --  (O);
    \draw[]  (S) -- (A);
    \draw[densely dotted]  (S) -- (S');
    \draw[densely dotted]  (h) -- (h');
    \draw[dim={$S$,-15pt}]  (S) -- (A);
    \draw[dim={$h$, 10pt}]  (h') --  (S');
    \draw[] (3.5,0) arc(180:166:1) 
    (3,0) node[above] {$\theta$};
    \draw[red,very thick,-latex] (h)-- +(1,0)
    node [above,black] {$dF$};
    \end{tikzpicture}
  \end{figure}

  따라서 총 돌림힘은 다음과 같이 구할 수 있다.
  \begin{align}
    \begin{split}\label{eq:2-2}
      \tau &=\int_0^S \rho g W (S-h)h \,dh 
      = \rho g W\left(\frac{1}{2}S^3-\frac{1}{3}S^3\right)
      =\frac{1}{6}\rho g WS^3 \\
      &=\frac{1}{6}(998\,\mathrm{kg/m^3})
      (9.80\,\mathrm{m/s^2}) 
      (314\,\mathrm{m})(35.0\,\mathrm{m})^3  \\
      &= 2.19\times 10^{10}\,\mathrm{N\cdot m}.
    \end{split}
  \end{align}
  총 돌림힘은 $2.19\times 10^{10}\,\mathrm{N\cdot m}$이다.
  \item[(다)] 모멘트팔의 길이를 $l$이라고 하면,
  \begin{align}
    \tau = lF_{gauge},\,\,\,l = \frac{\tau}{F_{gauge}}
  \end{align}
  이다. 식~\eqref{eq:2-1},~\eqref{eq:2-2}로부터 $l$은
  \begin{align}
    l = \frac{S}{3} = \frac{35.0\,\mathrm{m}}{3}
    =11.7\,\mathrm{m}
  \end{align}
  이다. 즉, 돌림힘의 모멘트팔은 $11.7\,\mathrm{m}$이다.
\end{itemize}

\vspace{1.cm}


\noindent {\bf 문제 3. (60pt)}
그림~\ref{fig:2}처럼 용수철 상수가 $3.00\times 10^4$ N/m인 용수철이
단단한 들보와 유압지렛대의 출력 피스톤 사이에 연결되어 있다.
\begin{figure}[htp]
  \centering
  \includegraphics[scale=0.65]{Qfig16-2.pdf}
    \caption{문제 3}
  \label{fig:2}
\end{figure}
 질량을 무시할 수 있는 빈 통이 입력 피스톤 위에 놓여 있다. 입력 피스톤의
단면적은 $A_i$이고 출력 피스톤의 단면적은 $18.0 A_i$이다. 처음에
용수철은 늘어나지 않은 길이이다. 천천히 빈 통에 모래를 부어서 몇 kg을
넣어야 용수철이 $5.00$ cm만큼 수축하겠는가? 

\noindent {\bf 풀이 : }
모래를 $m$만큼 넣었을 때 입력 피스톤에 작용하는 압력 $P_{in}$은
 \begin{align}
   P_{in} =\frac{F}{A} =\frac{mg}{A_i}
 \end{align}
 이다. 유체가 출력 피스톤에 작용하는 압력 $P_{out}$에 대해 생각해보자. 
 출력 피스톤이 올라가 
 스프링을 압축시키면 스프링은 출력 피스톤에 복원력 $F_{out}$을 가한다. 
 출력 피스톤이 스프링을 압축시키는 힘과
 스프링의 복원력이 평형을 이룰 때 출력 피스톤과 스프링은 정지한다. 
 이 때 $P_{out}$과 $F_{out}$는 다음 관계에 있다.
 \begin{align}
   P_{out} = \frac{F_{out}}{18A_i}
 \end{align}
 또한 스프링에 의한 복원력 $F_{out}$은
 \begin{align}
   F_{out} = kx = kh_2,
 \end{align}
 이고 압력 $P_{out}$은
 \begin{align}
  P_{out} = \frac{kh_2}{18A_i}
 \end{align}
 이다. 압력은 유체의 어디에서나 같으므로 $P_{in}=P_{out}$이다. 따라서,
 \begin{align}
  \frac{mg}{A_i} = \frac{kh_2}{18A_i},\,\,\,m=\frac{kh_2}{18g}
 \end{align}
 우리가 구하고자 하는 것은 $h_2 = 5.00\,\mathrm{cm}$일 때 
 모래의 질량 $m$이므로
 \begin{align}
   m = \frac{(3.00\times 10^4\,\mathrm{N/m})
   (5.00\times 10^{-2}\,\mathrm{m})}
   {18(9.80\,\mathrm{m/s^2})}
   =8.50\,\mathrm{kg}
 \end{align}
 를 얻을 수 있다.
\vspace{1.cm}


\noindent {\bf 문제 4. (40pt)}
속이 비어있는 쇠공이 물에 거의 잠긴
채 떠 있다. 바깥 반지름이 60.0 cm이고, 쇠의 밀도가
$7.87\,\mathrm{g/cm^3}$일 때 안쪽 반지름을 구하여라.

\noindent {\bf 풀이 : }
쇠공은 자신이 물속에서 차지한 부피를 가진 물의 무게를 부력으로 받는다.
쇠공이고 물에 거의 잠겨있으므로 바깥쪽 반지름을 $b$라고 하면 
부력 $F_b$는
\begin{align}
  F_b = V\rho_0 g =\frac{4}{3}\pi b^3\rho_0 g
\end{align}
이다. $\rho_0$는 물의 밀도이다. 
쇠공의 안쪽 반지름을 $a$, 쇠의 밀도를 $\rho$라 하면 쇠공이 받는 중력 $F_g$는
다음과 같다.
\begin{align}
  F_g = mg = \frac{4}{3}\pi(b^3-a^3)\rho g.
\end{align}
쇠공이 정지해있는 이유는
쇠공의 위로 작용하는 부력 $F_b$와 아래로 작용하는 중력 $F_g$가 같아 
평형을 이루기 때문이다. 즉, $F_b = F_g$이고
\begin{align}
  \frac{4}{3}\pi b^3\rho_0 g = \frac{4}{3}\pi(b^3-a^3)\rho g
\end{align}
이다. 안쪽 반지름을 구하기 위해 $a$에 대해 정리하면 다음과 같다.
\begin{align}
  a^3 = \left(1-\frac{\rho_0}{\rho}\right)b^3,\,\,\,
  a = {\left(1-\frac{\rho_0}{\rho}\right)}^{\frac{1}{3}}b
\end{align}
수치를 대입해서 안쪽 반지름을 구해보자.
\begin{align}
  a = {\left(1-\frac{0.998\,\mathrm{g/cm^3}}
  {7.87\,\mathrm{g/cm^3}}\right)}^{\frac{1}{3}}
  (60.0\,\mathrm{cm})= 57.3\,\mathrm{cm}.
\end{align}
안쪽 반지름은 $57.3\,\mathrm{cm}$이다.
\end{document}