\documentclass[floatfix,nofootinbib,superscriptaddress,fleqn]{revtex4-2} 
%\documentclass[aps,epsfig,tightlines,fleqn]{revtex4}
\usepackage[utf]{kotex}
\usepackage[HWP]{dhucs-interword}
\usepackage[dvips]{color}
\usepackage{graphicx}
\usepackage{bm}
%\usepackage{fancyhdr}
%\usepackage{dcolumn}
\usepackage{defcolor}
\usepackage{amsmath}
\usepackage{amsfonts}
\usepackage{amssymb}
\usepackage{amscd}
\usepackage{amsthm}
\usepackage[utf8]{inputenc}
 \usepackage{setspace}
%\pagestyle{fancy}
\usepackage{tikz}

\begin{document}

\title{\Large 2022년 1학기 물리학 I: 제3차 시험}
\author{김현철\footnote{Office: 5S-436D (면담시간 매주
    화요일-16:00$\sim$20:00)}} 
\email{hchkim@inha.ac.kr}
\author{Lee Hui-Jae} 
\email{hjlee6674@inha.edu}
\affiliation{Hadron Theory Group, Department of Physics,
Inha University, Incheon 22212, Republic of Korea }
\date{Spring semester, 2022}


\maketitle
 
\noindent {\bf 1번 풀이 : }        
질량이 $m$인 물체가 정삼각형의 각 꼭짓점에 놓여 있을 때 계의 총 중력 퍼텐셜에너지는
\begin{align}
  E_p =-3G\frac{m^2}{a}
\end{align} 
이고 한 물체를 무한히 먼 곳에 이동시켜 계에 두 물체만 남았다면 계의 총 중력 퍼텐셜에너지는
\begin{align}
  E'_p = -G\frac{m^2}{a}
\end{align}
이다. 외부에서 해준 일은 계의 총 에너지 변화량과 같으므로
\begin{align}
  W = E'_p - E_p = 2G\frac{m^2}{a}
\end{align}
이다.
 

\vspace{0.5cm} 
\noindent {\bf 2번 풀이 : } 
줄의 장력 $T$가 도르래에 돌림힘으로 작용하여 도르래를 회전시킨다. 따라서,
\begin{align}
  TR = I\alpha = \frac{1}{2}MRa \Longrightarrow T =\frac{1}{2}Ma
\end{align}    
이고 장력과 중력이 작용하여 중력이 $M/2$인 물체를 움직이도록 한다.
도르래와 물체가 줄로 연결되어 있으므로 가속도의 크기는 같다.
\begin{align}
  \frac{1}{2}Ma = \frac{1}{2}Mg - T.
\end{align}
두 식을 연립하면 장력 $T$는
\begin{align}
  T = \frac{1}{4}Mg
\end{align}
이다.

 

\vspace{0.5cm} 
\noindent {\bf 3번 풀이 : }
가장 멀리있을 때 거리, 속력을 $r_1, v_1$이라 하고
가장 가까이 있을 때 거리, 속력을 $r_2, v_2$라고 하면,
각운동량 보존법칙에 의해 
\begin{align}
  mv_1r_1 = mv_2r_2
\end{align}
이다. $r_1 = 2r_2$이므로
\begin{align}
  2v_1r_2 = v_2r_2\Longrightarrow 2v_1 = v_2
\end{align}
이다. 따라서 최대 선속력은 최소 선속력의 2배이다.

 

\vspace{0.5cm} 

\noindent {\bf 4번 풀이 : }        
팽창 전 별의 회전 운동에너지는
\begin{align}
  E_R = \frac{1}{2}I\omega^2
\end{align}
이다. 팽창 후 별의 회전관성은 3배 늘어난다. 외력이 작용하지 않았으므로
각운동량이 보존된다. 팽창 후의 각속도를 $\omega'$이라고 하면 각운동량 보존 법칙에 의해
\begin{align}
  I\omega = 3I\omega' \Longrightarrow \omega' =\frac{1}{3}\omega
\end{align}
이다. 따라서 팽창 후의 별의 회전 운동에너지 $E'_R$은
\begin{align}
  E'_R=\frac{3}{2}I\omega'^2=\frac{1}{6}I\omega^2
\end{align}
이다. 따라서 팽창 후 회전 운동에너지는 팽창 전 회전 운동에너지의 $\frac{1}{3}$배가 된다.

 

\vspace{0.5cm} 
\noindent {\bf 5번 풀이 : }
액체의 밀도를 $\rho_l$이라 하면 잠긴 부피 만큼의 물의 질량과
물체의 질량이 같으므로
\begin{align}
  L \rho_l = L_0 \rho
\end{align}
이고 $\rho_l$은
\begin{align}
  \rho_l = \frac{L_0}{L}\rho
\end{align}
이다.
\vspace{0.5cm}
 


\noindent {\bf 6번 풀이 : }        
베르누이 방정식에 의해
\begin{align}
  P_1 + \frac{1}{2}\rho v_1^2
  =  P_2 + \frac{1}{2}\rho v_2^2
\end{align}
이므로 $P_2$는
\begin{align}\label{eq:6-1}
  P_2 = P_1 + \frac{1}{2}\rho\left(v_1^2-v_2^2\right)
\end{align}
이다. 연속방정식을 이용해 $v_2$를 구해보자. 
\begin{align}
  A_1 v_1 = A_2 v_2
\end{align}
이고 $A_1$의 반지름이 $A_2$의 반지름보다 2배 더 길다. 따라서 $A_1 = 4A_2$이고
연속방정식으로부터
\begin{align}
  4v_1 = v_2
\end{align}
 임을 알 수 있다. 이를 식~\eqref{eq:6-1}에 대입하면
 \begin{align}
  P_2 = P_1 + \frac{1}{2}\rho\left(v_1^2-16v_1^2\right)
  =P_1 - \frac{15}{2}\rho v_1^2
 \end{align}
을 얻는다.

\vspace{0.5cm} 
\noindent {\bf 7번 풀이 : }        
막대의 회전에 대한 운동방정식은
\begin{align}
  I\alpha = -MgL\sin\theta \approx -MgL\theta
\end{align}
이다. 각 $\theta$가 매우 작을 때 근사를 취하였다. 따라서
\begin{align}
  \frac{1}{3}ML^2 \frac{d^2\theta}{dt^2} = -MgL\theta \Longrightarrow
  \frac{d^2\theta}{dt^2} = -\frac{3g}{L}\theta = -\omega^2\theta
\end{align}
이므로 각진동수 $\omega$는
\begin{align}
  \omega = \sqrt{\frac{3g}{L}}
\end{align}
임을 알 수 있다. 주기 $T$는
\begin{align}
  T = \frac{2\pi}{\omega} =2\pi\sqrt{\frac{L}{3g}}
\end{align}
이다. 이로부터 주기는 질량이 2배 늘어나고 길이도 2배 늘어나면 늘어나기 전 주기의
$\sqrt{2}$배가 된다.
\vspace{0.5cm}
 

\noindent {\bf 8번 풀이 : }  
그림이 주어지지 않아 풀 수 없음.
\vspace{0.5cm}
 

\vspace{0.5cm} 
\noindent {\bf 9번 풀이 : }
정상파의 기본진동수의 파형이라면 줄의 길이 $L$은 정상파의 파장 $\lambda$의 절반이다. 즉,
\begin{align}
  L = \frac{1}{2}\lambda
\end{align}
이다. 정상파의 파수를 $k$라 하면 정상파의 파장 $\lambda$은
\begin{align}
  \lambda = \frac{2\pi}{k}
\end{align}
이므로 줄의 길이 $L$은
\begin{align}
  L = \frac{\pi}{k} = 3\,\mathrm{m}
\end{align}
이다.
 

\vspace{0.5cm} 
\noindent {\bf 10번 풀이 : }       
원래 음원의 진동수보다 관측자가 듣는 음원의 진동수가 감소한느 경우는 관측자와 음원이 서로 멀어지는
경우이다. 따라서 답은 (2), (4)번이다.
\vspace{0.5cm}
 
\noindent {\bf 11번 풀이 : }
\begin{itemize}
  \item[(a)]
  물리진자에 작용하는 돌림힘 $\tau$는
  \begin{align}
    \tau = -mgh\sin\theta
  \end{align}
  이다. $\theta$가 매우 작다면 $\sin\theta\approx\theta$로 근사할 수 있으므로
  \begin{align}
    \tau \approx -mgh\theta
  \end{align}
  이다.
  \item[(b)] 
  문제 7번과 같이 각진동수 $\omega$를 구하면
  \begin{align}
    \omega = \sqrt{\frac{mgh}{I}}
  \end{align}
  이고 주기 $T$는
  \begin{align}
    T = 2\pi\sqrt{\frac{I}{mgh}}
  \end{align}
  이다.
\end{itemize}
\vspace{0.5cm}
 
\noindent {\bf 12번 풀이 : }       
중첩된 파동의 방정식은
\begin{align}
  y(x,t) = A\sin(kx-\omega t)+A\cos(kx-\omega t)
\end{align}
이다. 삼각함수 덧셈공식 $\sin(a+b) = \sin a\cos b+\sin b\cos a$를 생각하자.
$a = kx-\omega t$, $b= \pi/4$라고 생각하면
\begin{align}
  \begin{split}
    y(x,t) &= \frac{A}{\cos\frac{\pi}{4}}\sin(kx-\omega t)\cos\frac{\pi}{4}
    +\frac{A}{\sin\frac{\pi}{4}}\cos(kx-\omega t)\sin\frac{\pi}{4}  \\
    &=\sqrt{2}A\sin\left(kx-\omega t+\frac{\pi}{4}\right)
  \end{split}
\end{align}
이므로 진폭은 $\sqrt{2}A$이다.

 

\vspace{0.5cm} 
\noindent {\bf 주관식 1번 풀이 : }
\begin{itemize}
  \item[(가)]
  그림과 같이 놓인 원판의 질량중심을 지나는 회전축에 대한 회전관성 $I_{cm}$은
  \begin{align}
    I_{cm} = \frac{1}{2}MR^2
  \end{align}
  이다. 질량중심과 회전축까지의 거리 $h$는 $h=R$이므로
  평행축 정리를 이용하면
  \begin{align}
    I = I_{cm}+Mh^2=\frac{1}{2}MR^2 + MR^2 = \frac{3}{2}MR^2
  \end{align}
  이다. 따라서 주어진 회전축에 대한 회전관성 $I$는 $\frac{3}{2}MR^2$이다.
  \item[(나)] 
  그림이 주어지지 않아 풀 수 없음. 
\end{itemize}


\vspace{0.5cm} 
\noindent {\bf 주관식 2번 풀이 : }
\begin{itemize}
  \item[(가)]
  그림으로부터 $ x=0$인 지점의 변위와 속도가 0초일 때 이후로 10초일 때 같아지므로 주기 $T$는 10 s이다.
  전파속도 $v$는 
  \begin{align}
    v = \frac{\lambda}{T} = \frac{0.2\,\mathrm{m}}{10\,\mathrm{s}}=0.02\,\mathrm{m/s}
  \end{align}
  이고 파동의 최대변위가 0.5 m이므로 진폭 $A$는 0.5 m이다. 그래프를 통해 
  $t=$ 2.5 s일 때 변위 $y$는
  \begin{align}
    y = (0.5\,\mathrm{m})\sin\left(\left(\frac{2\pi}{10}\right)(2.5)\right)
    =(0.5\,\mathrm{m})\sin\frac{\pi}{2} = 0.5\,\mathrm{m}
  \end{align}
  임을 알 수 있다.
  \item[(나)]
  이 파동의 완전한 함수식을 알기 위해 파수 $k$와 각진동수 $\omega$를 구해야한다. 
  파수 $k$와 각진동수 $\omega$는 다음과 같이 구할 수 있다.
  \begin{align}
    k = \frac{2\pi}{\lambda},\,\,\, \omega = \frac{2\pi}{T}.
  \end{align}
  따라서 $T$와 $\omega$는
  \begin{align}
    k = \frac{2\pi}{0.2\,\mathrm{m}}=10\pi\,\mathrm{m^{-1}},\,\,\,
    \omega = \frac{2\pi}{10\,\mathrm{s}} = \frac{\pi}{5}\,\mathrm{s^{-1}}
  \end{align}
  이고 $+x$축으로 진행하는 파동의 파동함수 $y(x,t)$는
  \begin{align}
    y(x,t) = A\sin(kx-\omega t)
    = (0.5\,\mathrm{m})\sin\left((10\pi\,\mathrm{m^{-1}})x
    -\left(\frac{\pi}{5}\,\mathrm{s^{-1}}\right) t\right)
  \end{align}
  이다.
  \item[(다)]
  진폭이 $0.5\,\mathrm{m}$, 파장이 $0.2\,\mathrm{m}$인 한 파장의 사인함수가 그려진다.
\end{itemize}
\vspace{0.5cm}
 

\vspace{0.5cm} 
\noindent {\bf 주관식 3번 풀이 : } 
Quiz 13 참조.
중첩 원리에 의해 물체 $m$에 작용하는 중력의 크기는 속이 찼을 때 
작용하는 중력에 공동 만큼의 물체에 의한 중력을 뺀 값과 같다.
물체 $M$의 밀도를 $\rho$라고 하면
\begin{align}
    M = \frac{4}{3}R^3\rho ,\,\,\,\rho = \frac{3M}{4R^3}
\end{align}
이고 공동 만큼의 물체의 질량을 $M^\prime$이라고 하면
\begin{align}
    M^\prime = \frac{4}{3}{\left(\frac{1}{2}R\right)}^3\rho=\frac{1}{8}M
\end{align}
이다. 속이 찼을 때 중력을 $F_1$라고 하자. $F_1$은
\begin{align}
    F_1 = \frac{GMm}{d^2}
\end{align}
이다. 공동 만큼의 중력을 $F_2$라고 하면 $F_2$는 다음과 같이 구할 수 있다.
\begin{align}
    F_2 = \frac{GM^\prime m}{(d-\frac{1}{2}R)^2}=\frac{GMm}
    {2(2d-R)^2}.
\end{align}
위에서 말했듯이 실제 중력 $F$은 속이 찼을 때의 중력 $F_1$에 공동 만큼의 물체에 의한 중력 $F_2$를
뺀 값과 같다.  즉,
\begin{align}
    F= F_1 - F_2
\end{align}
이다. 따라서 $m$에 미치는 중력 $F$는
\begin{align}
    F = GMm\left(\frac{1}{d^2}-\frac{1}{2(2d-R)^2}\right)
\end{align}
이다.
\vspace{0.5cm}
 

\vspace{0.5cm} 




\end{document}