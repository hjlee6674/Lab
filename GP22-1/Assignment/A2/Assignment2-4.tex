%\documentclass[preprint,tightenlines,showpacs,showkeys,floatfix,
%nofootinbib,superscriptaddress,fleqn]{revtex4} 
\documentclass[floatfix,nofootinbib,superscriptaddress,fleqn,preprint]{revtex4} 
%\documentclass[aps,epsfig,tightlines,fleqn]{revtex4}
\usepackage[utf]{kotex}
\usepackage[HWP]{dhucs-interword}
\usepackage[dvips]{color}
\usepackage{graphicx}
\usepackage{bm}
%\usepackage{fancyhdr}
%\usepackage{dcolumn}
\usepackage{defcolor}
\usepackage{amsmath}
\usepackage{amsfonts}
\usepackage{amssymb}
\usepackage{amscd}
\usepackage{amsthm}
\usepackage[utf8]{inputenc}
 \usepackage{setspace}
 \usepackage{tikz,pgfplots}
 \usepackage{tkz-euclide}
 \usetikzlibrary{angles}
%\pagestyle{fancy}

\begin{document}

% \title{\Large 2021년 1학기 고전역학 I: Quiz 5}
% \author{김현철\footnote{Office: 5S-436D (면담시간 매주
%     화요일-16:00$\sim$20:00)}} 
% \email{hchkim@inha.ac.kr}
% \affiliation{Hadron Theory Group, Department of Physics, Inha
%   University, Incheon 402-751, Republic of Korea }
% \date{Spring semester, 2021}


% \vspace{1.cm}
% \begin{abstract}
% \noindent \textbf{ {\color{red}주의}: \color{blue} 단 한 번의 부정행위도 절대
%   용납하지 않습니다. 적발 시, 학점은 F를 받게 됨은 물론이고,
%   징계위원회에 회부합니다. One strike out임을 명심하세요.}\\
% \\
% 문제는 다음 쪽부터 나옵니다.  \\ \\
% {\bf Date:} 2021년 3월 16일 (화) 12:25-13:15 
% \\
% {\bf 학번:} \hspace{4cm}
% {\bf 이름:} 

% \end{abstract}
% \maketitle

\noindent {\bf 문제 4 [20pt].} 고층 아파트에 사는 영희와 순희는 창문을 통하여 
영철이가 지면에서 던진 공이 올라가는 것과 다시 내려가는 것을 보았다. 
영희는 2.0초 간격으로, 그리고 순희는 4.0초 간격으로 공이 올라갔다 
내려가는 것을 보았다. 영희와 순희의 집은 수직으로 얼마나 떨어져 있는가?

\vspace{1cm}

\noindent{\bf 답:} 
영철이가 공을 던지는 순간 공의 수직 방향 속력을 $v_0$ 라고 하면, 
시간에 따른 공의 높이는 다음과 같은 $t$ 에 대한 2차 함수이다.
\begin{align}
  h = -\frac{1}{2}gt^2+v_0t.
\end{align} $h$ 가 $t$ 에 대한 2차 함수이므로
 공이 최대 높이에 있을 때의 시간을 $t_h$ 라 하면 다음이 성립한다.
\begin{align}\label{eq:4.1}
  \left.\frac{dh}{dt}\right|_{t=t_h} 
  = -gt_h+v_0=0,\,\,\,t_h=\frac{v_0}{g}.
\end{align}

\begin{figure}[htbp]
  \centering
  \begin{tikzpicture}[baseline]
    \begin{axis}[
      ticks=none,
    axis y line=left,
    axis x line=bottom,
    xmax=16,xmin=0,
    ymin=0,ymax=16,
    xlabel=$t$,ylabel=$h$,
    width=8cm,
    anchor=center,
    ]
    \addplot [mark=none,domain=0:20] {-0.3*x^2+4*x} ;
    \addplot [mark=none,loosely dashed,domain=0:14] {4} node[right] {$h_1$};
    \addplot [mark=none,loosely dashed,domain=0:14] {10} node[right] {$h_2$};
    \addplot [thick,densely dotted,domain=0:6] coordinates {(40/6,0)(40/6,15)}
  node[left] {$t_h$};
    \end{axis}
    \end{tikzpicture}
  \caption{ 문제 4}
  \label{pic:10}
\end{figure}
공이 최대 높이에서 떨어지는 순간을 $t=0$ 라고 생각하면
내려가는 공을 본 것이므로, 영희는 $2.0$ 초에, 
순희는 $1.0$ 초에 공을 목격한 것이다.
최고점과 영희의 집 사이 거리를 $h_1$, 순희의 집사이 거리를 $h_2$ 라고 하면,
\begin{align}
  h_1 = \frac{1}{2}gt^2=\frac{1}{2}g(2.0\,\mathrm{s})^2 ,
\end{align}
이고,
\begin{align}
  h_2 = \frac{1}{2}gt^2=\frac{1}{2}g(1.0\,\mathrm{s})^2.
\end{align}
영희의 집과 순희의 집이 떨어진 거리는 다음과 같다.
\begin{align}
  \begin{split}
    h_1-h_2 =&\frac{1}{2}g(2.0\,\mathrm{s})^2
    -\frac{1}{2}g(1.0\,\mathrm{s})^2 \\
    =& \frac{1}{2}(9.8\,\mathrm{m/s^2})(3.0\,\mathrm{s})\\
    &\thickapprox 15\,\mathrm{m}.
  \end{split}
\end{align}

영희와 순희의 집은 약 15 m 만큼 떨어져 있다.
\end{document}  