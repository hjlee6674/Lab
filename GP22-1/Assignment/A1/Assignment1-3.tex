%\documentclass[preprint,tightenlines,showpacs,showkeys,floatfix,
%nofootinbib,superscriptaddress,fleqn]{revtex4} 
\documentclass[floatfix,nofootinbib,superscriptaddress,fleqn,preprint]{revtex4} 
%\documentclass[aps,epsfig,tightlines,fleqn]{revtex4}
\usepackage[utf]{kotex}
\usepackage[HWP]{dhucs-interword}
\usepackage[dvips]{color}
\usepackage{graphicx}
\usepackage{bm}
%\usepackage{fancyhdr}
%\usepackage{dcolumn}
\usepackage{defcolor}
\usepackage{amsmath}
\usepackage{amsfonts}
\usepackage{amssymb}
\usepackage{amscd}
\usepackage{amsthm}
\usepackage[utf8]{inputenc}
 \usepackage{setspace}
%\pagestyle{fancy}

\begin{document}

% \title{\Large 2021년 1학기 고전역학 I: Quiz 5}
% \author{김현철\footnote{Office: 5S-436D (면담시간 매주
%     화요일-16:00$\sim$20:00)}} 
% \email{hchkim@inha.ac.kr}
% \affiliation{Hadron Theory Group, Department of Physics, Inha
%   University, Incheon 402-751, Republic of Korea }
% \date{Spring semester, 2021}


% \vspace{1.cm}
% \begin{abstract}
% \noindent \textbf{ {\color{red}주의}: \color{blue} 단 한 번의 부정행위도 절대
%   용납하지 않습니다. 적발 시, 학점은 F를 받게 됨은 물론이고,
%   징계위원회에 회부합니다. One strike out임을 명심하세요.}\\
% \\
% 문제는 다음 쪽부터 나옵니다.  \\ \\
% {\bf Date:} 2021년 3월 16일 (화) 12:25-13:15 
% \\
% {\bf 학번:} \hspace{4cm}
% {\bf 이름:} 

% \end{abstract}
% \maketitle

\noindent {\bf 문제 3.} 아래 계산을 유효숫자를 고려하여 계산하여라.

\vspace{1 cm}
\noindent{\bf 답:} 
\begin{itemize}
  \item[(1)] For addition, we round the result up to the largest decimal of the participating numbers
  \begin{align}
    4.8\underline{7} + 12.\underline{3} = 17.\underline{1}7 = 17.\underline{2}
  \end{align} 
  \item[(2)] For multiplication, we round the result up to the smallest significant figures of the participating numbers. Here the smallest significant figures is 2.
  \begin{align}
    0.0035 \times 0.0789 = 0.00028
  \end{align} 
  \item[(3)] For division, the rule is similar with that for multiplication. Here the smallest significant figures is 4.
  \begin{align}
    \frac{423.5}{76.265} = 5.553
  \end{align} 
  \item[(4)] Similar with number 1, however we need to match the exponent first.
  \begin{align}
    (3.13\underline{4} + 0.23\underline{4})\times 10^3 = 3.36\underline{8}\times 10^3
  \end{align} 
  \item[(5)] Do the multiplication first, after that the subtraction. The rule for subtraction is similar with that for addition. The smallest significant figures for multiplication is 3.
  \begin{align}
    25.4 \times 52.34 = 1.33\times 10^3
  \end{align} 
  Before subtracting, we need to match the exponent first
  \begin{align}
    (1.3\underline{3}-0.02745\underline{3})\times 10^3 = 1.3\underline{0} \times 10^3
  \end{align}
  Note that the last 0 of the final result is also significant figure.
\end{itemize}

\end{document}  